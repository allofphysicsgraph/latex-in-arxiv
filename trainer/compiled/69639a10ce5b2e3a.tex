\begin{equation}
  \label{modmetric}
     G^{+++}_{ij} = [{\bf r^{-1} \cdot G_c \cdot (r^{-1})^T}]_{ij}\, ,
\end{equation} 
