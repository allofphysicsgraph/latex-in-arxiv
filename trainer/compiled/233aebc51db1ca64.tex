\begin{equation}
E_{(2, 6)} = -0.95938 \,T_{25} \,,
% % \label{eq:}
\end{equation}

