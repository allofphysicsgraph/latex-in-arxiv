\begin{equation}}
\newcommand{\Tfour}[5]{
  \begin{picture}(40,35)(0,0)
  \put(-2,20){\psfig{figure=g4legs.ps,width=30pt,angle=-90}}
  \put(27,10){$\scriptstyle  #1$}
  \put(24,2){$\scriptstyle   #2$}
  \put(21,-6){$\scriptstyle  #3$}
  \put(18,-14){$\scriptstyle #4$}
  \put(0,-14){$\scriptstyle  #5$}
   \end{picture}}
%%%%%%%%%%%%%%%%%%%%%%%%%%%%%%%%%%%%%%%%%%%%%%%%%%%
\newcommand{\Tthree}[4]{
  \begin{picture}(40,30)(0,0)
  \put(-2,20){\psfig{figure=g3legs.ps,width=30pt,angle=-90}}
  \put(28,8){$\scriptstyle   #1$}
  \put(25,-3){$\scriptstyle  #2$}
  \put(22,-14){$\scriptstyle #3$}
  \put(0,-14){$\scriptstyle  #4$}
   \end{picture}}
%%%%%%%%%%%%%%%%%%%%%%%%%%%%%%%%%%%%%%%%%%%%%%%%%%%
\newcommand{\Ttwo}[3]{
  \begin{picture}(40,35)(0,0)
  \put(-2,17){\psfig{figure=g2legs.ps,width=30pt,angle=-90}}
  \put(28,0){$\scriptstyle   #1$}
  \put(22,-14){$\scriptstyle  #2$}
  \put(0,-14){$\scriptstyle  #3$}
   \end{picture}}
%%%%%%%%%%%%%%%%%%%%%%%%%%%%%%%%%%%%%%%%%%%%%%%%%%%
\newcommand{\Tone}[2]{
  \begin{picture}(25,35)(0,0)
  \put(-5,17){\psfig{figure=g1legs.ps,width=30pt,angle=-90}}
  \put(21,-14){$\scriptstyle   #1$}
  \put(-1,-14){$\scriptstyle   #2$}
   \end{picture}}
%%%%%%%%%%%%%%%%%%%%%%%%%%%%%%%%%%%%%%%%%%%%%%%%%%%%%%%%%%%%%%%%%%%
\theoremstyle{plain}% default
\newtheorem{thm}{Theorem}[section]
\newtheorem{lem}[thm]{Lemma}
\newtheorem{prop}[thm]{Proposition}
\newtheorem*{cor}{Corollary}
\newtheorem*{KL}{Klein's Lemma}

\theoremstyle{definition}
\newtheorem{defn}{Definition}[section]
\newtheorem{conj}{Conjecture}[section]
\newtheorem{exmp}{Example}[section]

\theoremstyle{remark}
\newtheorem*{rem}{Remark}
\newtheorem*{note}{Note}
\newtheorem{case}{Case}
\input Makro98.tex
\begin{document}
\begin{titlepage}
\renewcommand{\thefootnote}{\fnsymbol{footnote}}
\begin{flushright}
ICN--2002--100
\end{flushright}
\vspace{0.1in}
\LARGE
\center{HOPF ALGEBRA PRIMITIVES AND RENORMALIZATION}
%\Large
%\vspace{0.2in}
\vspace{0.2in}
\center{H. Quevedo\footnote{E-mail: {\tt quevedo@nuclecu.unam.mx}},
M. Rosenbaum\footnote{E-mail: {\tt mrosen@nuclecu.unam.mx}} and
J. David Vergara\footnote{E-mail: {\tt vergara@nuclecu.unam.mx}}}

\center{Instituto de Ciencias Nucleares, \\
Universidad Nacional Aut\'onoma de M\'exico, \\A. Postal 70-543, \\ M\'exico D.F., M\'exico.}
\normalsize
\date{\today}
\begin{abstract}
%The perturbative expansion of the transition amplitude in quantum field theory
%is closely related to the Hausdorff series and to the combinatorics of the complete
%homogeneous and power sum symmetric functionals. The analysis of these
%relations leads naturally nts of the Hausdorff series are
 The analysis of the combinatorics resulting from the
 perturbative expansion of the transition amplitude in 
quantum field theories, and the relation of this expansion
 to the Hausdorff series leads naturally to consider an 
infinite dimensional Lie
 subalgebra and the corresponding enveloping Hopf 
algebra, to which the elements
 of this series are associated. We show that in 
the context of these structures the
 power sum symmetric functionals of the perturbative 
expansion are Hopf primitives
 and that they are given by linear combinations 
of Hall polynomials, or
 diagrammatically by Hall trees. We show that 
each Hall tree corresponds to sums
 of Feynman diagrams each with the same number 
of vertices, external legs and loops.
 In addition, since the Lie subalgebra admits a 
derivation endomorphism, we also show
 that with respect to it these primitives
are cyclic vectors generated by the free propagator, 
and thus provide a recursion relation by means of 
which the (n+1)-vertex connected Green functions can 
be derived systematically from the n-vertex ones. 
By the application of an algebra homomorphism to 
these primitives and the use of the Connes-Kreimer 
twisted antipode axiom together with the Birkhoff 
algebraic decomposition, we investigate their 
relevance to the renormalization process and
arrive in a rather straightforward and heuristic 
manner at the basic equation of renormalization
theory from which the explicit relations between 
the bare and physical parameters of the theory may
be derived, and from which the corresponding 
renormalized Green functions, as well as the Renormalization
Group equations in the Mass Independent Renormalization 
Scheme result.
\noindent MSC: 16W30, 57T05, 81T15, 81T75. 

\end{abstract}
\end{titlepage}

\setcounter{footnote}{0}

\section{Introduction}
%\Large
%\vspace{0.2in}
Based on the original work of Kreimer \cite{kreimer2}, a Hopf
algebra has been developed \cite{kreimer1, kreimer2001, conkre1,
conkre2} which provides the underlying mathematics behind the
Forest Formula in the process of renormalization in perturbative
quantum field theory (PQFT). This Hopf algebra is now generally
known as the Hopf Algebra of Renormalization and has been discussed
extensively in the above given references as well as others
contained therein. Basically it can be
represented by Feynman diagrams or decorated rooted
trees, where decorations are one-particle irreducible (1PI)
divergent diagrams without subdivergences. Other Hopf algebras
related to rooted trees and
to the Hopf Algebra of Renormalization have been discussed in the
literature, such as the vector space Hopf algebra of rooted trees
of Larson and Grossman \cite{gross}. The connection of this
algebra with the algebra of Kreimer and Connes was analyzed in
\cite{pana} and more recently revised in \cite{hoff}. For a
formulation of the several Hopf algebras described in terms of
trees in terms of a single mathematical construction see van
der Laan \cite{laan}.\\

The essential point of the Kreimer-Connes formalism is that {\bf
given that a theory is renormalizable} an appropriately defined
twisted antipode, based on the minimal subtraction scheme of
renormalization, generates the counterterms corresponding to the
BPHZ Forest Formula and the antipode axiom provides a systematic
procedure for deriving the physically correct and finite
expression for a given diagram.\\
 However, the mere fact that the
twisted antipode axiom, or for that matter the Forest Formula,
provide a finite answer does not suffice to make the theory
physical. It is crucial, in order that the theory be
renormalizable, that the resulting counterterms are of the same
form as those in the original Lagrangian and that they can be
absorbed into the bare parameters of the renormalized Lagrangian
in a consistent manner. Generally this not possible, and in such a
case the theory is described as non-renormalizable.\\
Of course if
we know {\it a priori} that the theory is renormalizable then the
Hopf algebra of decorated rooted trees or of Feynman diagrams
remains most valuable both as the mathematical structure behind
the Forest Formula as well as for a systematic construction
of renormalized Green functions.\\

The purpose of this paper is to show that a combined twisted
antipode axiom and algebraic Birkhoff decomposition applied to
primitives of other Hopf algebras that appear naturally in the
early processes of PQFT, at a pre-Feynman-diagram level, may also be
related to renormalization and provide a complementary insight on
the mathematical structures underlying the mechanics of that
process.\\

A partial motivation for the present work is provided by our
previous  study \cite{chryss} of normal coordinates in the space
of undecorated rooted trees, where we discussed their relevance
to the concept of
k-primitiveness and their role in the process of renormalization.
We showed there that for undecorated ladder trees, or to that
effect for non-branched trees with only one decoration (such as in rainbow
diagrams), the renormalization of the associated normal
coordinates is a one step procedure. However, when the diagrams
for a theory involve more than one decoration (as is usually the
case) ladder normal coordinates are in general no longer primitive,
even though one can still expect them to posses a milder pole
structure than that present in the rooted tree coordinates. Also there
were some pertinent questions left open in the above cited paper
concerning the physical interpretation of those normal coordinates
and whether perturbation theory could be formulated directly in terms
of them without having to go first through the algebra of rooted trees
or Feynman diagrams.
The Hopf algebras considered here provide to some extent answers to these
questions.\\

The paper is structured as follows: In section 2 we begin with a brief review of
the main steps that lead to the
perturbative expansion of the transition amplitude, and describe its relation to
the Hausdorff series and to the associated Hopf algebras for which the power sum functionals
are primitives. In subsections 2.1.1-2.1.3 the free algebra related to these
structures is discussed and it is shown that the functional primitives are linear combinations of the Hall polynomials that generate the Lie subalgebra of our free algebra. We also show that this Lie subalgebra admits a derivation endomorphism with respect to which all the primitives are cyclic vectors generated by the free propagator. The diagrammatic representation of the primitives in terms of Hall trees is also discussed in this section and shown to give a clear image of this cyclicity and of the iteration process by means of which the $(n+1)$-vertex connected Green functions can be constructed from
the $n$-vertex ones.  Section 3 is devoted to the study of the
algebraic Birkhoff decomposition in the context of the Hopf algebras here considered, and the possible relevance of these algebras to the
the process of renormalization through the implementation
the twisted antipode axiom of Kreimer and Connes, combined with the Mass
Independent Renormalization Scheme. This procedure thus allows us
% one of the algebraic Hopf
%structures we have found can be used
to derive the expressions
that relate bare and physical parameters of the theory, the equations
of the renormalization group and the renormalized Green functions.




%We find the explicit expression that relates these primitives with the connected Green functions, and show that
%it can be used to generate $(n+1)$-vertex connected Green function
%the $n$-vertex ones.
%We discuss the relations between the complete homogeneous
%symmetric and power sum symmetric functionals which occur in this series and
%show that their non linear relation corresponds to a change to normal coordinates.
%We find that the
%different terms of this expansion can be generated recursively by means
%of an operator which we derive explicitly. Specific non linear combinations
%of these terms turn out to be primitive elements of various Hopf algebraic
%structures which are analyzed in sections 2.1.1-2.1.3. We also show that a
%derivation operator can be introduced that recursively generates all the primitive
%elements.  We find the explicit expression that relates the functionals of
%the primitive elements with the connected Green functions, and show that
%it can be used to generate $(n+1)$-vertex connected Green function
%the $n$-vertex ones.  Section 3 is
%devoted to the study of the possible relevance of the Hopf algebras
%in the process of renormalization. This is done by implementing
%the Birkhoff algebraic decomposition. We will see that for the Mass
%Independent Renormalization Scheme, one of the algebraic Hopf
%structures we have found can be used to derive the expressions
%that relate bare and physical parameters of the theory, the equations
%of the renormalization group and the renormalized Green functions.

\section{Algebraic Structures in Perturbation Quantum Field Theory}

Let us begin with a brief summary of the essential steps in PQFT
leading to the Green functions, with the dual purpose of making our presentation
self contained as well as for identifying the basic mathematical and physical
entities to which the Hopf algebras considered here are related.\\

For an arbitrary field theory the Euclidean
transition amplitude (the formulation in Minkowski space is achieved by analytic continuation) is given by
\begin{equation}
W_{E}[{\bf J}]=N\int {\mathcal D}{\bf \Phi}\;  \e^{-\int d^{d} x
[{\cal L}_0+ {\cal L}_{int} -{\bf J\cdot \Phi}]}. \label{01}
\end{equation}
Here $\Phi$ denotes the set of fields appearing in the theory, and
${\bf J}$ denotes the set of arbitrary currents introduced to drive each
field.
Using functional derivatives the amplitude (\ref{01}) can be rewritten as
\begin{equation}
W_{E}[{\bf J}]=\e^{-\langle{\cal
L}_{int}\left(\frac{\delta}{\delta {{\bf J}_{x}}}\right)\rangle
_{x}} \e^{-Z^{0}[{\bf J}]} W_{0}[0] \label{02}
\end{equation}
where ${\cal L}_{int}\left(\frac{\delta}{\delta {{\bf
J}_{x}}}\right)$ is the Lagrangian of interaction written in terms
of functional derivatives of the field currents ${\bf J}(x)$, and
$W_0[{\bf J}]=  \e^{-Z^{0}[{\bf J}]}W_0[0]$ is the free generating
functional. The symbol $\langle\;\rangle_{x}$ stands for
integration over the variable $x$ (after acting to the right with
the functional derivative). Note that the functional derivatives
with respect to the currents act according to the Leibnitz rule on
the term $\e^{-Z^{0}[{\bf J}]}$, and functional derivatives that
go through to the right of that term cancel
when acting on $W_{0}[0]$. Thus here $W_{E}[{\bf J}]$ is a functional and not an operator.\\

To simplify our exposition we shall consider the neutral
scalar $\varphi^4$ theory in Euclidean 4-dimensions when doing
explicit calculations. It is well known that this theory is
renormalizable and a clear discussion of the steps leading to its
renormalization may be found in \cite{ramond}. For this case  the transition
amplitude in (\ref{01}) reduces to
\begin{equation}
W_{E}[J]=N\int {\mathcal D}\varphi\; \e^{-\int d^{4} x [\frac{1}{2}\partial_{\mu} \varphi
\partial_{\mu} \varphi +
\frac{1}{2} m^{2}
\varphi^{2}+V(\varphi)-J\varphi]}, \label{1b}
\end{equation}
and (\ref{02}) becomes
\begin{equation}
W_{E}[J]= \e^{-\langle V(\frac{\delta}{\delta J_{x}})\rangle
_{x}} e^{-Z^{0}[J]} W_{0}[0], \label{1}
\end{equation}
where
\begin{equation}
\langle V(\frac{\delta}{\delta J_{x}})\rangle  _{x}=\int d^{4}\;x
\; \frac{\lambda}{4!} \frac{\delta^{4}}{\delta J_{x}^{4}},
\label{1a}
\end{equation}
\begin{equation}
W_{0}[0]= N\int {\mathcal D}\varphi\; \e^{-\int d^{4} x [\frac{1}{2}\partial_{\mu} \varphi
\partial_{\mu} \varphi + \frac{1}{2} m^{2}
\varphi^{2}]}, \label{2}
\end{equation}
\begin{equation}
 Z^{0}[J]=\frac{1}{2}\langle J(x)\Delta_{xy} J(y)\rangle  _{xy},\label{3}
\end{equation}
and
\begin{equation}
\Delta_{xy}=\frac{1}{(2\pi)^4}\int d^{4}p \frac{\e^{ip\cdot(x-y)}}{p^{2} + m^{2}}\label{4}
\end{equation}
is the Feynman propagator in 4-dimensional Euclidean space.\\

Writing $W_{E}[J]=\e^{-Z_{E}[J]}$ and rearranging (\ref{1})
results in
\begin{equation}
Z_{E}[J]= - \ln W_{0}[0] + Z^{0}[J] - \ln \left(1 + \e^{Z^{0}[J]}
\left(e^{-\langle V(\frac{\delta}{\delta J})\rangle
}-1\right)\;\e^{-Z^{0}[J]}(1)\right),\label{5}
\end{equation}
where, in order to make both sides of the above equation
consistent, we have explicitly included the action on the identity
function of the operator inside the logarithm, so as to cancel
derivations to the right of $\e^{-Z^{0}[J]}$. Now let
\begin{equation}
\sigma(\lambda):=\e^{Z^{0}[J]}\left(\e^{-\langle
V(\frac{\delta}{\delta J})\rangle
}-1\right)\;\e^{-Z^{0}[J]}(1)=\sum_{k\geq 1} \lambda^{k}S_{k}[J],
\label{6}
\end{equation}
where the second equality is the formal power series expansion of the first one in terms of
the coupling constant $\lambda$ in the potential.
If we further define
\begin{equation}
\psi(\lambda):=\sum_{k\geq 1} \lambda^{k}\psi_{k}[J]=\ln (1 + \sum_{k\geq 1} \lambda^{k}S_{k}[J]),\label{7}
\end{equation}
it then readily follows from (\ref{5}) that
\begin{equation}
Z_{E}[J]= -\ln W_{0}[0]+ Z^{0}[J] - \sum_{k\geq 1} \lambda^{k} \psi_{k}[J].\label{8}
\end{equation}

Note that in the theory of symmetric functions \cite{mac,gelfand} an
expression like (\ref{7}) relates the complete homogeneous symmetric functions to
the so called power sum symmetric functions or Schur polynomials, so we can use the
combinatorics of that theory
% It is known, in addition, from the theory of symmetric functions \cite{mac,gelfand}
%that the relation between the so called power sum symmetric functions or Schur polynomials
%and the complete homogeneous symmetric functions is given by an expression like (\ref{7}), so
%we can use the results of that theory
to write down the explicit (invertible) relation between the functionals $\psi_{k}[J]$ and the $S_{i}[J]$,
as homogeneous polynomials of order $k$ in the latter. This is given by
\begin{equation}
\psi_{k}[J]=\sum_{|I|=k} (-1)^{l(I)-1}\;\frac{S^I}{l(I)}, \label{11}
\end{equation}
where $I$ is a composition $I = (i_{1},\dots, i_{r})$ of
nonnegative integers, $|I|= \sum_{k} i_{k}$ is its weight,
$l(I)=r$ is its length and $S^I= S_{i_{1}}S_{i_{2}}\dots
S_{i_{r}}$.
In terms of the Feynman diagrammatic representation, the $\psi_{k}[J]$ correspond to linear
combinations of connected graphs each with $k$ vertices. \\

Defining the operator
\begin{eqnarray}
X&=&-\frac{1}{\lambda} \e^{Z^{0}[J]}\left(\langle
V(\frac{\delta}{\delta J_{x_1}})\rangle  _{x_1}\right)
\;\e^{-Z^{0}[J]}\nonumber\\
 &=& -\frac{1}{\lambda}\sum_{n\geq 0}\frac{1}{n!} \text {ad}(Z^{0}[J])^{n}
 \left(\langle V(\frac{\delta}{\delta J_{x_1}})\rangle  _{x_1}\right),\label{9}
\end{eqnarray}
where the adjoint operator in the second equality is defined as
the right normed bracketing: $\text {ad}(b)^{n}(a)\equiv
[b,[b,\dots, [b,a]]\dots ]$, one can verify that the functionals
$S_{k}$ introduced above are given by the recursion relations
\begin{eqnarray}
S_{1}[J]&=& X(1)\nonumber \\
S_{n}[J]&=&\frac{1}{n}X(S_{n-1}[J])(1),\label{10}
\end{eqnarray}
where $X(1)$ denotes the action of the operator (\ref{9}) on the
identity function. \\

In the same way that $X$ generates a recursion relation for the functionals $S_{k}$,
we will show bellow that a derivation operator can be defined which when acting on the
$\psi_{k}[J]$ leads to expressions analogous to (\ref{10}). We first analyze some of the
algebraic structures behind the operators occurring in (\ref{1}) and (\ref{5}).
\\

\subsection{Algebraic Structures}

As mentioned in the Introduction, one of our goals in this paper is to
investigate the relevance to the process of renormalization of
Hopf algebra primitives that occur naturally at a prediagram stage
in PQFT. Therefore these primitives must be related to the two entities appearing
in the perturbation expansion of the transition amplitude, {\it i.e.} the complete
homogeneous symmetric functionals and the power sum symmetric functionals.

From the theory of symmetric functions we know that one can
associate respective Hopf algebra structures to these two sets of
functionals for which the $\psi_{k}[J]$ are primitives. In
addition to the above, there is a free algebra generated
by the operator $\langle V \rangle $ and the functional $Z^0[J]$
for which the $\psi_{k}[J]$ are also primitives. We describe these
three Hopf algebras which are related by the following diagram
 
 

\[
\begin{array}{cccc}
K< \left\lbrace S_{k}\right\rbrace> \simeq K< \left\lbrace \psi_{k}\right\rbrace > & \subset K< \left\lbrace
\Psi_{k}\right\rbrace > & \hookrightarrow & K < Y >
\\
&\bigcup & &    \bigcup \\
& {\frak L}_{K}(\{\Psi_{k}\}) & \hookrightarrow & {\frak L}_{K}(Y) \\[10pt]
&\hspace{40pt}\rotatebox{45}{$\bigcup$} &           &  \\[-17pt]
&                          &           & \hspace{-20pt}\rotatebox{-45}{$\bigcup$} \\
&                          & \psi_{k}  &
\end{array}
\]


%ext, starting with the later
%algebra as it has more structure and contains the basic elements from which we can derive
%the derivation operator mentioned above.
%If one views the $S_{i}[J]$'s as coordinates in an infinite dimensional space,
%then one can think of the $\psi_{k}[J]$ functionals as normal coordinates on that space.

\subsubsection{The Free Algebra $K<Y>$ and its Hopf Algebra Structure}
Let $K<Y>$ be the unital free associative 
$K$-algebra over a field of characteristic
zero (including ${\Bbb Q}$) and
generated by the two-letter alphabet $Y=\{Z^{0}[J], \langle
V(\frac{\delta}{\delta J_{x}})\rangle  _{x}\}$ of non-commuting
variables, with concatenation as multiplication and unit (neutral)
element ${\bf 1}$ the empty word.
Let ${\frak L}_K(Y)$ be the
infinite dimensional free Lie algebra on $Y$ where its elements
are submodules of $K<Y>$ with Lie bracket as multiplication.
$K<Y>$ is the enveloping algebra of ${\frak L}_K(Y)$. We can give
$K<Y>$ a Hopf algebra structure \cite{reut} by defining a
primitive coproduct on the alphabet letters:
\begin{eqnarray}
\Delta({\bf 1})&=&{\bf 1}\otimes {\bf 1},\\
\Delta(Z^{0}[J])&=&{\bf 1}\otimes Z^{0}[J]+ Z^{0}[J]\otimes {\bf 1},\\
\Delta(\langle V\rangle  )&=&{\bf 1}\otimes \langle V\rangle  +
\langle V\rangle  \otimes {\bf 1},\label{ll.1}
\end{eqnarray}
and extending it to words by the connection axiom. The antipode is given by
\begin{eqnarray}
S(a)&=& -a, \;\;\; (a=\langle V\rangle  ,\;\text{or}\; Z^{0}[J])\\
S({\bf 1})&=&{\bf 1},\label{ll.2}
\end{eqnarray}
and is extended to words by the antihomorphism
\begin{equation}
S(a_{1}\dots a_{n})=S(a_{n})\dots S(a_{1}).\label{ll.3}
\end{equation}
The counit map $\varepsilon:K<Y>\rightarrow K$ is defined on the generating letters by
\begin{eqnarray}
\varepsilon(a)&=&0 \;\; (a=\langle V\rangle  ,\;\hbox{or}\; Z^{0}[J])\\
\varepsilon({\bf 1})&=&1,\label{ll.4}
\end{eqnarray}
and is extended to words by the connection axiom.
All the elements (Lie polynomials) $P\in {\frak L}_K(Y)$ are primitives of this Hopf algebra:
\begin{equation}
\Delta(P)={\bf 1}\otimes P+ P\otimes {\bf 1},\label{ll.5}
\end{equation}
and
\begin{eqnarray}
S(P)&=&-P,\\
\varepsilon(P)&=&(P,{\bf 1}), \label{ll.6}
\end{eqnarray}
where $(P,{\bf 1})$ is the coefficient in $P$ of the unit element.\\

%In fact, $K<Y>$ can be given an additional Hopf algebra structure by taking
%shuffle as product and the adjoint of the concatenation, $\Delta^{\prime}$,
%as coproduct, {\it ie.} \cite{reut}
%\begin{equation}
%\Delta^{\prime}(w)=\sum_{u,v} (w,uv)\;u\otimes v,\label{lll.1}
%\end{equation}
%where $w$ is any word over $Y$ and $u,v$ are elements of the free
%monoid on $Y$. We shall not consider this alternative in what
%follows, even though its relation
%to renormalization may be worthwhile investigating.\\

We introduce now on $K<Y>$ a derivation  which maps $\langle
V(\frac{\delta}{\delta J_{x}})\rangle  _{x}$ onto 0, and
$Z^{0}[J]$ onto $X$ by means of the operator
\begin{equation}
D=-X\frac{\partial}{\partial Z^{0}[J]}.\label{d.0}
\end{equation}
Clearly $D$ is an endomorphism on $K<Y>$.\\


Since $Z^{0}[J]$ and $\langle V(\frac{\delta}{\delta
J_{x}})\rangle  _{x}$ do not commute, the action of the derivation
$D$ on an arbitrary function of $Z^{0}[J]$ is given by
\cite{reut}:
\begin{equation}
D(f(Z^{0}[J]))=\sum_{k\geq 1} \frac{1}{k!} \text
{ad}(Z^{0}[J])^{k-1}\left(D
Z^{0}[J]\right)f^{(k)}\left(Z^{0}[J]\right), \label{d.1}
\end{equation}
and, in particular,
\begin{eqnarray}
D(\e^{-Z^{0}[J]})&=&\sum_{k\geq 1} \frac{1}{k!} \text {ad}(-Z^{0}[J])^{k-1}(X)\ \e^{-Z^{0}[J]}\nonumber\\
&=& \left(\frac{\e^{\text {ad}(-Z^{0}[J])} -1}{\text {ad}(-Z^{0}[J])}\right)(X)\ \e^{-Z^{0}[J]}\nonumber\\
&=&\e^{-Z^{0}[J]}(X)e^{Z^{0}[J]}\cdot \e^{-Z^{0}[J]}\nonumber\\
&=&-\frac{1}{\lambda}\langle V\left(\frac{\delta}{\delta
J_{x}}\right)\rangle  _{x}\e^{-Z^{0}[J]}, \label{d.2}
\end{eqnarray}
where in going from the third to the last line we have made use of
(\ref{9}). Moreover since $D(\langle V(\frac{\delta}{\delta
J_{x}})\rangle  _{x})=0$, we immediately get
\begin{equation}
\lambda^{n}D^{n}(\e^{-Z^{0}[J]})=\left(-\langle
V\left(\frac{\delta}{\delta J_{x}}\right)\rangle
_{x}\right)^{n}\e^{-Z^{0}[J]}.\label{d.5}
\end{equation}
Using now once more the fact that $D$ is a derivation, we have that the exponential
map $\mu=\e^{\lambda D}$ is a homomorphism of algebras and
\begin{eqnarray}
\mu(\e^{-Z^{0}[J]})&=&\sum_{n\geq 0}\frac{\lambda^{n}}{n!}D^{n}(\e^{-Z^{0}[J]})\nonumber\\
                  &=&\sum_{n\geq 0}\frac{1}{n!}\left(-\langle V\left(\frac{\delta}{\delta J_{x}}
                  \right)\rangle  _{x}\right)^{n}\e^{-Z^{0}[J]}\nonumber\\
                  &=&\e^{-\langle V(\frac{\delta}{\delta J_{x}})\rangle  _{x}}\e^{-Z^{0}[J]}.\label{d.6}
\end{eqnarray}
Furthermore, since $\mu$ is a continuous homomorphism,
\begin{eqnarray}
\e^{-\langle V(\frac{\delta}{\delta J_{x}})\rangle  _{x}}\e^{-Z^{0}[J]}&=&\mu\left(\e^{-Z^{0}[J]}\right)\nonumber\\
&=&\e^{\mu (-Z^{0}[J])}\nonumber\\
&=& \exp\left(\sum_{n\geq
0}\frac{\lambda^{n}}{n!}D^{n}(Z^{0}[J])\right),\label{d.7}
\end{eqnarray}
where
\begin{equation}
D^{n}(Z^{0}[J])=(-1)^{n} \underbrace{\left(X
\frac{\partial}{\partial Z^{0}[J]}\right)\dots\left(X
\frac{\partial}{\partial
Z^{0}[J]}\right)}_{n}(Z^{0}[J]).\label{ll.7}
\end{equation}
Writing
\begin{equation}
\e^{\Psi}=\e^{-\langle V(\frac{\delta}{\delta J_{x}})\rangle
_{x}}\e^{-Z^{0}[J]},\label{ll.8}
\end{equation}
we obtain from (\ref{d.7}) and (\ref{ll.8}) the Hausdorff series relation
\begin{equation}
\Psi= Z^{0}[J]+\sum_{n\geq 1}\frac{\lambda^{n}}{k!}D^{k}(Z^{0}[J]).\label{ll.9}
\end{equation}

Let us now define
\begin{equation}
\Psi_{k}:=\frac{1}{k!} D^{k}(Z^{0}[J]),\label{ll.10}
\end{equation}
which exhibits the operators $\Psi_{k}$ as cyclic vectors with respect to $D$ generated by $Z^{0}[J]$,
and rewrite $X$, introduced in (\ref{9}), as
\begin{equation}
X=-\frac{1}{\lambda}\sum_{j=0}^{d}\frac{1}{j!}[Z^{0}[J],\langle
V\left(\frac{\delta}{\delta J_{x}}\right)\rangle
_{x}]_{j},\label{d.13}
\end{equation}
where $[Z^{0}[J],\langle V(\frac{\delta}{\delta J_{x}})\rangle
_{x}]_{j}$ is defined recursively by $[Z^{0}[J],\langle
V(\frac{\delta}{\delta J_{x}})\rangle
_{x}]_{j}=\break [Z^{0}[J],[Z^{0}[J], \langle V(\frac{\delta}{\delta
J_{x}})\rangle  _{x}]_{j-1}]$, $[Z^{0}[J],\langle
V(\frac{\delta}{\delta J_{x}})\rangle  _{x})]_{0}= \langle
V(\frac{\delta}{\delta J_{x}})\rangle  _{x}$, and the upper index
$d$ in the sum above is the degree of the functional derivative in
$\langle V(\frac{\delta}{\delta J_{x}})\rangle  _{x}$ ($n$=4 for the $\varphi^{4}$ theory).\\

It clearly follows from this and (\ref{ll.7}) that the cyclic vectors $\Psi_{k}$ are elements
of ${\frak L}_K(Y)$ and, hence, primitive elements of the Hopf algebra $K<Y>$.\\

In addition, since $D$ is also an endomorphism on ${\frak L}_K(Y)$ there is a
Hochschild cohomology associated with this algebra for which $D$ is a 1-cochain
$D:{\frak L}_K(Y)\rightarrow {\frak L}_K(Y)$ with coboundary
\begin{equation}
bD(P)=({\text id}\otimes D)\Delta(P)-D(P)\otimes {\bf 1}; \;\; P\in {\frak L}_K(Y).\label{a.252}
\end{equation}

Evidently $bD(P)=0$, because of (\ref{ll.5}),  so the Lie polynomials $P$, and
in particular the $\Psi_{k}$, are 1-cocycles for this cohomology.\\


Note that by applying both sides of (\ref{ll.9}) to the identity function and
recalling (\ref{1}) we recover (\ref{8}), with the functional $\psi_{k}[J]$ given by
\begin{equation}
\psi_{k}[J]=-\frac{1}{k!}D^{k}(Z^{0}[J])(1).\label{d.10}
\end{equation}
Furthermore, since the $\psi_{k}[J]$ are made up by linear combinations of Lie
monomials that do not cancel when acting from the left on the identity they are obviously
also elements of ${\frak L}_K(Y)$ and primitives of the Hopf algebra $K<Y>$.\\


%Applying now both sides of (\ref{d.7}) to the identity function and recalling the expressions (\ref{1}) and (\ref{8}), we get
%\begin{eqnarray}
%\e^{-\mu (Z^{0}[J])}(1)&=& \e^{-<V(\frac{\delta}{\delta J_{x}})>_{x}}\e^{-Z^{0}[J]}(1)\nonumber\\
 %                      &=&\exp({-Z^{0}[J]+\sum_{k\geq 1}\lambda^{k}\psi_{k}[J]}),\label{d.8}
%\end{eqnarray}
%or
%\begin{equation}
%\mu (Z^{0}[J])(1)=\sum_{n\geq 0}\frac{\lambda^{n} D^{n}(Z^{0}[J])}{n!}(1)= Z^{0}[J]-\sum_{k\geq 1}\lambda^{k}\psi_{k}[J]. \label{d.9}
%\end{equation}
%This in turn implies
%\begin{equation}
%\psi_{n}[J]=-\frac{D^{n}(Z^{0}[J])}{n!}(1),\label{d.10}
%\end{equation}


We can derive a recursion relation for the cyclic functionals $\psi_{k}[J]$
by observing that the right side of (\ref{d.10}) can be written as
\begin{equation}
D^{n}(Z^{0}[J])(1)=(-1)^{n}\left( \underbrace{\left(X
\frac{\partial}{\partial Z^{0}[J]}\right)\dots\left(X
\frac{\partial}{\partial
Z^{0}[J]}\right)}_{n-1}X\right)(1).\label{d.10a}
\end{equation}
In particular, for $n=1$
\begin{equation}
-D(Z^{0}[J])(1)=\left(X\frac{\partial}{\partial
Z^{0}[J]}(Z^{0}[J])\right)(1)=X(1)=\psi_{1}, \label{d.11}
\end{equation}
which is in agreement with (\ref{10}) and (\ref{11}).\\

To further see how to interpret the right side of
(\ref{d.10a}) for $n\geq 2$, use (\ref{d.13}) and note that in
order to take into account the fact that the identity cancels
functional derivatives acting on it, we must first apply the $n-1$
derivations in (\ref{d.10a}) to $X$ following the Leibnitz rule,
evaluate each of the resulting derivations on $X$ inside the
commutators by acting on the identity function according to
(\ref{d.10}), and finally act with the resulting bracket
polynomial on the identity. Thus
%of the $n$-th order derivation in the right side  of (\ref{d.10}) is of the form
%then have from (\ref{d.10}) and (\ref{d.10a}) that
\begin{equation}
\begin{split}
\psi_{2}&=-\frac{1}{2}\left(\left(X \frac{\partial}{\partial Z^{0}[J]}\right)X\right)(1) \\
&=\frac{1}{2\lambda}\left(X \frac{\partial}{\partial
Z^{0}[J]}\right)\left(\langle V\rangle
-[\langle V\rangle  ,Z^{0}[J]]+\frac{1}{2} [[\langle V\rangle  ,Z^{0}[J]],Z^{0}[J]]\right. \\
&\ \left. + \sum_{j=3}^{d}\frac{1}{j!}[Z^{0}[J],\langle V\rangle  ]_{j}\right)(1) \\
&=\frac{1}{2\lambda}(-[\langle V\rangle
,\psi_{1}]+\frac{1}{2}[[\langle V\rangle  ,\psi_{1}],Z^{0}[J]]
+\frac{1}{2}[[\langle V\rangle  ,Z^{0}[J]],\psi_{1}]+\dots)(1),
\label{ddd.1}
\end{split}
\end{equation}


\begin{equation}
\begin{split}
\psi_{3}&=\frac{1}{6}\left(\left(X \frac{\partial}{\partial
Z^{0}[J]}\right)
\left(X \frac{\partial}{\partial Z^{0}[J]}\right)X\right)(1)                           \\
&= -\frac{1}{6\lambda}\left(X \frac{\partial}{\partial
Z^{0}[J]}\right)\left(-[\langle V\rangle  ,X] +\frac{1}{2}[[\langle V\rangle  ,X],Z^{0}[J]] \right.         \\
&\ \left. +\frac{1}{2}[[\langle V\rangle  ,Z^{0}[J]],X]+\dots\right)(1)               \\
&=-\frac{1}{6\lambda}\left(-\left[\langle V\rangle  ,\left(\left(X
\frac{\partial}{\partial Z^{0}[J]}
\right)X\right)(1)\right]\right)                    \\
&\ +\frac{1}{2}\left[\left[\langle V\rangle  ,\left(\left(X
\frac{\partial}{\partial
Z^{0}[J]}\right)X\right)(1)\right],Z^{0}[J]\right]   \\
&\ +\frac{1}{2}\left[[\langle V\rangle  ,Z^{0}[J]],
\left(\left(X \frac{\partial}{\partial Z^{0}[J]}\right)X\right)(1)\right] 
+[[\langle V\rangle  ,\psi_{1}],\psi_{1}]+\dots \bigg)(1) \\
& =-\frac{1}{6\lambda}\left(2[\langle V\rangle  ,\psi_{2}]-[[\langle V\rangle  ,\psi_{2}],Z^{0}[J]] +[[\langle V\rangle  ,\psi_{1}],\psi_{1}]\right. \\
&\ \left. -[[\langle
V\rangle  ,Z^{0}[J]],\psi_{2}]+\dots\right)(1).\label{ddd.2}
\end{split}
\end{equation}

Iterating on the above, we get the general recursion relation
\begin{equation}
\psi_{n+1}[J]=\frac{1}{n+1}D\psi_{n}[J]=-\frac{1}{n+1}\psi_{1}\frac{\partial}
{\partial Z^{0}[J]}\psi_{n}.\label{d.12}\hspace{.5in}\square
\end{equation}
It should be recalled however that in the implementation of (\ref{d.12}) one has to take
 $\psi_{1}=-\frac{1}{\lambda}\sum_{j=1}^{d}\frac{1}{j!}\left[Z^{0}[J],\langle V(\frac{\delta}
 {\delta J_{x}})\rangle  _{x}\right]_{j}$, perform the derivations to the required order and then
  evaluate on the identity function. \\

%%%%%%%%%%%%%%%%%%%%%%%%%%%%%%%%%%%%%%%%%%%%%%%%%%%%%%%%%%%%%%%%%%%%%%%%%%%%%%%%%%%%%%%%%%%%%%%%%%%%%%%
%%%%%%%%%%%%%%%%%%%%%%%%%%%%%%%%%%%%%%%%%%%%%%%%%%%%%%%%%%%%%%%%%%%%%%%%%%%%%%%%%%%%%%%%%%%%%%%%%%%%%%%

We can express the above results in graphical form by making use of Hall trees.
To this end recall \cite{reut} that each Hall tree $h$ of order at least 2 can be written as $h=(h^{\prime},h^{\prime\prime})$, where $h^{\prime}$ and $h^{\prime\prime}$ are the immediate left and right subtrees, respectively, and such that the total ordering 
\begin{equation}
h<h^{\prime\prime}, \ \ h^{\prime}<h^{\prime\prime} \ \ \hbox{and either } 
 h^{\prime}  \in Y, \hbox{ or } h^{\prime}=(x,y) \hbox{ and } y\geqslant h^{\prime\prime}
\end{equation} 
is satisfied. This ordering is lexicographical and is determined by the letters in the alphabet $Y$ that label the leaves. Also, each node in a tree corresponds to a Lie bracket and the foliage $f(h)$ of the tree is the canonical mapping defined by $f(a)=a$ if $a$ is in $Y$ and $f(h)=f(h^{\prime})f(h^{\prime\prime})$ if $h=(h^{\prime},h^{\prime\prime})$ is of degree $\geq 2$.
Now, since a Hall word is the foliage of a unique Hall tree, and since for each Hall word $h$ there is a Lie polynomial $P_{h}$, it can be shown that these Hall polynomials form an infinite dimensional basis of the Lie algebra ${\frak L}_K(Y)$ viewed as a $K$-module.  Parenthetically, one also has that the decreasing products of Hall polynomials $P_{h_1}\dots P_{h_n}$, $h_{1}\geq\dots h_{n}\;$ form a basis of the free associative algebra $K<Y>$.\\
Consequently, using the ordering $\langle  V \rangle <Z^{0}$ for
our two letter alphabet $Y$ and the algorithm given in the proof
of Theorem 4.9 in \cite{reut}, we can always write the $\psi_{i}[J]$'s, 
as derived from (\ref{ll.10}) and (\ref{d.12}), as linear combinations
of Hall polynomials (equivalently Hall trees). Thus, for example 
\vskip -10pt
\begin{equation}
\Psi_1 \equiv X = -\frac{1}{\lambda} \bigg( \langle  V \rangle -
\Tone{Z^0}{\langle  V \rangle} +\frac{1}{2}\Ttwo{Z^0}{Z^0}{\langle
V \rangle} - \frac{1}{6}\Tthree{Z^0}{Z^0}{Z^0}{\langle  V \rangle}
+ \frac{1}{24} \Tfour{Z^0}{Z^0}{Z^0}{Z^0}{\langle  V \rangle}
\bigg),\label{f.1}
\end{equation}
so
\begin{equation}
\psi_1 [J] = \Psi_1(1) = -\frac{1}{2\lambda}
\Ttwo{Z^0}{Z^0}{\langle  V \rangle} +\frac{1}{6\lambda}
\Tthree{Z^0}{Z^0}{Z^0}{\langle  V \rangle} -\frac{1}{24\lambda}
\Tfour{Z^0}{Z^0}{Z^0}{Z^0}{\langle  V \rangle};\label{f.2}
\end{equation}
%\vskip -30pt

and
\begin{eqnarray}
\Psi_2 &=& -\frac{1}{2\lambda} \Tone{\Psi_1}{\langle  V \rangle}
+\frac{1}{4\lambda} \bigg(
  \Ttwo{Z^0}{\Psi_1}{\langle  V \rangle} +
  \Ttwo{\Psi_1}{Z^0}{\langle  V \rangle}
\bigg) \nonumber\\ &&
- \frac{1}{12\lambda}
\bigg(
   \Tthree{Z^0}{Z^0}{\Psi_1}{\langle  V \rangle} +
   \Tthree{Z^0}{\Psi_1}{Z^0}{\langle  V \rangle} +
   \Tthree{\Psi_1}{Z^0}{Z^0}{\langle  V \rangle}
\bigg) \nonumber\\ && + \frac{1}{48\lambda} \bigg(
\Tfour{Z^0}{Z^0}{Z^0}{\Psi_1}{\langle  V \rangle} +
\Tfour{Z^0}{Z^0}{\Psi_1}{Z^0}{\langle  V \rangle} +
\Tfour{Z^0}{\Psi_1}{Z^0}{Z^0}{\langle  V \rangle} +
\Tfour{\Psi_1}{Z^0}{Z^0}{Z^0}{\langle  V \rangle}
\bigg),\label{f.3}
\end{eqnarray}

\begin{eqnarray}
\psi_2 [J] &=& -\frac{1}{2\lambda}
\Tone{\hspace{-5pt}\psi_1[J]}{\langle  V \rangle} \hspace{5pt}
+\frac{1}{4\lambda} \bigg(
  \Ttwo{Z^0}{\psi_1 [J]}{\langle  V \rangle} +
  \Ttwo{\psi_1[J]}{Z^0}{\langle  V \rangle} \hspace{7pt}
\bigg)\nonumber \\ &&
- \frac{1}{12\lambda}
\bigg(
   \Tthree{Z^0}{Z^0}{\psi_1[J]}{\langle  V \rangle} +
   \Tthree{Z^0}{\psi_1[J]}{Z^0}{\langle  V \rangle} \hspace{7pt} +
   \Tthree{\psi_1[J]}{Z^0}{Z^0}{\langle  V \rangle} \hspace{10pt}
\bigg) \nonumber\\ && + \frac{1}{48\lambda} \bigg(
\Tfour{Z^0}{Z^0}{Z^0}{\psi_1 [J]}{\langle  V \rangle} +
\Tfour{Z^0}{Z^0}{\psi_1[J]}{Z^0}{\langle  V \rangle} +
\Tfour{Z^0}{\psi_1[J]}{Z^0}{Z^0}{\langle  V \rangle}\hspace{7pt} +
\Tfour{\psi_1[J]}{Z^0}{Z^0}{Z^0}{\langle  V \rangle}\hspace{10pt}
\bigg).\label{f.4}
\end{eqnarray}
\\
\vskip10pt
Note that (\ref{f.2}) and (\ref{f.4}) for the functionals $\psi_1[J]$ and $\psi_2[J]$
imply acting on the identity function from the left with the diagrams and replacing
in addition  the $\Psi_1$ in the foliage of (\ref{f.3}) by $\psi_1[J]$. The corresponding Hall tree for $\psi_2[J]$ is obtained by grafting (\ref{f.2}) onto each of the branches labeled with $\psi_1[J]$ and implementing the above mentioned algorithm. The procedure is then iterated to whatever order of the $\psi$'s one is interested.\\
%We call the foliage $f(t)$ of a Hall tree $t$, the canonical mapping defined by $f(a)=a$ if $a$ is in $Y$ and $f(t)=f(t^{\prime})f(t^{\prime\prime})$ if $t=(t^{\prime},t^{\prime\prime})$ is of degree $\geq 2$.
%Now, since a Hall word is the foliage of a unique Hall tree, and since for each Hall word $h$ there is a Lie polynomial $P_{h}$, it can be shown these Hall polynomials form an infinite dimensional basis of the Lie algebra ${\frak L}_K(Y)$ viewed as a $K$-module. Consequently, our $\psi_{i}[J]$'s are given as linear combinations of Hall polynomials (equivalently Hall trees). Parenthetically, one also has that the decreasing products of Hall polynomials $P_{h_1}\dots P_{h_n}$, $h_{1}\geq\dots h_{n}\;$ form a basis of the free associative algebra $K<Y>$.\\






%\begin{equation}
%Psi_1 = X =-\frac{1}{\lambda}\left( <V> -
%  \hbox{\epsfig{file=lavz0.eps,height=6mm,width=7mm}}+ \frac{1}{2}
%   \hbox{\epsfig{file=lvz0z0.eps,height=6mm,width=7mm}}
%  -\frac{1}{6} \hbox{\epsfig{file=lvz0z0z0.eps,height=7mm,width=7mm}} + \frac{1}{24}
%  \hbox{\epsfig{file=lvz0z0z0z0.eps,height=7mm,width=7mm}}
%   \right)
%   , \label{naaa.1}
%\end{equation}
%so

% \begin{equation}
 % \psi_1[J] =\Psi_1(1) =- \frac{1}{2\lambda} %\hbox{\epsfig{file=lvz0z0.eps,height=6mm,width=7mm}}
 %  +\frac{1}{6\lambda} \hbox{\epsfig{file=lvz0z0z0.eps,height=7mm,width=7mm}} - \frac{1}{24}
  % \hbox{\epsfig{file=lvz0z0z0z0.eps,height=7mm,width=7mm}}
   % , \label{naaa.2}
 %\end{equation}
%Derivating on (\ref{naaa.1}) we get
%\begin{equation}
%\begin{split}
%\Psi_2 &= -\frac{1}{2\lambda^2}
%\hbox{\epsfig{file=lvP.eps,height=6mm,width=7mm}}+ \frac{1}{4\lambda^2}
% \left(\hbox{\epsfig{file=lvPz0.eps,height=6mm,width=7mm}}
 %+\hbox{\epsfig{file=lvz0P.eps,height=6mm,width=7mm}}
 %\right)
 %-\frac{1}{12\lambda^2} \left( \hbox{\epsfig{file=lvPz0z0.eps,height=7mm,width=7mm}} +
%5\hbox{\epsfig{file=lvz0Pz0.eps,height=7mm,width=7mm}} +
%\hbox{\epsfig{file=lvz0z0P.eps,height=7mm,width=7mm}}\right)
%5%\\
%&\ +\frac{1}{48\lambda^2} \left(\hbox{\epsfig{file=lvPz0z0z0.eps,height=7mm,width=7mm}} +
%\hbox{\epsfig{file=lvz0Pz0z0.eps,height=7mm,width=7mm}} +
%\hbox{\epsfig{file=lvz0z0Pz0.eps,height=7mm,width=7mm}} +
%\hbox{\epsfig{file=lvz0z0z0P.eps,height=7mm,width=7mm}}
%\right), \label{naaa.3}
%\end{split}
%\end{equation}

%The expression for the functional $\psi_2[J]$ is obtained by
%replacing in the above diagrams $\Psi_1$ by $\psi_1[J]$, and the corresponding Hall trees
%result by grafting the Hall trees in (ref{naaa.2}) into this expression. \\

%In order to actually perform the calculations of the
%$\psi_{k}[J]$'s with (\ref{d.10}) note  %one first writes
%$X$ as a polynomial series in $Z^{0}[J]$ and
%$<V(\frac{\delta}{\delta J_{x}})>_{x}$. Using%
%that (\ref{9}) can be written as
%\begin{equation}
%X=-\frac{1}{\lambda}\sum_{j=0}^{n}
%\frac{1}{j!}[Z^{0}[J],<V(\frac{\delta}{\delta J_{x}})>_{x}]_{j},\label{d.13}
%\end{equation}
%where $[Z^{0}[J],<V(\frac{\delta}{\delta J_{x}})>_{x}]_{j}$
% is defined recursively by $[Z^{0}[J],<V(\frac{\delta}{\delta
%J_{x}})>_{x}]_{j}=[Z^{0}[J],[Z^{0}[J],<V(\frac{\delta}{\delta J_{x}})>_{x}]_{j-1}]$,
%$[Z^{0}[J],<V(\frac{\delta}{\delta J_{x}})>_{x})]_{0}=
%<V(\frac{\delta}{\delta J_{x}})>_{x}$, and the upper
%index $n$ in the sum above is the degree of the functional
%derivative in $<V(\frac{\delta}{\delta
%J_{x}})>_{x}$ ($n$=4 for the $\varphi^{4}$ theory).\\

%Applying (\ref{d.13}) to the identity function gives $\psi_{1}[J]$.
%\begin{equation}
%X(1)=\psi_{1}=-\frac{1}{\lambda}\sum_{j=1}^{n}\frac{1}{j!}[Z^{0}[J],<V(\frac{\delta}{\delta %J_{x}})>_{x}]_{j}(1),\label{dd.13}
%\end{equation}
%where the upper index $n$ in the sum above is the degree of the functional derivative in %$<V(\frac{\delta}{\delta J_{x}}>_{x}$ ($n$=4 for the $\varphi^{4}$ theory).\\
%To derive $\psi_{2}$ we act with $D$ on (\ref{d.13}), replacing the $Z^{0}[J]$'s
%by $\psi_{1}[J]$ in the
%multi-commutators one by one in accordance with the
%Leibnitz rule and then evaluating the resulting operator
%on the identity function. $\psi_{3}$ is obtained by acting
%with $D$ on the resulting $\psi_{2}$, applying
%again Leibnitz rule and making use of (\ref{d.12})
%(since the $\psi_{k}$'s depend implicitly on $Z^{0}[J]$).
%The procedure is  repeated successively to whatever order of derivation is desired.

By a straightforward calculation one can verify that the final expressions for $\psi_1[J]$ and
$\psi_2[J]$ in terms of propagators for the $\varphi^4$ theory are
%%%%%%%%%%%%%%%%%%%%%%%%%%%%%%%%%%%%%%%%%%%%%%%%%%%%%%%%%%%%%%%%%%%%%%%%%%%%%%%%%%%%%%%%%%%%%%%%%%
%%%%%%%%%%%%%%%%%%%%%%%%%%%%%%%%%%%%%%%%%%%%%%%%%%%%%%%%%%%%%%%%%%%%%%%%%%%%%%%%%%%%%%%%%%%%%%%%%%
\begin{equation}
\begin{split}
\psi_{1}[J]& =-\frac{1}{4!}[\langle \Delta_{xa}\Delta_{xb}\Delta_{xc}\Delta_{xd}J_{a}J_{b}J_{c}J_{d}\rangle   \\
&\ -6\langle \Delta_{xx}\Delta_{xa}\Delta_{xb}J_{a}J_{b}\rangle +
3\langle \Delta_{xx}^{2}\rangle  ],\label{12}
\end{split}
\end{equation}

\begin{equation}
\begin{split}
\psi_{2}[J]&=-\frac{1}{2} \langle J_{a}\Delta_{ax} (\frac{1}{6}
\Delta_{xy}^{3}+
 \frac{1}{4}\Delta_{xx}\Delta_{xy}\Delta_{yy})\Delta_{yb}J_{b}\rangle  _{xyab}         \\
&\ -\frac{1}{8} \langle
J_{a}\Delta_{ax}\Delta_{xy}^{2}\Delta_{yy}\Delta_{xb}J_{b}\rangle
_{xyab}
               \\
&\ +\frac{2}{4!}\langle
J_{a}\Delta_{ax}\Delta_{xx}\Delta_{xy}\Delta_{yb}\Delta_{yc}\Delta_{yd}J_{b}
J_{c}J_{d}\rangle  _{xyabcd}\\
&\ +\frac{3}{2(4!)}\langle
J_{a}J_{b}\Delta_{ax}\Delta_{bx}\Delta_{xy}^{2}
\Delta_{yc}\Delta_{yd} J_{c}J_{d}\rangle  _{xyabcd} \\
&\ -\frac{1}{2(3!)^{2}}\langle
J_{a}J_{b}J_{c}\Delta_{xa}\Delta_{xb}\Delta_{xc}\Delta_{xy}\Delta_{yd}
\Delta_{ye}\Delta_{yf}J_{d}J_{e}J_{f}\rangle  _{xyabcdef} \\
&\ + \frac{1}{48}\langle \Delta_{xy}^{4}\rangle  _{xy}  +
\frac{3}{2(4!)}\langle
\Delta_{xx}\Delta_{xy}^{2}\Delta_{yy}\rangle  _{xy},\label{13}
\end{split}
\end{equation}
The expressions for higher order $\psi$'s are increasingly more lengthy, but amenable
to a systematic derivation by the above procedure. This we have done by developing a
REDUCE program which confirms the results given above.\\

%In addition to the above described Hopf algebra associated with the free algebra $K<Y>$,
%there are two other Hopf algebras that occur naturally in the formalism of PQFT which we discuss next.

%As a parenthetical remark, note that since the derivation $D$ is an endomorphism on the algebra
%$K ( Z^{0}[J], <V(\frac{\delta}{\delta J_{x}})>_{x})$, there is an isomorphism between this algebra and the non-unital universal graded differential algebra $\Omega(K ( Z^{0}[J], <V(\frac{\delta}{\delta J_{x}})>_{x}))=\oplus_{n\geq 0} \Omega_{n}(K ( Z^{0}[J], <V(\frac{\delta}{\delta J_{x}})>_{x}))$
%linearly generated by the currents
%\begin{equation}
%Da_{i},\label{cc.1}
%\end{equation}
%\begin{equation}
%[a_{0},Da_{i}],\label{cc.2}
%\end{equation}
%\hspace{2.4in}\vdots
%\begin{equation}
%[\dots [[a_{0}, Da_{1}], Da_{2}], \dots],Da_{n-1}],Da_{n}],\;\;n\geq 1\label{cc.3}
%\end{equation}
%where $a_{0}, a_{1},\dots a_{n}\dots$, denote elements of the algebra $K ( Z^{0}[J], <V(\frac{\delta}{\delta J_{x}})>_{x})$. An $n$-chain in the vector space of currents $\Omega_{n}(K ( Z^{0}[J], <V(\frac{\delta}{\delta J_{x}})>_{x}))$ is then a linear combination of elements of the form (\ref{cc.3}), where each derivation on an element $a_{i}$ of the algebra results in a functional with one more vertex.\\

%We can construct an homology for the algebra $\Omega(K ( Z^{0}[J], <V(\frac{\delta}{\delta J_{x}})>_{x}))$ by extending the de Rham definition of the boundary operator $b$ to commutators as follows:
%\begin{equation}
%b =0,\;\; {\text on} \; \Omega_{0}(K ( Z^{0}[J], <V(\frac{\delta}{\delta J_{x}})>_{x}))=K ( Z^{0}[J], <V(\frac{\delta}{\delta J_{x}})>_{x}),\nonumber
%\end{equation}
%\begin{equation}
%b(Da_{i})=0,\nonumber
%\end{equation}
%\hspace{2.4in}\vdots
%\begin{equation}
%\begin{split}
%b([\dots [[a_{0}, Da_{1}], Da_{2}], \dots],Da_{n-1}],Da_{n}])
%=\\
%[a_{n-1},[\dots [[a_{0}, Da_{1}], Da_{2}], \dots],Da_{n-2}],Da_{n}]\\
%+ [a_{n},[\dots [[a_{0}, Da_{1}], Da_{2}], \dots],Da_{n-2}],Da_{n-1}].\label{cc.4}
%\end{split}
%\end{equation}


%Hence, since $\psi_{1}[J]\in \Omega_{0}(K ( Z^{0}[J], <V(\frac{\delta}{\delta J_{x}})>_{x}))$ and
%$D$ is an endomorphism, it follows from (\ref{d.12}) that
%\begin{equation}
%b(D\psi_{k}[J])=0 \;\;\;k\geq1,\label{cc.55}
%\end{equation}
%so the $\psi_{k}[J]$'s are k-vertex 1-cycles in this homology. \\

\subsubsection{The Hopf algebra $K< \left\lbrace S_{k}\right\rbrace>$}

As we have seen in Eq.(\ref{8}) the power sum symmetric functionals $\psi_{k}[J]$ appear
naturally in PQFT as a result of a perturbative expansion (in the coupling parameter of
the theory)
of the transition amplitude. In terms of the $S_{i}[J]$ these $\psi_{k}[J]$ are
 given  (cf  Eq.(\ref{11}))
by the Schur polynomials. Using as generators the $S_{i}$'s, one can construct a
 universal enveloping
algebra $K< \left\lbrace S_{k}\right\rbrace>$ by introducing a Poincar{\'e}-Birkhoff-Witt basis
$\{1, S_{i_1}, S_{i_1}S_{i_2},
\dots\}$, and defining multiplication $m$ as the disjoint union of the elements of
this basis. Further,
a coalgebra structure can be generated by defining the coproduct \begin{equation}
\Delta(S_{k})=\sum_{i=0}^{k} S_{i}\otimes S_{k-i},\;\;\; S_{0}\equiv {\bf 1},\label{16}
\end{equation}
and a counit $\varepsilon$ as the augmentation of the algebra by
\begin{equation}
\varepsilon(S_{0})=1,\;\; \varepsilon(S_{k})=0, \; k\neq 0. \label{16i}
\end{equation}


We can give this coalgebra the structure of a Hopf algebra by additionally
defining an  antipode $S$ as
 the involutive homomorphism (because of commutativity)
\begin{equation}
S(S_{k})=-S_{k}- m(S\otimes{\text id}){\tilde \Delta}(S_{k}),\label{17}
\end{equation}
where $\tilde{\Delta}$ is the coproduct operation with the primitive contributions removed.\\

Since $K< \left\lbrace S_{k}\right\rbrace>$ is commutative, by the Milnor-Moore
theorem there is a cocommutative Hopf  algebra in duality with it
which is necessarily isomorphic to the universal enveloping
algebra ${\mathcal U}({\frak L})$, where $\frak L$ is a Lie
algebra.
The generators $Z_{i}$ of $\frak L$ are infinitesimal characters
of $K< \left\lbrace S_{k}\right\rbrace>$, {\it i.e} they are linear mappings
$Z_{i}:K< \left\lbrace S_{k}\right\rbrace>\rightarrow K< \left\lbrace S_{k}\right\rbrace>$ fulfilling the conditions :
\begin{eqnarray}
\langle  Z_{i}, S_{k}\rangle &=& \delta_{ik}, \label{19}\\
\langle  Z_{i}, S_{k}S_{l}\rangle&=& \langle  Z_{i}, S_{k}\rangle
\varepsilon(S_{l})+ \varepsilon(S_{k})
 \langle  Z_{i}, S_{l}\rangle.\label{19a}
\end{eqnarray}
We shall denote \cite{kastler} by $\partial \text{Char}K< \left\lbrace S_{k}\right\rbrace>$ the set of
infinitesimal
characters of $K< \left\lbrace S_{k}\right\rbrace>$.
Note that $\frak L$ is Abelian, since the coproduct in $K< \left\lbrace S_{k}\right\rbrace>$ is cocommutative.\\

The exponential mapping $\sum_{i}\alpha_{i} Z_{i}\rightarrow
\text{exp}(\sum_{i}\alpha_{i} Z_{i})
\in{\mathcal G}$, equipped with a convolution product $\ast$ and unit ${\bf 1_{*}}$:
\begin{eqnarray}
\langle  \chi\ast\eta, S_{k}\rangle&=&\langle
\chi\otimes\eta,\Delta S_{k}\rangle,\;\;
\chi,\eta\in {\mathcal G},\label{19b}\\
\langle  {\bf 1_{*}},S_{k}\rangle&=&\varepsilon(S_{k}),\label{19c}
\end{eqnarray}
together with the inverse
\begin{equation}
\chi^{-1}=\chi\circ S, \label{19d}
\end{equation}
generates a subgroup ${\mathcal G}$ of the group of characters of $K< \left\lbrace S_{k}\right\rbrace>$.\\

${\mathcal G}$ is dual to the Hopf algebra $K< \left\lbrace S_{k}\right\rbrace>$ and it is multiplicative,
{\it i.e.}
it satisfies
\begin{equation}
\langle  \chi, S_{k}S_{l}\rangle=\langle  \chi, S_{k}\rangle
\langle  \chi, S_{l}\rangle\;\;\;\;\chi\in{\mathcal G}.\label{20}
\end{equation}

Moreover, from (\ref{19b}) and the fact that the Lie algebra of $\partial
\text{Char}K< \left\lbrace S_{k}\right\rbrace>$
is Abelian, we have that the convolutive product in our case reduces to
\begin{equation}
\e^{\sum_{i}\alpha_{i} Z_{i}} \ast\e^{\sum_{j}\beta_{j}Z_{j}}=\e^{\sum_{i}(\alpha_{i}
+\beta_{i})
 Z_{i}}.\label{20a}
\end{equation}

\subsubsection{The Hopf algebra $K<\{\psi_{k}\}>$}

The Hopf algebra $K< \left\lbrace S_{k}\right\rbrace>$ induces another Hopf algebra
$K<\{\psi_{k}\}>$ by applying a change to normal coordinates
\cite{chryss} to our original Poincar\'e-Birkhoff-Witt basis,
 constructed from the $S_{i}$ coordinates, by means of the map with the canonical element
 ${\bf C}=\e^{\sum_{i} Z_{i}\otimes\psi_{i}}$ which acts as an identity map, {\it i.e.}:
\begin{equation}
\langle  \e^{\sum_{i} Z_{i}\otimes\psi_{i}} , S_{k}\otimes {\text
id }\rangle=S_{k}.\label{20b}
\end{equation}

It is not difficult to verify that the non-linear relation between the $\psi_{k}$'s
and $S_{i}$'s
resulting from (\ref{20b}) is given by (\ref{11}), and that the $\psi_{k}$'s acquire a
primitive coproduct
\begin{equation}
\Delta \psi_{k}= {\bf 1}\otimes\psi_{k}+ \psi_{k}\otimes {\bf 1},\label{17a}
\end{equation}
and an antipode given by
\begin{equation}
S(\psi_{k})=-\psi_{k}.\label{18}
\end{equation}
Note that since (\ref{11}) is invertible $K< \left\lbrace S_{k}\right\rbrace>\simeq K< \left\lbrace \psi_{k}\right\rbrace >$.

As a parenthetical remark we point out that the normal coordinates that we construct here
differ from those recently discussed in \cite{chryss} in relation to the Hopf algebra
of rooted trees, by the fact that in the context of the latter the
comultiplication given by (\ref{17a}) corresponded to non-branched trees,
and that for the more general case of branched trees it was determined by application
of the Baker-Hausdorff-Campbell formula.\\

%It is important to stress at this stage that since it is only the functionals
%$\psi_{k}[J]$ that appear at the end in the perturbative expressions for the
%transition amplitudes and, as noted, there are at least three Hopf algebras
%for which these functionals are primitives, it would appear that we have some
% degree of ``embarrass de richesse" as for the choice of Hopf algebras in our formalism.
%This is not quite so, first because as we have shown above we can view the relation
%of the $S_{i}[J]$ to the $\psi_{k}[J]$ as a mere non-linear change to normal
%coordinates. Furthermore, since the commutators of the $\Psi_{k}$ have a primitive
%coproduct they are elements of a free Lie algebra ${\frak L}_{K}(\{\Psi_{k}\})$ so
%that the algebras $K<Y>$, $K<\{\psi_{k} \}>$ and  are also related by the following diagram

%\[
%\begin{array}{cccc}
%K< \left\lbrace \psi_{k}\right\rbrace > & \subset K< \left\lbrace
%\Psi_{k}\right\rbrace > & \hookrightarrow & K < Y >
%\\
%&\bigcup & &    \bigcup \\
%& {\frak L}_{K}(\{\Psi_{k}\}) & \hookrightarrow & {\frak L}_{K}(Y) \\
%&\bigcup & &    \bigcup \\
%&\psi_{k} & &\psi_{k}
%\end{array}
%\]

%Hence, keeping in mind the larger mathematical structure of the free algebra $K<Y>$
%(such as the cyclicity of the primitives and the relation of the Lie monomials to sums
%of Feynman diagrams, which we shall be using in future work), we assume this algebra
%for the remainder of our presentation.


\subsection{Hopf Primitives and Connected Green Functions}

 In order to establish explicitely the relation of the $\psi_{k}[J]$ functionals to the Green functions,
observe that because the n-legged connected Green functions in PQFT are obtained from the functional variation
\begin{equation}
G_{E}^{(n)}(x_{1},\dots, x_{n})= -\frac{\delta^{n}Z_{E}[J]}{\delta J_{1}\dots \delta J_{n}}
\vline\;_{J=0},\label{14}
\end{equation}
we have that
\begin{equation}
\psi_{k}[J]=\frac{1}{\lambda^{k}}\sum_{n=0} \frac{1}{(n)!}\langle
G_{k}^{(n)}(x_{1}, \dots, x_{n})J_{1}\dots J_{n}\rangle
,\label{14a}
\end{equation}
where $G_{k}^{(n)}$ are the Euclidean Green functions resulting
from adding all the connected Feynman graphs with $k$ vertices and
${n}$ external legs ($n$ is even for the $\varphi^{4}$ theory). Note also that (\ref{14a}) contains
contributions from the Green functions $G_{k}^{(0)}$ which
correspond to vacuum terms. These contributions may be absorbed in
the $\ln W_{0}[0]$
  term in (\ref{8}) via the normalization constant $N$.\\

%The connected Green functions are related to the more easily
%renormalizable 1PI Green functions, $\Gamma_{E}^{(2n)}$, generated
%by the effective action, by functional differentiation of the
%equation:
%\begin{equation}
%\delta^4(x-y)= \int d^4 z \;\frac{\delta^2 \Gamma}{\delta
%\varphi(y)\delta\varphi(z)}\frac{\delta^2 Z}{\delta J(z)\delta J(x)} = \int d^4
%z \; \Gamma^{(2)}_E (y,z) G^{(2)}_E(z,x),
%\end{equation}
%so all connected Green functions are determined in terms of the
%1PI Green functions and vice versa. However, since we are not so much concerned
%with explicit calculations, and since the connected Green functions are more
%directly related to the Hopf algebra primitives $\psi_{k}$'s, we shall
%continue using them in our presentation.\\

The analytical expressions for the graphs composing the Green
functions result, in general, in ultraviolet divergences, and the
standard procedure for removing them is to first apply dimensional
regularization and then successively the Forest Formula. It is at
this stage where the Kreimer-Connes Hopf algebra formalism
provides an important insight into the underlying mathematics
behind the Forest Formula of renormalization. Indeed, by noting
that each Feynman diagram corresponds to a decorated rooted tree
(or a sum of decorated rooted trees in the case of overlapping
diagrams) and that these rooted trees, as well as the diagrams
themselves, are the generators of respective universal enveloping
Hopf algebras, these authors have shown that the operation with a
twisted antipode on the algebra provides a systematic application
of the Forest Formula and, consequently, a systematic procedure
for generating the counterterms needed by the theory in order to
cancel the unwanted infinities. Connes and Kreimer further show
\cite{conkre2} the relation between the algebra of characters,
dual to their Hopf algebra, and the Birkhoff algebraic
decomposition. A detailed exposition of these ideas may be found
in the papers cited in the Introduction.\\
Here we only stress once
more the fact that in our approach we deal with the Hopf primitives
$\psi_{k} \in {\frak L}_{K}(Y)$
rather than with Feynman
graphs directly. As we have seen, these Hopf primitives are
Lie polynomials where each monomial is a
Hall tree that itself corresponds to a sum of Feynman diagrams
with $k$-vertices each and an equal number of external legs and loops.
This last observation can be read off directly from the word composed
by the foliage of a Hall tree. First because the number of external legs $E$
in the Feynman diagrams constituting a Hall tree is given by
\begin{equation}
E=2N_{Z^{0}}-N_{\frac{\delta}{\delta J}} V,\label{f.50}
\end{equation}
where
\begin{eqnarray*}
N_{Z^{0}}&=&\text {number of times the letter $Z^{0}$ appears in the word}\\
N_{\frac{\delta}{\delta J}}&=& \text {degree of the functional derivation in the potential}\\
V&=&\text {number of vertices}\\
 &=&\text {number of times the letter $\langle V\rangle  $ appears in the word},
\end{eqnarray*}
so all Feynman diagrams composing a given Hall tree have the same number of external legs.
Second, because the topology of the Feynman diagrams implies that the number of
loops is fixed by the number of vertices and the number of external legs
by the relation
\begin{equation}
(N-2)V=E+2L-2,\label{f.51}
\end{equation}
where $N$ is the total number of legs per vertex ($N$=4 for the $\phi^{4}$ theory),
and $L$ is the number of loops, so also all Feynman diagrams composing a Hall tree have the same loop number. \\

In order to see how the above discussed algebras and their primitives relate to renormalization, let us begin
by reviewing the essential ideas behind the algebraic Birkhoff decomposition.
% It is also interesting to note that only for
%diagrams containing no divergent subdiagrams there corresponds
%(modulo irreducibility) a Lie monomial. The implications of this
%in terms of the specific form of containment of ${\frak L}_{K}(Y)$
%in the algebra of Feynman diagrams is under current investigation.\\


\section{The Algebraic Birkhoff Decomposition\\
 and Renormalization}

Let ${\bf A}$ denote the (unital) $K$-algebra of meromorphic functions
on the Riemann sphere with poles at the origin, and
let $Hom_{K-{\text alg}}(K<Y>
,{\bf A})$ be the set of unital $K$-algebra homomorphisms
 $\phi:\{ K<Y>\} \rightarrow{\bf A}$, considered as a $K$-algebra
 under the convolution product \cite{kastler}:
\begin{equation}
(\phi \star \phi^{\prime})(w)=m_{{\bf
A}}(\phi\otimes\phi^{\prime})(\Delta w), \;\phi,\phi^{\prime}\in
Hom_{K-alg}(K<Y>,{\bf A}),\label{22.a}
\end{equation}
where $w$ is any word over $Y$.

Note that the elements in $Hom_{K-alg}(K<Y>,{\bf A})$ are in a one-to-one
correspondence with the {\bf A}-loop $\gamma: \text{\bf PC}^{1}-\{0,\infty\}
\rightarrow {\mathcal G}$, via the bijection:
\begin{equation}
\gamma\leftrightarrow \underline{\gamma}:\langle  \gamma(z), w
\rangle=\underline{\gamma}(w)(z), \;\;\; w\in K<Y>,\; \; z\in \;
\Bbb C -\{0\}.\label{22c}
\end{equation}
We therefore have the group homomorphism
\begin{equation}
\gamma=\gamma^{\prime}\gamma^{\prime\prime}\leftrightarrow\underline{\gamma}=
\underline{\gamma}^{\prime}\star\underline{\gamma}^{\prime\prime}.\label{22d}
\end{equation}

Furthermore, let ${\bf A}={\bf A}_{-}\oplus {\bf A}_{+}$ be a
Birkhoff sum of the $K$-linear multiplicative subspaces ${\bf
A}_{-}$ and  ${\bf A}_{+}$, where ${\bf A}_{-}= \{\text
{polynomials in $z^{-1}$ without constant}$ $\text{term}\}$ and
${\bf A}_{+}= \{\text {Restriction to $(\Bbb C -\{0\})$ of
functions in ${\bf Holom}(\Bbb C)$}\}$. The projection $T: {\bf
A}\rightarrow {\bf A}_{-}$ satisfying the multiplicative constraints
($T$ is also known as the Rota-Baxter operator and (\ref{23a}) bellow as the
Rota-Baxter relations)
\begin{equation}
T(ab) + (Ta)(Tb) = T[(Ta)b + a(Tb)], \; a,b\in {\bf A},\label{23a}
\end{equation}
allows us to define the K-linear map $\phi_{-}:K<Y>\rightarrow {\bf A}_{-}$ by
means of the iterative equations
\begin{eqnarray}
\phi_{-} (w)&=&-T[\phi(w)+ m_{{\bf A}}(\phi_{-}\otimes\phi)\tilde{\Delta}(w)],\label{24a}\\
\phi_{-} ({\bf 1})&=& {\bf 1}_{{\bf A}}.\label{24b}
\end{eqnarray}
The homomorphism $\phi_{+}$, having its range in ${\bf A}_{+}$, is then given by the algebraic
 Birkhoff decomposition
\begin{equation}
\phi_{+}=\phi_{-}\star \phi.\label{25f}
\end{equation}
Moreover, since the Lie polynomials $\psi_{k}$'s are primitive to start with, we have that
(\ref{24a}) results in
\begin{equation}
\phi_{-} (\psi_{k})= -T[\phi(\psi_{k})],\label{25}
\end{equation}
so $\phi_{-} (\psi_{k})$ is minus the principal part of the Laurent series expansion of
$\phi (\psi_{k})$ about the origin,
while (\ref{22.a}) and (\ref{25}) yield
\begin{equation}
(\text {id}-T)\phi(\psi_{k})=\phi_{+}(\psi_{k})\in {\bf A}_{+}.\label{25e}
\end{equation}

Observe that (\ref{24a}) may be related to the Connes-Kreimer twisting ansatz for the
antipode by making the identifications
\begin{equation}
 S_{R}(\phi(\psi_{k}))=\phi_{-} (\psi_{k}),\label{25c}
\end{equation}
and
\begin{equation}
R=T,\label{25cc}
\end{equation}
where the subscript $R$ here denotes the specific choice
of a renormalization scheme satisfying (\ref{23a}).

Let us now apply formally the homomorphism $\phi\in
Hom_{K-alg}(K<Y>,{\bf A})$ to the antipode axiom for
$\psi_{k}\in \ker \varepsilon$. We thus get
\begin{equation}
\begin{split}
\phi (m\circ (S\otimes \text{id})\Delta(\psi_{k}))&=m_{\bf
A}\circ(\phi\circ S\otimes
 \phi)\Delta(\psi_{k})\\ &
=m_{\bf A}\circ(S_{\bf A}\otimes
\text{id})(\phi\otimes\phi)\Delta(\psi_{k})=0,\label{25a}
\end{split}
\end{equation}
where $m_{\bf A}$ and $S_{\bf A}$ indicate that these operators now act on the
Hopf algebra of functions in ${\bf A}$.
This result, however, is not very useful for renormalization
purposes so, in order to be able to take into account the degree
of freedom allowed in renormalization due to the fact that
arbitrary scales occur as a result of the dimensional
regularization procedure, we
replace the antipode $S_{\bf A}$ in (\ref{25a}) by a twisted
antipode $S_{R}$, using  the relation $\phi\circ S=S_{R}\circ
\phi$, following Connes and Kreimer procedure. We thus have
\begin{equation}
m_{\bf A}\circ(S_{R}\otimes \text{id})(\phi\otimes\phi)\Delta(\psi_{k})\approx 0,\label{25b}
\end{equation}
where now the right side is no longer strictly zero for
$\psi_{k}\in  \ker\varepsilon$, but is only zero modulo terms
which we shall show correspond to the renormalized diagrams contained in the
$\psi_{k}$.\\
Making use of (\ref{17a}), (\ref{25c}) together with (\ref{25}) and (\ref{25e})
in the left side of (\ref{25b}), we get
\begin{equation}
m_{\bf A}\circ(\phi_{-}\otimes\phi)\Delta(\psi_{k})=(\text {id}-T)
\phi(\psi_{k})=\phi_{+}(\psi_{k}).\label{25d}
\end{equation}

Consequently, the antipode axiom has now become a homomorphism map from
primitives in the algebra $K<Y>$ to
the algebra of holomorphic functions with range in ${\bf A}_{+}$.\\

In the next subsection we will show how the Hopf primitives $\psi_{k}[J]$ together
with their linear algebraic Birkhoff decomposition (\ref{25e}),
and the specific choice of
the Mass Independent Renormalization Scheme, for which we know
that there are no finite contributions to the counterterms and for
which it can be shown \cite{kreimer6} that the multiplicativity of
$S_{R}$ is compatible with the multiplicative constraints
(\ref{23a}), lead to the basic equations of renormalization and the finite
and physically correct renormalized Green functions in PQFT.

\subsection{Renormalization}

Let us denote by $\epsilon$ the radius of an infinitesimal circle in the
Riemann sphere, and begin by identifying the mapping $\phi_{\epsilon}
\in Hom_{K-alg}(K<Y>,{\bf A})$ of the Hopf algebra indeterminates $\psi_{k}$
with the functionals $\psi_{k}[J]$,
after regularizing and redefining $\lambda$ so it remain dimensionless in
the regularized space of dimension $d=4-4\epsilon$. That is, we let
\begin{equation}
\phi_{\epsilon}(\psi_{k})= \psi_{k}[J],\;\;\; \text {with}\;\;
\lambda\rightarrow\lambda\;\mu^{2\epsilon}. \label{28}
\end{equation}
From (\ref{8}), (\ref{25e}), (\ref{25}) and (\ref{14a}) we then
have
\begin{equation}
Z_{E}[J]-\sum_{k\geq 1} \lambda^{k}\phi_{-}(\psi_{k})=-\ln W_{0}[0]+ Z^{0}[J] -
\sum_{k\geq 1}
\lambda^{k}\phi_{+}(\psi_{k}).\label{28a}
\end{equation}
Here
\begin{equation}
\begin{split}
\phi_{+}(\psi_{k})=\frac{1}{\lambda^{k}}\sum_{n\geq 1}
\frac{1}{(n)!}\left(\langle  G_{k}^{(n)}(x_{1},
\dots, x_{n})J_{1}\dots J_{n} \rangle \right. \\
-\langle T[G_{k}^{(n)}(x_{1},\dots, x_{n})]J_{1}\dots
J_{n} \rangle\Big),\label{28bb}
\end{split}
\end{equation}
where we have put the projector $T$ inside of the integration in the second
term of (\ref{28bb}) after taking into account that the currents $J_{i}$ are
good test functions, so the pole structure of the integrals is determined
by that of the Green functions.
In addition, this projection by $T$ to $\bf {A}_{-}$ of the connected Green
functions is taken to be given by the Mass Independent Renormalization Scheme.\\

Note that since the power to which the letter $\langle V \rangle$
occurring in a word describing the polynomials in ${\frak
L}_{K}(Y)$ corresponds to the number of vertices in each of its
associated diagrams, we can therefore use this criterion to
introduce a gradation in the free Lie algebra  ${\frak L}_{K}(Y)$.
Moreover, since the diagrams in each $\psi_{k}[J]$ have the same
number of vertices and since $\psi_{k}[J]\in {\frak L}_{K}(Y)$,
their subindex denotes their gradation in this algebra. Also note
that the linear decomposition $\phi=\phi_{+}-\phi_{-}$ used to
derive (\ref{28a}) is only made possible
by the primitiveness of the $\psi_{k}$'s in the Hopf algebra here considered.\\

Now, even though the left side of (\ref{28a}) is finite (because the right hand
is by construction)  when $\epsilon \rightarrow 0$, the theory is only
renormalizable if the terms \break
$(- \sum_{k\geq 1} \lambda^{k}\phi_{-}(\psi_{k}))$ in the left
side of (\ref{28a}) can be canceled out by a counterterm Lagrangian which, in
the regularized dimension $d=2(2-\epsilon)$, is of the form
\begin{equation}
{\mathcal L}_{\text {c.t.}}=\frac{1}{2}A\;
\partial_{\mu}\varphi\partial_{\mu}\varphi +\frac{1}{2}B\;
m^{2}\varphi^{2}+\frac{\lambda}{4!}\mu^{2\epsilon}C \varphi^{4}.\label{28e}
\end{equation}
Adding (\ref{28e}) to the original $\varphi^{4}$ Lagrangian
\begin{equation}
{\mathcal L}= \frac{1}{2}\partial_{\mu}\varphi\partial_{\mu}\varphi
+\frac{1}{2}m^{2}\varphi^{2}+\frac{\lambda}{4!}\mu^{2\epsilon} \varphi^{4},\label{28d}
\end{equation}
yields then the renormalized Lagrangian ${\mathcal L}_{R}= {\mathcal L}+
{\mathcal L}_{\text {c.t.}}$ of the theory,
after taking $\lim\epsilon\to 0$.\\

In terms of the bare field and bare parameters
\begin{equation}
\begin{split}
\varphi_{b}&\equiv (1+A)^{1/2}\varphi\equiv
Z_{\varphi}^{1/2}\varphi,  \\
m_{b}^{2} &
\equiv m^{2}\left(\frac{1+B}{1+A}\right)=m^{2}(1+B)Z_{\varphi}^{-1},
\label{a.2} \\
\lambda_{b}&\equiv\lambda\mu^{2\epsilon} (1+C)Z_{\varphi}^{-2},
\end{split}
\end{equation}
we can write
\begin{equation}
{\mathcal L}_{R}= \frac{1}{2}\partial_{\mu}\varphi_{b}\partial_{\mu}\varphi_{b}
+\frac{1}{2}m_{b}^{2}\varphi_{b}^{2}+\frac{\lambda_{b}}{4!}\mu^{2\epsilon}
\varphi_{b}^{4}.\label{a.4}
\end{equation}
The $(Z_{E}[J])_{b}$ associated to (\ref{a.4}), where the subscript $b$
denotes evaluation in terms of the bare parameters, can be written as
\begin{equation}
\begin{split}
(Z_{E}[J])_{b}&=-\ln(W_{0}[0])_{b} \\
&\ -\sum_{k\geq 0}\sum_{n\geq 1} \frac{1}{(n)!}\langle
(G_{k})_{b}^{(n)}(x_{1},\dots, x_{n}; \lambda_{b},
m_{b},\epsilon)(J_{1})_{b},\dots, (J_{n})_{b}\rangle
\hspace{4in},\label{a.5a}
\end{split}
\end{equation}
where we have incorporated the $(Z^{0}[J])_{b}$ in the last summand by making
use of (\ref{14}). Furthermore, note that by adjusting the $N$ in $W_{0}[0]$
we can set it equal to one, and rescaling in addition the bare currents by means of
\begin{equation}
(J_{k})_{b}=Z_{\varphi}^{-1/2}(\lambda,\frac{m}{\mu},\epsilon) J_{k},\label{a.7}
\end{equation}
(the inverse of the first equation in (\ref{a.2})) and Fourier
transforming, we get \begin{equation} (\tilde
{Z}_{E}[\tilde J])_{b}=-\sum_{n\geq 1}\sum_{k \geq 0}
\frac{1}{(n)!}\langle
(\tilde{G}^{(n)}_{k})_{b}(p_{1}, \dots, p_{n};\lambda_{b},
m_{b},\epsilon)Z_{\varphi}^{-\frac{n}{2}}\tilde{J}_{1}\dots
\tilde{J}_{n}\rangle  , \label{a.88}
\end{equation}
where a symbol with a tilde is used to denote the Fourier
transformed variable, and $\langle \;\rangle  $
 stands now for integration over momentum space.\\

In the Mass Independent Scheme the counterterms contain no finite contributions and
are polynomials in inverse powers of $\epsilon$ with coefficients depending only on $\lambda$,
so renormalization implies writing the bare parameters as \cite{collins}:
\begin{equation}
\begin{split}
\lambda_{b}&=\mu^{2\epsilon}[\lambda +\sum_{j\geq 1}\sum_{k\geq j}
\frac{a_{jk}
 \lambda^{k}}{\epsilon^{j}}],\\
m_{b}^{2}&=m^{2}[1+\sum_{j\geq 1}\sum_{k\geq j}\frac{b_{jk}
\lambda^{k}}{\epsilon^{j}}],
\label{a.14}\\
Z_{\varphi}&=1+\sum_{j\geq 1}\sum_{k\geq j}\frac{c_{jk}
\lambda^{k}}{\epsilon^{j}},
\end{split}
\end{equation}
and evaluating the coefficients in (\ref{a.14}). This is done by
substituting these equations into the right side of (\ref{a.88})
and applying the requirement of finiteness of the resulting
expression, order by order in the $\lambda$. Finding the
coefficients in (\ref{a.14}) to all orders of $\lambda$ is
tantamount to renormalizing the theory. One then obtains in
general that
\begin{equation}
\begin{split}
\sum_{n\geq 1}\sum_{k \geq 0} \frac{1}{(n)!}\langle
(\tilde{G}^{(n)}_{k})_{b}(p_{1}, \dots, p_{n};\lambda_{b},
m_{b},\epsilon)Z_{\varphi}^{-\frac{n}{2}}\tilde{J}_{1}
\dots \tilde{J}_{n}\rangle  =\hspace{1in}\\
-\tilde Z_{E}[\tilde{J}] +\sum_{k\geq 1}
\lambda^{k}\tilde{\phi}_{-}(\psi_{k}) +\sum_{n\geq 1}\sum_{k \geq
1}\lambda^{k}\langle \tau_{k}(p_{1},\dots, p_{n}, m, \mu, a, b,
c)\tilde{J}_{1}\dots \tilde{J}_{n}\rangle  ,\label{a.155}
\end{split}
\end{equation}
where $\tau_{k}(p_{1},\dots, p_{n}, m, \mu, a, b, c)$ are finite
functions in ${\bf A}_{+}$, depending on the previously evaluated
coefficients in (\ref{a.14}), and are also explicitely determined
by that process. Substituting (\ref{28a}) in the right side of
(\ref{a.155}) and again setting $W_{0}[0]=1$ by choosing the
ill-defined constant $N$ in it appropriately, we finally arrive at
\begin{equation}
\begin{split}
\sum_{k \geq 0}(\tilde{G}^{(n)}_{k})_{b}(p_{1},\dots, p_{n};\lambda_{b}, m_{b},
\epsilon)Z_{\varphi}^{-\frac{n}{2}}=\sum_{k\geq 0}\tilde{G}_{k}^{(n)}(p_{1},\dots, p_{n};
\lambda ,m, \mu,\epsilon),\\
\;n\geq 1,\label{a.156}
\end{split}
\end{equation}
where we have defined
\begin{equation}
\begin{split}
{\tilde G}_{k}^{(n)}(p_{1},\dots, p_{n};\lambda ,m,\mu,\epsilon)
:=[{\tilde G}_{k}^{(n)}(p_{1},\dots, p_{n})
 -T({\tilde G}_{k}^{(n)}(p_{1},\dots, p_{n})]\\
+ \lambda^{k}\tau_{k}(p_{1},\dots, p_{2n}, m,
 \mu, a, b, c),\  k\geqslant 1 \label{aa.6}
\end{split}
\end{equation}
These are the renormalized Green functions of the theory.\\

For the scalar $\varphi^{4}$ theory used here to exemplify our discussion,
one finds that to order $\lambda^{2}$:
\begin{eqnarray}
\lambda_{b}&=&\mu^{2\epsilon}[\lambda + \frac{3\lambda^{2}}{32\pi^{2}\epsilon}+
{\mathcal O}(\lambda^{3})],\nonumber\\
m_{b}^{2}&=&m^{2}[1+\frac{\lambda}{32\pi^{2}\epsilon}+
 \frac{\lambda^{2}}{24(16\pi^{2})^{2}\epsilon}+
 \frac{2\lambda^{2}}{(32\pi^{2})^{2}\epsilon^{2}}+
{\mathcal O}(\lambda^{3})],\nonumber\\
(Z_{\varphi})_{b}&=&1-\frac{\lambda^{2}}{24(16\pi^{2})^{2}\epsilon}+
{\mathcal O}(\lambda^{3}),\label{a.15c}
\end{eqnarray}
and
\begin{equation}
\begin{split}
\tau_{1}(p_{1},p_{2}, m, \mu, a, b, c)& =0,\\
\tau_{2}(p_{1},p_{2}, m, \mu, a, b, c)& =
-\frac{4m^{2}}{(32\pi^{2})^{2}(p^{2}+m^{2})^{2}}\left[\frac{1}{2}(\frac{\pi^{2}}{6}-1
+(1-\gamma)^{2})\right.  \\
& \  \  \ \left.
-(1-\gamma)\ln\frac{m^{2}}{4\pi\mu^{2}}\right],\label{aa.15}
\end{split}
\end{equation}
where $\gamma$ here denotes the Euler-Mascheroni constant.\\

It is at this stage that we can make contact with the Forest
Formula or with its Hopf algebra equivalent of Kreimer and Connes.
Indeed, we have seen that the theory is renormalizable if and only if
(\ref{a.156}) can be solved consistently for the coefficients of the bare
parameters. But only then we will have that
%Obviously, the relation between the $\tau_{k}$ contributions to the renormalized
%Green functions, as derived above, and the Hopf Algebra of Renormalization is given by
\begin{equation}
\tau_{k}(p_{1},p_{2}, m, \mu, a, b, c)=\frac{1}{\lambda^{k}}
\lim_{\epsilon\rightarrow 0}({\text id}-T)\left[ m_{{\bf A}}(\phi_{-}
\otimes\phi)\tilde{\Delta}\left( \hbox{\epsfig{file=fig1.eps,height=6mm,width=1.3cm}}\right)
 \right] , \label{aaa.1}
\end{equation}
where the right side corresponds to the twisted antipode axiom formulation of
the Hopf Algebra of Renormalization (without the primitive contributions of the
coproduct), and the blob \epsfig{file=fig1.eps,height=6mm,width=1.3cm} with
$2n$ external legs denotes the $k$-vertex 1PI Green functions.
Thus for our example
\begin {equation}
\begin{split}
\tau_{2}(p_{1},p_{2}, m, \mu, a, b, c)&=\frac{1}{\lambda^{2}}
\lim_{\epsilon\rightarrow 0}({\text id}-T)\left[m_{{\bf
A}}(\phi_{-}
\otimes\phi)\tilde{\Delta}\left(\hbox{\epsfig{file=fig2.eps,height=5mm,width=1.3cm}}
 + \hbox{\epsfig{file=fig3.eps,height=5mm,width=1.3cm}}\right) \right] \\
=&\frac{1}{\lambda^{2}}\lim_{\epsilon\rightarrow 0}({\text id}-T)
\left[3\phi_{-}\left(
\hbox{\epsfig{file=fig4.eps,height=5mm,width=1.3cm}}\right)
\phi\left( \hbox{\epsfig{file=fig5.eps,height=5mm,width=1.3cm}}\right) \right. \\
& \ +\left. \phi_- \left(\hbox{\epsfig{file=fig5.eps,height=5mm,width=1.3cm}} \right) 
 \phi\left( \hbox{\epsfig{file=fig5.eps,height=5mm,width=1.3cm}}\right) \right]\\
=&\lim_{\epsilon\rightarrow 0}({\text
id}-T)\frac{1}{(p^{2}+m^{2})^{2}}\left[-\frac{4m^{2}}{(32\pi^{2})^{2}\epsilon}
\left(\frac{1}{\epsilon}+ 1-\gamma 
\right. \right. \\
&  \ -\ln\left(\frac{m^{2}}{4\pi\mu^{2}}\right)
+\frac{\epsilon}{2}\left(\frac{\pi^{2}}{6}-1+(1-\gamma)^{2}\right)\\
& \left. \left. -\epsilon(1-\gamma)
\ln\left(\frac{m^{2}}{4\pi\mu^{2}}\right)+ {\mathcal O}(\epsilon^{2})\right)\right]\\
=&-\frac{4m^{2}}{(32\pi^{2})^{2}(p^{2}+m^{2})^{2}}\left(\frac{1}{2}\left(\frac{\pi^{2}}{6}-1+
(1-\gamma)^{2}\right) \right. \\
& \ \left. -(1-\gamma)\ln\left(\frac{m^{2}}{4\pi\mu^{2}}\right)\right),\label{aaa.2}
\end{split}
\end{equation}

