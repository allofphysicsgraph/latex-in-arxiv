\begin{equation}}
\def\bc             {boundary condition}
\def\cala          {{\mathcal A}}
\def\calc          {{\mathcal C}}
\def\calg          {{\mathcal G}}
\def\calh          {{\mathcal H}}
\def\call          {{\mathcal L}}
\def\calm          {{\mathcal M}}
\def\calv          {{\mathcal V}}
\def\complex       {{\mathbb C}}
\def\eE            {{\rm e}}
\def\End           {{\rm End}}
\def\eps           {\varepsilon}
\def\eq            {\,{=}\,}
\newcommand\erf[1] {(\ref{#1})}
\def\F             {{\mathbb F}}
\newcommand\Frac[2]{\mbox{\large$\frac{#1}{#2}$}}
\def\Hom           {{\rm Hom}}
\newcommand\hsp[1] {\mbox{\hspace{#1 em}}}
\def\id            {{\rm id}}
\def\ii            {{\rm i}}
\def\iN            {\,{\in}\,}
\newcommand\labl[1]{\label{#1}}
\def\Lie           {{\rm Lie}\,}
\newcommand\nxt[1] {\\\raisebox{.12em}{\rule{.35em}{.35em}}\hsp{.6}#1}
\def\one           {{\bf 1}}
\def\oti           {\,{\otimes}\,}
\def\reals         {{\mathbb R}}
\def\RI            {Riemann}
\def\rmd           {{\rm d}}
\def\scs           {\scriptstyle }
\def\twodim        {two-dimensional}
\def\U             {{\rm U}}
\def\zet           {{\mathbb Z}}
% End of own macros 

\begin{document}

\begin{flushright} \mbox{$\,$}\\[-13mm]
                             {\tt HU-EP-03/02}\\[1mm] {\tt hep-th/0301181}
\\[22mm]{\,}\end{flushright}

\title [Lie algebras, Fuchsian differential equations and CFT correlators]
       {Lie algebras, Fuchsian differential equations
       \\ and CFT correlation functions}

\author{J\"urgen Fuchs}
\address{Avdelning fysik, Karlstads universitet,
Universitetsgatan 5, S\,--\,651\,88\, Karlstad}
\email{jfuchs@fuchs.tekn.kau.se}
\author{Ingo Runkel}
\address{ Institut f\"ur Physik, HU Berlin,
Invalidenstr.\ 110, \,D\,--\,10115 Berlin}
\email{ingo@physik.hu-berlin.de}
\author{Christoph Schweigert}
\address{Institut f\"ur theor.\ Physik,
RWTH Aachen, Sommerfeldstr.\ 14, D\,--\,52074\, Aachen}
\email{schweige@physik.rwth-aachen.de}

\subjclass[2000]{81R10,81T40,33C80,18D10}
\date{December 31, 2002}
\dedicatory{}

\keywords{Hypergeometric functions, conformal blocks, (modular) tensor 
categories, conformal field theory, WZW theories}

\begin{abstract}
Affine Kac-Moody algebras give rise to interesting systems of differential 
equations, so-called
Knizhnik-Zamolodchikov equations. The monodromy properties of their solutions
can be encoded in the structure of a mo\-du\-lar tensor category on 
(a subcategory
of) the representation category of the affine Lie algebra. We discuss the 
relation between these solutions and physical correlation functions
in two-dimensional conformal field theory.
In particular we report on a proof for the existence of the latter on 
world sheets of arbitrary topology. 
\end{abstract}

\maketitle 


\section{Some venerable differential equations}

One of the surprises in the theory of Kac-Moody algebras is the observation
that they give a new and powerful handle on problems that, a priori, do not
have an algebraic flavor, including such which are not even genuinely 
infinite-dimensional. In this contribution we discuss one such application of
structures related to affine Kac-Moody algebras: properties of solutions to 
differential equations.

\medskip

In the middle of the 18th century Euler, Cauchy and other analysts, 
the successors of the founding fathers like 
Newton, Leibniz, and Johann and Jakob Bernoulli, developped 
a strong interest in ``special functions'' and series. Arguably, the 
simplest non-trivial series one can think of is the geometric series
  $$ f(z) = 1+z + z^2 +\ldots  $$
It obeys the ordinary differential equation
  $$ z\,(1\,{-}\,z)\, f'' + (1\,{-}\,3z)\, f' - f = 0 \, . $$
To be of interest in complex analysis, a generalization of the geometric series 
should, of course, possess a non-zero radius of convergence. Furthermore,
a useful guiding principle is that it should also obey a simple differential 
equation that allows one to determine its main properties. Around the year 
1750, such considerations lead Euler to the {\em hypergeometric series}
  $$ F(\alpha,\beta,\gamma;z):= 
  1 + \sum_{n=1}^\infty  \frac {\alpha(\alpha{+}1)\cdots (\alpha{+}n{-}1)
  \,\beta(\beta{+}1)\cdots(\beta{+}n{-}1)} {n!\, \gamma(\gamma{+}1)\cdots
  \,(\gamma{+}n{-}1)}\, z^n \,.$$
It obeys the hypergeometric differential equation
  \begin{equation}
  z(z\,{-}\,1)\, w'' + [(\alpha\,{+}\,\beta\,{+}\,1)\,z -\gamma]\,w' 
  +\alpha\beta\, w = 0 \,. \labl{dg} \end{equation}
This linear differential equation is Fuchsian \cite{fuch},
i.e.\ enjoys the particularly nice property that
all three singular points that the coefficient functions acquire upon 
division by $z(z{-}1)$ (namely 0, 1, and $\infty$) are regular singularities.
This means that when one approaches
such a point in a wedge region of opening angle less than $\pi/2$ the 
solutions will not diverge stronger than polynomially.

It is quite remarkable that Euler's interest in this series did {\em not\/} 
arise from concrete applications. Indeed, in his famous 1857 paper \cite{riem}
Riemann calls Euler's motivation a ``theoretisches Interesse'', while
he points out there are ``numerous applications in physical and astronomic
investigations''. (This makes a strong point in favour of
pure science that is motivated by internal questions.)

Differential equations of the type \erf{dg} and their solutions are
indeed ubiquitous in mathematical physics. They include
\nxt Legendre polynomials, introduced by Legendre around 1800 in his
     study of gravitational potentials:
  $$ \frac1{\sqrt{1-2\rho z + \rho^2}} = \sum_{n=0}^\infty \rho^n P_n(z)
  \,; $$
\nxt Bessel functions, describing the radial part of the wave function of
     a free particle in quantum mechanics;
\nxt Laguerre polynomials, the radial part of the hydrogen atom wave
     functions.
\\
The last two classes of functions are {\em confluent\/} hypergeometric
functions; they are defined by the limit
  $$ \Phi(\alpha,\gamma;z ) = \lim_{\beta\to\infty} 
  F(\alpha,\beta,\gamma;\Frac z\beta) \, . $$

So far there is clearly not much algebra in the game. A first hint on the
possible relevance of algebra comes from the observation that
a deeper understanding of many special functions is afforded by relating 
them to representation functions of suitable
Lie groups. For example, Bessel functions are related to representation
functions of irreducible unitary representations of the group of
motions of the Euclidean plane. (For a review of the relation 
between special functions and Lie groups, see e.g.\ \cite{HElg,DIeu}.)

In this contribution, we consider interesting generalizations of
hypergeometric differential equations that have their origin in 
representation-theoretic structures, involving in particular the 
representation theory of (classes of) Kac-Moody algebras. 
They are largely motivated by models of conformal quantum field theory.

In Section 2 we introduce a system
of differential equations. In section 3 properties of their solutions
are discussed, in particular the asymptotic behaviour
and monodromies. These properties possess a convenient interpretation
in terms of tensor categories. In section 4 we finally give 
the underlying motivation from theoretical physics and present
a theorem about correlation functions in conformal field theories.

\section{The Kniznik-Zamolodchikov equation}

Let us select a real compact Lie group $G$. It admits a bi-invariant metric
that allows us to introduce dual bases $\{a_l\}$ and $\{a^l\}$ of the Lie
algebra $\Lie G$ of $G$. The element
  $$ \Omega := \sum_{l=1}^{\dim\,G} a_l \oti a^l \,\in \Lie G \oti \Lie G $$ 
does not depend on the choice of bases.
Moreover, its action on the tensor product $V_1\oti V_2$ of two 
finite-dimensional $G$-modules $V_1$ and $V_2$ is diagonalizable.

We are interested in functions 
  \be f: \quad \calm_N \to V_1\otimes \cdots \otimes V_N  \labl7
  \end{equation} 
