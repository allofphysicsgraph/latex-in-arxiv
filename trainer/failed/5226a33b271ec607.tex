\begin{equation}}
\newcommand{\bea}{\begin{eqnarray}}

\newcommand{\Eps}{\Epsilon}
\newcommand{\gM}{\mathcal{M}}
\newcommand{\dD}{\mathcal{D}}
\newcommand{\gG}{\mathcal{G}}
\newcommand{\pa}{\partial}
\newcommand{\eps}{\epsilon}
\newcommand{\La}{\Lambda}
\newcommand{\De}{\Delta}
\newcommand{\nonu}{\nonumber}
\newcommand{\beq}{\begin{eqnarray}}
\newcommand{\eeq}{\end{eqnarray}}
\newcommand{\ka}{\kappa}
\newcommand{\an}{\ensuremath{\alpha_0}}
\newcommand{\bn}{\ensuremath{\beta_0}}
\newcommand{\dn}{\ensuremath{\delta_0}}
\newcommand{\al}{\alpha}
\newcommand{\bm}{\begin{multline}}
\newcommand{\fm}{\end{multline}}
\newcommand{\de}{\delta}



%\input title
\begin{titlepage} 
%\begin{flushright}
%version1\\
%\today
%\end{flushright}
\vspace*{0.5cm}
\begin{center}
{\Large\bf TBA equations for excited states in the Sine-Gordon model}
\end{center}
%
\vspace{2.5cm}
\begin{center}
{\large J\'anos Balog and \'Arp\'ad Heged\H us}
\end{center}
\bigskip
\begin{center}
Research Institute for Particle and Nuclear Physics,\\
Hungarian Academy of Sciences,\\
H-1525 Budapest 114, P.O.B. 49, Hungary\\ 
\end{center}
%
\vspace{3.2cm}
\begin{abstract}
We propose TBA integral equations for multiparticle soliton (fermion)
states in the Sine-Gordon (massive Thirring) model. 
This is based on T-system and Y-system
equations, which follow from the Bethe Ansatz solution in the light-cone 
lattice formulation of the model. Even and odd charge sectors are
treated on an equal footing, corresponding to periodic and twisted
boundary conditions, respectively. The analytic properties of the Y-system  
functions are conjectured on the basis of the large volume solution of
the system, which we find explicitly. A simple relation between the TBA
Y-functions and the counting function variable of the alternative
non-linear integral equation (Destri-deVega equation) description
of the model is given. At the special value $\beta^2=6\pi$ of the
Sine-Gordon coupling exact expressions for energy and momentum
eigenvalues of one-particle states are found.
\end{abstract}


\end{titlepage}




\section{Introduction}
%\input intro

A better theoretical understanding of finite size (FS) effects
is one of the most important problems in Quantum Field Theory (QFT).
The study of FS effects is a useful method of analysing the structure
of QFT models and it is an indispensable tool in the numerical
simulation of lattice field theories. 

L\"uscher \cite{Luscher}
derived a general formula for the FS corrections to particle masses
in the large volume limit. This formula, which is generally applicable for
any QFT model in any dimension, expresses the FS mass corrections in 
terms of an integral containing the forward scattering amplitude
analytically continued to unphysical (complex) energy. It is most
useful in $1+1$ dimensional integrable models \cite{KM1}, where the
scattering data are available explicitly.

The usefulness of the study of the mass gap in finite volume is 
demonstrated~\cite{LWW} by the introduction of the L\"uscher-Weisz-Wolff 
running coupling that enables the interpolation between the
large volume (non-perturbative) and the small volume (perturbative)
regions in both two-dimensional sigma models and QCD.

An important tool in the study of two-dimensional integrable field
theories is the Thermodynamic Bethe Ansatz (TBA).
This thermodynamical method was initiated by Yang and Yang \cite{YY} and  
allows the calculation of the free energy of the particle system.
The calculation was applied to the XXZ model by Takahashi and
Suzuki~\cite{TS} who derived the TBA integral equations for
the free energy starting from the Bethe Ansatz solution
of the system and using the \lq\lq string hypothesis'' describing
the distribution of Bethe roots. The form of the resulting TBA
equations strongly depends on whether the anisotropy parameter $p_0$ (see
Section 2) is an integer, rational or irrational number.

The TBA equations also determine FS effects
in relativistic (Euclidean) invariant two-dimensional field theory
models where the free energy is related to the ground state energy in finite
volume by a modular transformation interchanging spatial extension
and (inverse) temperature. Zamolodchikov \cite{Zamo90} initiated
the study of TBA equations for two-dimensional integrable models by
pointing out that TBA equations can also be derived starting from the 
(dressed) Bethe Ansatz equations formulated directly in terms of the
(infinite volume) scattering phase shifts of the particles.
In this approach the FS dependence of the ground state energy 
has been studied \cite{TBAlist} in many integrable models, 
mainly those formulated as perturbations of minimal conformal models.

TBA methods have been developed also in lattice statistical physics
\cite{KP}. Actually the basic functional relations, Y-system equations,
TBA integral equations and also other techniques playing important role
in the TBA analysis of continuum models have been originally
introduced here. TBA equations describing the FS energy of excited
states were proposed for some models. Also non-linear integral equations, 
similar to the Destri-deVega equations, appeared here first. 
An important model, the tricritical Ising model perturbed 
by its $\phi_{1,3}$ operator, has been studied in detail~\cite{PCA}.
The TBA equations describing all excited states are worked out
in this example, both for the massive and the massless perturbed models.

The TBA description of excited states is less systematic in continuum
models. The excited state TBA systems first studied \cite{Fendley,Martins}
are not describing particle states, they correspond to ground states
in charged sectors of the model. An interesting suggestion is to
obtain excited state TBA systems by analytically continuing~\cite{Martins}
those corresponding to the ground state energy. TBA equations
for scattering states were suggested for perturbed field theory models
by the analytic continuation method \cite{DT}. Excited state TBA
equations were also suggested for scattering multi-particle states
for the Sine-Gordon model at its $N=2$ supersymmetric point \cite{Fendley2}.

The BLZ program \cite{BLZ} has been initiated to derive functional
relations, Y-systems, and TBA integral equations, both for the ground state
and for excited states, directly in the continuum. The construction 
is complete for the conformal field theory limit of the models only
although some examples in massive perturbed models are also worked out.
The original construction is based on the $R$-matrix corresponding to 
the quantum deformed loop algebra $U_q(A^{(1)}_1)$ and can be applied to 
minimal models perturbed by their $\phi_{1,3}$ field but it works also 
in the $U_q(A^{(2)}_2)$ case~\cite{BE}, which is relevant to the
$\phi_{1,2}$ and $\phi_{2,1}$ perturbations. In this family the TBA
equations for the first two excited states in the non-unitary perturbed
${\cal M}_{3,5}$ model are worked out explicitly.

An alternative to the TBA equations is provided by the Destri-deVega (DdV)
non-linear integral equations. They were originally suggested~\cite{DdV0}
for the ground state of the Sine-Gordon (SG) or the closely
related massive Thirring (MT) model, but can be systematically
generalized to describe excited states as well; the (common)
even charge sector of the SG/MT model~\cite{DdV2}, the odd charge
sector in both SG and MT \cite{DdV1} and even the states of perturbed 
minimal models,
which can be represented as restrictions of the SG model \cite{DdVres}. 
The advantage of this non-linear integral equation approach is that
when available (SG/MT model and restrictions), it gives a systematic
description of all excited states. The method has been generalized to
higher rank imaginary coupling affine Toda models~\cite{ZJ}.

Meanwhile an alternative derivation of the TBA equations, 
which is independent of the validity of the string hypothesis 
has been found in the lattice approach for the generalized Hubbard
model \cite{JKS} and also for the XXZ model \cite{Kuniba}. Both models
are solvable on the lattice by the Bethe Ansatz method and the new
derivation of the TBA equations proceeds via the introduction of
T-system and Y-system functions satisfying functional relations
which can be transformed into TBA-type integral equations. Instead of 
the string hypothesis used in the original derivation \cite{TS}, here
the basic assumption is about the analytic structure of the Y-system
functions (distribution of their zeroes) and the validity of these
assumptions can be convincingly demonstrated numerically, at least
for small lattices, by starting from the numerical solution of the 
Bethe Ansatz equations. The Takahashi-Suzuki results for the ground
state energy are reproduced and new TBA equations for the excited states
are found in this way. Actually for the excited state problem the TBA
equations are supplemented by quantization conditions restricting
the allowed values of particle momenta, which is natural for particles
confined in a box.

In this paper we borrow the ideas of the lattice approach and
apply them to the continuum QFT case. Following the steps of the
lattice construction, we systematically build the T-system and Y-system
functions for all multi-soliton (multi-fermion) states of the SG (MT)
model. We find the appropriate continuum TBA equations and
quantization conditions and a link between the TBA and the DdV
equations.  

Our starting point is the light-cone lattice
regularization \cite{LC} of the SG/MT model and the Bethe Ansatz solution, 
which we briefly recall in Section 2. We use twisted boundary
conditions on the lattice and show that by changing the value of the parameter
characterizing the boundary conditions we can describe both the even
and the odd sectors of the model on an equal footing. (Previously only
the even sector was treated in the light-cone lattice formulation
and the properties of the states belonging to the odd sector were
conjectured by postulating the corresponding DdV equation in the
continuum limit \cite{DdV1}.) In Section 3 we introduce
the T-system and Y-system functions and establish the link between
the TBA Y-system and the \lq\lq counting function'' of the DdV
approach. This link is explicitly given by (\ref{FY}), which is one of our main
results. In Section 4 simple properties of the counting function
are recalled. In Section 5 we write down the light-cone lattice
TBA equations together with the quantization conditions. In section 6
we take the (finite volume) continuum limit of the equations. 
Multi-particle energy and momentum expressions are given in Section 7.
In Section 8 we discuss a special case ($p_0=4$) where we can find
exact expressions for the one-particle energies and momenta.
In section 9 we discuss the DdV equation and its analytic continuation
to the entire complex rapidity plane. In Section 10, using the link
between the DdV equation and the TBA Y-system, we find the explicit 
solution of the infinite volume limit of the TBA problem. 

To transform the Y-system equations into TBA integral equations (and
quantization conditions) we need to know the analytic properties
of the Y-system functions, in particular the distribution of their zeroes. 
From the explicit solution found in Section 10 we know this
distribution in the infinite volume limit.
Our main assumption in this paper is that the qualitative properties
of this distribution remain the same for finite volume.
Using this conjecture we can write down the complete set of equations
determining multi-particle momenta and energies in the TBA approach.
The ground state (no particles), one-particle, and two-particle
problems are discussed in detail in Sections 11, 12, and 13, 
respectively. We have verified the validity of our main assumption by 
numerically comparing the results of the TBA approach with those
obtained using the DdV equations. This is briefly described in Section~14 
and finally our conclusions are summarized in Section 15.
The technical details of the transformation of 
Y-system type functional relations into
TBA type integral equations is discussed in the Appendix.

For simplicity, in this paper we restrict attention to the
repulsive case $p_0>2$ of the SG coupling and consider integer $p_0$ only. 
Moreover, we consider multi-soliton states only and no states
containing both solitons and anti-solitons at the same time. We believe
that appropriate TBA systems can be found also for more general
couplings and more general states but the special cases we are
considering in this paper are sufficient to present the main ideas and
assumptions. Finally we note that although in the SG/MT case the
excited state TBA description is \lq\lq superfluous'' since we already
have the DdV equations to study FS physics, one can hope that the
simple pattern of the excited state TBA systems we find here means 
that similar systems can be found also in other systems, where no DdV
type alternative is available.




%\newpage

\section{Light-cone approach to twisted SG model}
%\input LC
Our starting point is the Bethe Ansatz solution of the integrable lattice
regularization of the Sine-Gordon field theory \cite{LC}. Here we briefly
summarize the results of this approach \cite{DdV0,DdV2,DdV1} 
to the Sine-Gordon (and
massive Thirring) model. The fields of the regularized theory
are defined at sites (\lq\lq events") of a light-cone lattice and the
dynamics of the system is defined by 
translations in the left and right lightcone directions. These are given by
transfer matrices of the six-vertex model with anisotropy $\gamma$ and
alternating inhomogeneities. This approach is particularly useful for
calculating the finite size dependence of physical quantities.
We take $N$ points ($N$ even) in the spatial direction and use twisted periodic
boundary conditions. The lattice spacing is related to $L$, the
(dimensionful) size of the system:
\be
a=\frac{L}{N}.
\end{equation}

