
\begin{document}

\title{$\xZ_n$-Graded Topological Generalizations of Supersymmetry and Orthofermion Algebra}
\author{A Mostafazadeh}
\address{Department of Mathematics, Ko\c{c} University, Rumelifeneri Yolu, 34450 Sariyer, Istanbul, Turkey}
\begin{abstract}
We review various generalizations of supersymmetry and discuss their relationship. In particular, we show how supersymmetry, parasupersymmetry, fractional supersymmetry, orthosupersymmetry, and 
the $\xZ_n$-graded topological symmetries are related.
\end{abstract}
% Leave the next line commented out!
% \maketitle

\section{Introduction}

The advent of supersymmetric quantum mechanics (SQM) in the 1980s \cite{witten-82} and its remarkable applications \cite{susy} have since motivated many researchers to seek for generalizations of SQM. Most of these generalizations are algebraic in nature in the sense that they are defined in terms of an operator algebra involving a central element called the Hamiltonian $H$ and a number of noncentral operators ${\cal Q}_a$ and ${\cal Q}_a^\dagger$ called the symmetry generators such that this operator algebra generalizes the algebra of SQM, namely \cite{nicolai,witten-82}
	\xbe
	{\cal Q}_a^2={\cal Q}_a^{\dagger 2}=0,~~~~~[{\cal Q}_a,H]_-=0,~~~~~~
	[{\cal Q}_a,{\cal Q}_b^\dagger]_+=2\delta_{ab}H,
	\label{sqm}
	\end{equation}

