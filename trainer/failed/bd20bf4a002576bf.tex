\begin{equation}\label}
\newcommand{\ee}{\end{equation}}
