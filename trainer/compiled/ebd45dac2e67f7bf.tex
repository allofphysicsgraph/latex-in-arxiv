\begin{equation}\label{ffcf}
{\textstyle [\,\left<f^i\right>,\,\left<f^j\right>\, ]}=c_{ijk}
{\textstyle \left< \right.} f^k {\textstyle \left. \right>}\,.
\end{equation} 
