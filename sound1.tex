\documentclass[aps,prl,groupedaddress,fleqn,twocolumn,10pt]{revtex4-1}
\usepackage{graphicx}
\usepackage{amsmath}
\usepackage{gensymb}
\usepackage{float}
\usepackage{physics}
\usepackage{xcolor}
%\newcommand\omicron{o}

\begin{document}

\raggedbottom

\title{Speed of sound from fundamental physical constants}
\author{K. Trachenko}
\affiliation{School of Physics and Astronomy, Queen Mary University of London, Mile End Road, London, E1 4NS, UK}
\author{B. Monserrat}
 \affiliation{Department of Materials Science and Metallurgy, University of Cambridge, 27 Charles Babbage Road, Cambridge CB3 0FS, United Kingdom}%
 \affiliation{Cavendish Laboratory, University of Cambridge, J. J. Thomson Avenue, Cambridge CB3 0HE, United Kingdom}
\author{C. J. Pickard}
 \affiliation{Department of Materials Science and Metallurgy, University of Cambridge, 27 Charles Babbage Road, Cambridge CB3 0FS, United Kingdom}%
\affiliation{Advanced Institute for Materials Research, Tohoku University, Sendai, Japan}
\author{V. V. Brazhkin}
\affiliation{Institute for High Pressure Physics, RAS, 108840, Troitsk, Moscow, Russia}

\begin{abstract}
Two dimensionless fundamental physical constants, the fine structure constant $\alpha$ and the proton-to-electron mass ratio $\frac{m_p}{m_e}$ are attributed a particular importance from the point of view of nuclear synthesis, formation of heavy elements, planets, and life-supporting structures. Here, we show that a combination of these two constants results in a new dimensionless constant which provides the upper bound for the speed of sound in condensed phases, $v_u$. We find that $\frac{v_u}{c}=\alpha\left(\frac{m_e}{2m_p}\right)^{\frac{1}{2}}$, where $c$ is the speed of light in vacuum. We support this result by a large set of experimental data and first principles computations for atomic hydrogen. Our result expands current understanding of how fundamental constants can impose new bounds on important physical properties.
\end{abstract}

\maketitle

\section{Introduction}

Several notable properties of condensed matter phases are defined by fundamental physical constants. The Bohr radius gives a characteristic scale of interatomic distance on the order of the Angstrom, in terms of electron mass $m_e$, charge $e$, and Planck constant $\hbar$. These same fundamental constants enter the Rydberg energy, setting the scale of a characteristic bonding energy in condensed phases and chemical compounds \cite{ashcroft}.

Among the fundamental constants, those that are {\it dimensionless} and do not depend on the choice of units, play a special role in physics \cite{barrow}. Two important dimensionless constants are the fine structure constant $\alpha$ and the proton-to-electron mass ratio, $\frac{m_p}{m_e}$. The finely-tuned values of $\alpha$ and $\frac{m_p}{m_e}$, and the balance between them, governs nuclear reactions such as proton decay and nuclear synthesis in stars, leading to the creation of the essential biochemical elements, including carbon. This balance provides a narrow ``habitable zone'' in the ($\alpha$,$\frac{m_p}{m_e}$) space where stars and planets can form and life-supporting molecular structures can emerge \cite{barrow}.

We show that a simple combination of $\alpha$ and $\frac{m_p}{m_e}$ results in another dimensionless quantity which has an unexpected and specific implication for a key property of condensed phases, the speed at which waves travel in solids and liquids, or the speed of sound, $v$. We find that this combination provides an upper bound for $v$, $v_u$, as
\begin{equation}
\frac{v_u}{c}=\alpha\left(\frac{m_e}{2m_p}\right)^{\frac{1}{2}},
\label{v0}
\end{equation}
\noindent where $c$ is the speed of light in vacuum.

We support this result with a large set of experimental data for different systems, and the first principles modelling of atomic hydrogen.

Identifying and understanding bounds on physical properties is important from the point of view of fundamental physics, predictions for theory and experiment, as well as searching for and rationalizing universal behavior (see, e.g., \cite{kss,zaanen3,hartnoll,zaanen2,spin,behnia,zaanen1,behnia1,hartnoll1}). Properties for which bounds were recently discussed include viscosity and diffusivity. The proposed {\it lower} bounds for these two properties feature in a range of areas including, for example, strongly-interacting field theories, quark-gluon plasmas, holographic duality, electron diffusion, transport properties in metals and superconductors, and spin transport in Fermi gases \cite{kss,zaanen3,hartnoll,zaanen2,spin,behnia,zaanen1,behnia1,hartnoll1}. Recently, two of us found a lower bound for the kinematic viscosity of liquids set by fundamental physical constants \cite{sciadv}. Here, we propose a new, {\it upper}, bound for the speed of sound in condensed matter phases in terms of fundamental constants.

Apart from setting the speed of elastic interactions in solids, $v$ is related to elasticity, hardness and affects important low-temperature thermodynamic properties such as energy, entropy and heat capacity \cite{landau}. As discussed below, the upper bound of $v$ sets the smallest possible entropy and heat capacity at a given temperature.

In solids, $v$ depends on elastic properties and density. These strongly depend on the bonding type and structure which are inter-dependent \cite{phillips}. As a result, it was not thought that $v$ can be predicted analytically without simulations, contrary to other properties such as energy or heat capacity which are universal in the classical harmonic approximation \cite{landau}. In view of this, representing the upper bound of $v$ in terms of fundamental constants is notable.

\section{Results and discussion}

There are two approaches in which $v$ can be evaluated. The two approaches start with system elasticity and vibrational properties, respectively.

We begin with system elasticity. The longitudinal speed of sound is $v=\left({\frac{M}{\rho}}\right)^{\frac{1}{2}}$, where $M=K+\frac{4}{3}G$, $K$ is the bulk modulus, $G$ is the shear modulus, and $\rho$ is the density. It has been ascertained that elastic constants are governed by the density of electromagnetic energy in condensed matter phases. In particular, a clear relation was established between the bulk modulus $K$ and the bonding energy $E$: $K=f\frac{E}{a^3}$, where $a$ is the interatomic separation and $f$ is the proportionality coefficient \cite{diamond,diamond1}. This relation can be derived up to a constant given by the second derivative of the function representing the dependence of energy on volume. For a majority of strongly-bonded solids, $f$ varies in the range 1-4 \cite{diamond,diamond1}. The same data implies the proportionality coefficient between $M$ and $\frac{E}{a^3}$ in the range of about 1-6. Combining $v=\left({\frac{M}{\rho}}\right)^{\frac{1}{2}}$ and $M=f\frac{E}{a^3}$ gives $v=f^{\frac{1}{2}}\left(\frac{E}{m}\right)^{\frac{1}{2}}$, where $m$ is the mass of the atom or molecule, and we used $m=\rho a^3$. The factor $f^{\frac{1}{2}}$ is about 1-2 and can be dropped in an approximate evaluation of $v$. Then,

\begin{equation}
v=\left(\frac{E}{m}\right)^{\frac{1}{2}}.
\label{v01}
\end{equation}

We now recall that the bonding energy in condensed phases is given by the Rydberg energy on the order of several eV \cite{ashcroft} as
\begin{equation}
E_{\rm R}=\frac{m_ee^4}{32\pi^2\epsilon_0^2\hbar^2},
\label{rydberg}
\end{equation}
\noindent where $e$ and $m_e$ are electron charge and mass.

$E_{\rm R}$ is used for order-of-magnitude estimations of the bonding energy $E$ \cite{ashcroft}. Using $E=E_{\rm R}$ from (\ref{rydberg}) in (\ref{v01}) gives
\begin{equation}
v=\alpha\left(\frac{m_e}{2m}\right)^{\frac{1}{2}}c,
\label{v00}
\end{equation}
\noindent where $\alpha=\frac{1}{4\pi\epsilon_0}\frac{e^2}{\hbar c}$ is the fine structure constant.

A result similar to (\ref{v00}) can be obtained in the second approach that starts with the consideration of the vibrational properties of the system. The longitudinal speed of sound, $v$, can be evaluated as the phase velocity from the longitudinal dispersion curve $\omega=\omega$($k$) in the Debye approximation: $v=\frac{\omega_{\rm D}}{k_{\rm D}}$, where $\omega_{\rm D}$ and $k_{\rm D}$ are Debye frequency and wavevector, respectively. Using $k_{\rm D}=\frac{\pi}{a}$, where $a$ is the interatomic (inter-molecule) separation, gives
\begin{equation}
v=\frac{1}{\pi}\omega_{\rm D}a.
\label{v001}
\end{equation}

We recall that the characteristic scale of interatomic separation is given by the Bohr radius $a_{\rm B}$ on the order of the Angstrom as
\begin{equation}
a_{\rm B}=\frac{4\pi\epsilon_0\hbar^2}{m_e e^2}.
\label{bohr}
\end{equation}

We now use the known ratio between the phonon energy, $\hbar\omega_{\rm D}$, and $E$. The phonon energy $\hbar\omega_{\rm D}$ can be approximated as $\hbar\left(\frac{E}{ma^2}\right)^{\frac{1}{2}}$, where $m$ is the mass of the atom. Taking the ratio $\frac{\hbar\omega_{\rm D}}{E}$ and using $a=a_{\rm B}$ from (\ref{bohr}) and $E=E_{\rm R}$ from (\ref{rydberg}) gives $\frac{\hbar\omega_{\rm D}}{E}$, up to a constant factor close to unity, as
\begin{equation}
\frac{\hbar\omega_{\rm D}}{E}=\left(\frac{m_e}{m}\right)^{\frac{1}{2}}.
\label{ratio}
\end{equation}

Using (\ref{ratio}) in (\ref{v001}) gives
\begin{equation}
v=\frac{Ea}{\pi\hbar}\left(\frac{m_e}{m}\right)^{\frac{1}{2}}.
\label{v1}
\end{equation}

$v$ in (\ref{v00}), up to a constant factor, can now be obtained by using $a=a_{\rm B}$ from (\ref{bohr}) and $E=E_{\rm R}$ from (\ref{rydberg}) in (\ref{v1}). Alternatively, the same result can be found by (a) recalling that the bonding energy, or the characteristic energy of electromagnetic interaction, is $E=\frac{\hbar^2}{2m_ea^2}$ and (b) using this $E$ and $a=a_{\rm B}$ (\ref{bohr}) in (\ref{v1}).

As compared to the first approach, the second approach to evaluating $v$ involves additional approximations, including evaluating $v$ from the dispersion relation in the Debye model, using $a=a_{\rm B}$ in (\ref{bohr}), and the ratio between the phonon and bonding energies (\ref{ratio}). We therefore focus on the result from the first approach, Eq. (\ref{v00}).

We now discuss Eq. (\ref{v00}) and its implications. $m_e$ characterises electrons, which are responsible for the interactions between atoms. The electronic contribution is further reflected in the factor $\alpha c$ ($\alpha c\propto\frac{e^2}{\hbar}$), which is the electron velocity in the Bohr model.

We note that $\alpha c$ and $v$ do not depend on $c$. The reason for writing $v$ in terms of $\alpha c$ in Eq. (\ref{v00}) and the ratio $\frac{v_u}{c}$ in terms of $\alpha$ in Eq. (\ref{v0}) is two-fold. First, it is convenient and informative to represent the bound in terms of the ratio $\frac{v_u}{c}$, similarly to the ratio of the Fermi velocity and the speed of light $\frac{v_{\rm F}}{c}$ commonly used. Second, it is $\alpha$ (together with $\frac{m_p}{m_e}$) that is given fundamental importance and is finely tuned to result in proton stability and to enable the synthesis of heavy elements \cite{barrow} and, therefore, the existence of solids and liquids where sound can propagate to begin with.

$m$ in (\ref{v00}) characterises atoms involved in sound propagation. Its scale is set by the proton mass $m_p$: $m=Am_p$, where $A$ is the atomic mass. Recall that $a_{\rm B}$ in (\ref{bohr}) and $E_{\rm R}$ in (\ref{rydberg}) are characteristic values derived for the H atom. We similarly set $A=1$ and $m=m_p$ in (\ref{v00}) to arrive at the upper bound of $v$ in (\ref{v00}), $v_u$, as

\begin{equation}
v_u=\alpha\left(\frac{m_e}{2m_p}\right)^{\frac{1}{2}}c~\approx~36,100~\frac{\rm m}{\rm s},
\label{v3}
\end{equation}

\noindent and observe that $v_u$ depends on fundamental physical constants only, including the dimensionless fine structure constant $\alpha$ and the proton-to-electron mass ratio.

Equation (\ref{v3}) is the extension of (\ref{v00}) to atomic hydrogen. We will calculate $v$ in atomic H later in the paper.

Combining Eqs. (\ref{v00}), (\ref{v3}), and $m=Am_p$ gives
\begin{equation}
v=\frac{v_u}{A^\frac{1}{2}}.
\label{a}
\end{equation}


