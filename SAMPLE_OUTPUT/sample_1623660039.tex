\documentclass[]{article}

%opening
\title{}
\author{}

\begin{document}

\maketitle


\begin{abstract}
	We calculate quantum corrections to the mass of the vortex in $N=2$
	supersymmetric abelian Higgs model in $2+1$ dimensions. We put the system
	in a box and apply the zeta function regularization. The boundary conditions 
	inevitably violate a part of the supersymmetries. Remaining supersymmetry
	is however enough to ensure isospectrality of relevant operators in bosonic
	and fermionic sectors. A non-zero correction to the mass of the vortex
	comes from finite renormalization of couplings.\\
	PACS: 11.27.+d; 12.60.Jv 
\end{abstract}

\begin{equation}
	\rho_B(\omega ) - \rho_F(\omega )\propto \delta (\omega ) \,,
	\label{rrd}
\end{equation}

\begin{equation}
	\Delta E\propto \int d\omega\, \omega (\rho_B(\omega ) - \rho_F(\omega ))
	\,.\label{DEspectdens}
\end{equation}

\begin{equation}
	\Delta E=\Delta E_B - \Delta E_F,\qquad \Delta E_{B,F} = \frac 12
	\sum_{\omega_{B,F}} \omega_{B,F} \,,\label{DEBF}
\end{equation}

\begin{equation}
	\Delta E^{\rm reg}_{B,F}=\frac 12 \sum_{\omega_{B,F}\ne 0} \omega_{B,F}^{1-2s}
	\,,
	\label{DEreg}
\end{equation}

\begin{eqnarray} 
	&&{\cal L}={\cal L}_B+{\cal L}_F \,,\label{LLL}\\
	&&{\cal L}_B=-\frac 14 F_{\mu\nu}F^{\mu\nu} -|D_\mu \phi |^2
	-\frac 12 (\partial_\mu w )^2 
	-\frac{e^2}2 \left( |\phi|^2 -v^2 \right)^2
	-e^2 w^2 |\phi|^2 \,,\label{LB} \\
	&&{\cal L}_F=i\bar\psi \gamma^\mu D_\mu \psi 
	+i\bar\chi \gamma^\mu \partial_\mu \chi
	-i \sqrt{2} e (\bar\psi \chi \phi -\bar\chi\psi \phi^*)
	+ew\bar\psi\psi \,,\label{LF}
\end{eqnarray}

\begin{eqnarray}
	&&\delta A_\mu =i\left( \bar\eta \gamma_\mu \chi -
	\bar\chi\gamma_\mu \eta \right) \,,\nonumber \\
	&&\delta \phi = \sqrt{2} \bar\eta \psi\,,\qquad
	\delta w =i (\bar\chi \eta -\bar \eta \chi ) \,,\nonumber \\
	&&\delta \chi = \gamma^\mu \eta \left( \partial_\mu w +
	\frac i2 \epsilon_{\mu\nu\lambda}F^{\nu\lambda} \right)
	+i\eta (e |\phi |^2 -ev^2 ) \,,\nonumber \\
	&&\delta \psi =-\sqrt{2} \left( i\gamma^\mu \eta D_\mu \phi
	-\eta e w \phi \right) \label{susy}
\end{eqnarray}

\begin{equation}
	\gamma^0=\left( \begin{array}{cc}
		1&0\\
		0&-1 \end{array}\right)\,,\qquad
	\gamma^1=\left( \begin{array}{cc}
		0&1\\
		-1&0 \end{array}\right)\,,\qquad
	\gamma^2=\left( \begin{array}{cc}
		0&i\\
		i&0 \end{array}\right) \label{gamma}
\end{equation}

\begin{equation}
	\gamma^\mu \gamma^\nu =-g^{\mu\nu} -i\epsilon^{\mu\nu\rho}\gamma_\rho
	\,.\label{gagag}
\end{equation}

\begin{equation}
	\eta=\left( \begin{array}{c} \eta_+ \\ \eta_- \end{array} \right),
	\label{spinorcom}
\end{equation}

\begin{eqnarray}
	&&(D_1+iD_2)\phi =0\,,\label{bog1}\\
	&&F_{12}+e(|\phi |^2 -v^2 )=0 \,.\label{bog2}
\end{eqnarray}

\begin{equation}
	\phi =f(r) e^{in\theta},\qquad eA_j=\epsilon_{jk} \frac {x^k}{r^2}
	(a(r)-n) \label{vortex}
\end{equation}

\begin{eqnarray}
	&&\frac 1r \frac{d}{dr} a(r)=e^2\left( f^2(r)-v^2 \right)\,,\nonumber\\
	&&r \frac{d}{dr} \ln f(r) = a(r) \,.\label{af-eqs}
\end{eqnarray}

\begin{eqnarray}
	&& f(0)=0\,,\qquad f(\infty )=v \,,\label{cononf}\\
	&& a(0)=n\,,\qquad a(\infty )=0 \,.\label{conona}
\end{eqnarray}

\begin{equation}
	E^{\rm cl}=2\pi n v^2 \,.\label{Eclass}
\end{equation}

\begin{equation}
	{\cal L}_{\rm gf}=-\frac 12 \left[ \partial_\mu \alpha^\mu 
	-ie (\varphi^* \phi - \varphi \phi^*) \right]^2\,,\label{gfterm}
\end{equation}

\begin{equation}
	{\cal L}_{\rm ghost}= \sigma^* ( \partial_\mu \partial^\mu 
	-2e^2 \phi \phi^* )\sigma \,.\label{ghostac}
\end{equation}





\begin{abstract}
	We discuss a new class of brane models (extending both $p$-brane and $Dp$-brane 
	cases) where the brane tension appears as an {\em additional dynamical degree of
		freedom}~ instead of being put in by hand as an {\em ad hoc}~ dimensionfull scale.
	Consistency of dynamics naturally involves the appearence of additional higher-rank
	antisymmetric tensor gauge fields on the world-volume 
	% (beyond the standard $Dp$-brane world-volume $U(1)$ vector gauge field) 
	which can couple to charged
	lower-dimensional branes living on the original $Dp$-brane world-volume.
	The dynamical tension has the physical meaning of electric-type field strength of
	the additional higher-rank world-volume gauge fields. It obeys Maxwell (or 
	Yang-Mills) equations of motion (in the string case $p=1$) or their higher-rank
	gauge theory analogues (in the $Dp$-brane case). This in particular triggers
	a simple classical mechanism of (``color'') charge confinement.    
\end{abstract}



\begin{equation}
	\omega_F^2 \left( \begin{array}{c} U \\ V \end{array} \right)
	= \left( \begin{array}{cc} D_F D_F^\dag ,& 0 \\
		0,& D_F^\dag D_F \end{array} \right)
	\left( \begin{array}{c} U \\ V \end{array} \right) \,,
	\label{sqDir}
\end{equation}

\begin{equation}
	U=\left( \begin{array}{c} \psi_+ \\ \chi_- \end{array} \right)
	\,,\qquad 
	V=\left( \begin{array}{c} \psi_- \\ \chi_+ \end{array} \right)
	\label{defUV}
\end{equation}

\begin{equation}
	D_F = \left( \begin{array}{cc} D_+,& -\sqrt{2} e\phi \\
		-\sqrt{2}e\phi^*,&\partial_-
	\end{array}\right)\,,\qquad
	-D_F^\dag = \left( \begin{array}{cc} D_-,& \sqrt{2} e\phi \\
		\sqrt{2}e\phi^*,&\partial_+
	\end{array}\right)\,.\label{DFDF}
\end{equation}

\begin{equation}
	D_\pm := D_1\pm i D_2\,,\qquad 
	\partial_\pm :=\partial_1\pm i \partial_2 \,.\label{Ddpm}
\end{equation}

\begin{equation}
	\Delta_w=-\partial_j\partial_j +2e^2|\phi |^2 \,.\label{opwa0}
\end{equation}

\begin{equation}
	\omega_B^2 
	\left( \begin{array}{c} \varphi \\ {i\alpha_+}{\sqrt{2}} \end{array} \right)
	:= -\partial_0^2
	\left( \begin{array}{c} \varphi \\ {i\alpha_+}{\sqrt{2}} \end{array} \right)
	= D_F^\dag D 
	\left( \begin{array}{c} \varphi \\ {i\alpha_+}{\sqrt{2}} \end{array} \right)\,,
	\label{omB}
\end{equation}

\begin{eqnarray}
	&&D_F^\dag D_F = -\left( \begin{array}{cc} D_j^2 - e^2 (3|\phi |^2
		-v^2 ), &  -\sqrt{2} e (D_-\phi ) \\
		-\sqrt{2} e (D_+\phi^*), & \partial_j^2 -2e^2 |\phi |^2
	\end{array} \right) \label{DdagD}\,,\\
	&&D_F D_F^\dag = -\left( \begin{array}{cc}D_j^2 - e^2 (|\phi |^2
		+v^2 ), & 0 \\ 0, & \partial_j^2 -2e^2 |\phi |^2
	\end{array} \right)\,.\label{DDdag}
\end{eqnarray}

\begin{equation}
	u_\pm = e^{\pm i\theta} \left( u_r \pm \frac ir u_\theta \right)\,,
	\label{pmrt}
\end{equation}

\begin{equation}
	{\cal B}^{[\alpha]} \alpha_\mu |_{\partial M}=0,\qquad
	{\cal B}^{[\sigma]} \sigma |_{\partial M}=0 \,,\label{bops}
\end{equation}

\begin{equation}
	{\cal B}^{[\alpha]} \partial_\mu \sigma |_{\partial M}=0 \,.
	\label{gauinv}
\end{equation}

\begin{equation}
	\alpha_0|_{\partial M}=0,\quad
	\alpha_\theta |_{\partial M}=0,\quad
	\left( \partial_r +\frac 1r \right) \alpha_r |_{\partial M}=0,\quad
	\sigma |_{\partial M}=0.\label{asbc}
\end{equation}

\begin{equation}
	\left( \partial_r +\frac 1r \right) \partial_r \sigma =
	\left[ -\Delta_w \sigma \right] +
	\left[ - \frac 1{r^2}\partial_\theta^2 + 2\phi\phi^* \right]\sigma
	\,,\label{ginar}
\end{equation}

\begin{equation}
	\left[ -\Delta_w \sigma \right]|_{\partial M} =
	-\sum_k \omega_k^2 \sigma_k |_{\partial M} =0.\label{1stterm}
\end{equation}

\begin{equation}
	w|_{\partial M}=0 \,.\label{bcw}
\end{equation}

\begin{equation}
	\alpha_r =\Re \left( e^{-i\theta} \alpha_+ \right)\,,\qquad
	\alpha_\theta = r \Im \left( e^{-i\theta }\alpha_+ \right)\,.
	\label{rthetaplus}
\end{equation}

\begin{equation}
	\Re \left( e^{-i\theta} \chi_+ \right) |_{\partial M}=0,
	\qquad \left( \partial_r +\frac 1r \right) \Im
	\left( e^{-i\theta} \chi_+ \right) |_{\partial M}=0.
	\label{bcchip}
\end{equation}

\begin{eqnarray}
	&&V(\omega )=\omega^{-2} D_F^\dag U(\omega )\,,\label{VomU}\\
	&&U(\omega )=\omega^{-2} D_F V(\omega )\label{UomV}
\end{eqnarray}

\begin{equation}
	U_1(\omega ) =\omega^{-2} \left( D_+ V_1(\omega ) 
	-\sqrt{2} e \phi V_2(\omega ) \right).
	\label{U1VV}
\end{equation}

\begin{equation}
	\Re \left( e^{-i\theta}\phi^* U_1 \right)|_{\partial M}=
	\Re \left( e^{-i\theta}\phi^* \psi_+ \right)|_{\partial M}=
	0\,.\label{bcU1D}
\end{equation}

\begin{equation}
	U_2(\omega )=\omega^2 \left( \partial_- V_2 
	-\sqrt{2} e\phi^* V_1 \right)\,.
	\label{U2VV}
\end{equation}

\begin{equation}
	\Im \left( U_2 \right)|_{\partial M}=0,\qquad
	\Im \left( \phi^* V_1 \right)|_{\partial M}=0 \,.
	\label{bcU2V1}
\end{equation}










\begin{equation}
	\Im \left( \phi^* \varphi \right)|_{\partial M}=0 \,,\qquad
	\left( \partial_r -2(\partial_r \ln \phi^* ) \right)\Re 
	\left( \phi^* \varphi \right)|_{\partial M}=0 \,.\label{bcphi}
\end{equation}

\begin{equation}
	\partial_r \Re (\chi_-)|_{\partial M}=0\,,\qquad
	\left( \partial_r +\frac 1r \right) 
	\Im \left( e^{-i\theta} \phi^* \psi_+ \right)|_{\partial M}=0\,.
	\label{bcUUN}
\end{equation}

\begin{equation}
	\delta \alpha_+ =2i\eta_-^* \chi_+ \,.\label{delap}
\end{equation}

\begin{equation}
	\Re (\eta_-) =0 \label{Imeta}
\end{equation}

\begin{equation}
	\Delta E^{\rm ren} = \Delta E(V+B)^{\rm ren} - \Delta E (B)^{\rm
		ren} + \Delta E^{\rm f.r.} \,,\label{DE3DE}
\end{equation}

\begin{equation}
	\Delta E_B^{\rm reg} = \frac 12 \sum_{\omega_B\ne 0}
	\omega_B^{1-2s} =\frac 12 \sum_{\omega_F\ne 0} \omega_F^{1-2s} =
	\Delta E_F^{\rm reg} \,.\label{DEBDEF}
\end{equation}

\begin{equation}
	\Delta E(V+B)^{\rm ren} =0 \,.\label{1term}
\end{equation}

\begin{equation}
	\tilde \Delta =-\partial_j^2 +2e^2v^2.\label{asop}
\end{equation}

\begin{equation}
	\partial_r \phi =-\frac ir D_\theta \phi \,.
\end{equation}

\begin{equation}
	(D_-\phi )= e^{-i\theta} 2 \partial_r \phi \ {\to}\ 0
\end{equation}

\begin{equation}
	\tilde \varphi =e^{i\beta(r) \theta} \varphi,\qquad
	\tilde \psi =e^{i\beta(r) \theta} \psi,\label{tilfiel}
\end{equation}

\begin{eqnarray}
	&&\left( \partial_r +\frac 1r \right) \Im 
	\left( e^{-i\theta} (i\alpha_+,\chi_+,\tilde\psi_+) \right) 
	\vert_{\partial M}=
	0\,,\nonumber\\
	&&\Re \left( e^{-i\theta} (i\alpha_+,\chi_+,\tilde\psi_+) \right) 
	\vert_{\partial M}=0\,.\label{effbc1}
\end{eqnarray}

\begin{eqnarray}
	&&\partial_r \Re \left( (\tilde\varphi,\tilde\psi_-,\chi_-) \right)
	\vert_{\partial M}=
	0\,,\nonumber\\
	&&\Im \left( (\tilde\varphi,\tilde\psi_-,\chi_-) \right)
	\vert_{\partial M}=0 \,.\label{effbc2}
\end{eqnarray}

\begin{equation}
	\Delta E (B)^{\rm ren}=0 \,.\label{DEBren}
\end{equation}

\begin{equation}
	W_m(s)=\frac 12 \sum \omega (m)^{1-2s} = \frac 12 \zeta_m \left(
	s-\frac 12\right) \,, \label{mW}
\end{equation}

\begin{equation}
	\zeta_m \left( s-\frac 12\right)= \Gamma \left( s-\frac
	12\right)^{-1} \int d^2x \int\limits_0^\infty dt\, t^{s-\frac 12
		-1} K(t,x) \,.\label{zmKt}
\end{equation}

\begin{equation}
	K(t,x)=\langle x | e^{-t\Delta_m } | x\rangle = (4\pi t)^{-1}
	e^{-m^2t} \,.\label{Ktx}
\end{equation}

\begin{equation}
	{\cal W}_m =-\frac {m^3}{12\pi} \,.\label{cWm}
\end{equation}

\begin{equation}
	{\cal W}^{\rm 1-loop} = -\frac{e^3}{6\pi} \left[ \left( 3|\phi |^2
	-v^2 \right)^{3/2} -\left( 2|\phi |^2 \right)^{3/2} \right]
	\,.\label{W1loop}
\end{equation}

\begin{equation}
	{\cal W}^{\rm tot}={\cal W}^{\rm cl}(e+\hbar \delta e,v +\hbar\delta v)
	+\hbar {\cal W}^{1-loop} \,,\label{Wtot} 
\end{equation}


\begin{equation}
	{\cal W}^{\rm cl}(e,v)=\frac{e^2}2 \left( |\phi |^2 -v^2 \right) 
	\label{Wclass}
\end{equation}

\begin{equation}
	\delta v = -\frac{e}{4\sqrt{2}\pi}\,.\label{deltav}
\end{equation}

\begin{equation}
	\Delta E^{\rm f.r.}=\hbar (\delta v)\frac{dE^{\rm cl}}{dv}=
	-\frac{evn\hbar}{\sqrt{2}} \,.\label{DEfr}
\end{equation}

\begin{equation}
	\Delta E^{\rm ren}=-\frac{evn\hbar}{\sqrt{2}} \,.\label{finres}
\end{equation}

\begin{equation}
	\delta \chi_-=-\eta_- \left( \partial_0 w +i\epsilon_{ojk}
	\partial^j \alpha^k -2i e \Re ( \phi^* \varphi ) \right) \,.
	\label{delchim}
\end{equation}

\begin{equation}
	\Im (\delta \chi_-) |_{\partial M} \sim -\partial_0 w |_{\partial
		M}=0 \,,\label{apa1}
\end{equation}

\begin{equation}
	0=\partial_r \Re (\delta \chi_-)|_{\partial M} \sim
	\partial_r \left( -\epsilon_{ojk}
	\partial^j \alpha^k + 2e \Re ( \phi^* \varphi ) \right)|_{\partial M}
	\,.\label{apa2}
\end{equation}

	\begin{equation}
		X_i = \frac{x_i}{1-w} 
		{\rm ~~~and~~~}
		{\tilde X}_i = \frac{x_i}{1+w}.
	\end{equation}
	
	\begin{equation}
		g = {\rm diag} \left(1, \frac{-4}{1+R^2},\frac{-4}{1+R^2},\frac{-4}{1+R^2}
		\right),\ 
		{\rm where}\
		R^2 = X_1^2+X_2^2+X_3^2.
	\end{equation}
	
	\begin{equation}
		{\tilde X}_i = \frac{1}{R^2} X_i.
	\end{equation}
	
	\begin{equation} 
		\partial_{{\tilde X}_i} = \left(R^2 \delta_{ij} - 2 X_i X_j \right)
		\partial_{X_j}.
	\end{equation}
	
	\begin{eqnarray}
		\begin{array}{c}
			{e_0}^0 = 1, \\ \\
			{e_i}^i = - \frac{1+R^2}{2},
		\end{array}
		{\rm ~~~and~~~}
		\begin{array}{c}
			{{\tilde e}_0}^{~0} = 1, \\ \\
			{{\tilde e}_i}^{~i} =  \frac{1+{\tilde R}^2}{2},
		\end{array}
	\end{eqnarray}
	
	\begin{equation}
		{e_\alpha}^\mu {e_\beta}^\nu g_{\mu \nu} = \eta_{\alpha \beta},
	\end{equation}
	
	\begin{equation}
		\label{erotation}
		{\hat {\tilde e}}_i = \frac{1}{R^2} \left( 2 X_i X_j - \delta_{i j}\right) {\hat
			e}_j.
	\end{equation}
	
	\begin{equation}
		{\rm d} {\hat \theta}^\alpha + {\omega^\alpha}_\beta \wedge {\hat
			\theta}^\beta = 0,
	\end{equation}
	
	\begin{equation}
		\omega^{\alpha \beta} = \frac{2}{1+R^2} \left( X^\alpha {\rm d}
		X^\beta - X^\beta {\rm d} X^\alpha \right),
	\end{equation}
	

	
	\begin{equation}
		{\cal L}_{{\rm fermion}} = {\overline \psi} \left( i \gamma^\alpha 
		{e_\alpha}^\kappa \left(
		\partial_\kappa + \Omega_\kappa \right) 
		\right) \psi.
	\end{equation}
	
	\begin{equation}
		\label{L1}
		{\cal L}_{{\rm fermion}} = {\overline \psi} \left(X_i,t \right) 
		\left( i \gamma^0 \partial_t - 
		i \gamma^i \left(\frac{1+R^2}{2} \partial_{X_i} - X_i \right)
		\right)
		\psi \left(X_i,t \right),
	\end{equation}
	
	\begin{equation}
		\label{L2}
		{\cal L}_{{\rm fermion}} = {\overline {\tilde \psi}} \big({\tilde X}_i,t \big) 
		\left( i \gamma^0 \partial_t + 
		i \gamma^i \left(\frac{1+{\tilde R}^2}{2} \partial_{{\tilde X}_i} - 
		{\tilde X}_i \right)
		\right)
		{\tilde \psi} \big({\tilde X}_i,t \big),
	\end{equation}
	
	\begin{equation}
		{\tilde \psi}\big({\tilde X}_i\big) = \rho\, \psi \left( X_i \right).
	\end{equation}
	
	\begin{equation}
		T(\alpha)_{ij} = \cos \alpha~ \delta_{i j} + \sin
		\alpha~\epsilon_{ijk} {\hat X}_k + {\hat X}_i {\hat X}_j (1 - \cos \alpha)
	\end{equation}
	
	\begin{equation}
		\label{rhospin}
		\rho  = i \gamma_0 \gamma_5 \gamma_j {\hat X}_j.
	\end{equation}
	

	
	\begin{eqnarray}
		\psi = 
		\left(
		\begin{array}{c}
			\psi_1 \\
			\psi_2
		\end{array}
		\right).
	\end{eqnarray}
	
	\begin{eqnarray}
		\label{quat1}
		\tau_k \psi_i &=& \psi_i \sigma_k^T, \\
		\label{quat2}
		&=& - \psi_i \sigma_2 \sigma_k \sigma_2,
	\end{eqnarray}
	

	
	\begin{equation}
		\label{conformal}
		f(\mu) = 2 \arctan \left( k \tan \frac{\mu}{2} \right).
	\end{equation}
	
	\begin{equation}
		\label{Dirac1}
		\left( i \gamma^0 \partial_t - 
		i \gamma^i \left(\frac{1+R^2}{2} \partial_{X_i} - X_i \right)
		- g U^{\gamma_5}
		\right)
		\psi \big(X_i,t \big) = 0, 
	\end{equation}
	
	\begin{equation}
		\label{Dirac2}
		\left( i \gamma^0 \partial_t + 
		i \gamma^i \left(\frac{1+{\tilde R}^2}{2} \partial_{{\tilde X}_i} - 
		{\tilde X}_i \right)
		-g U^{\gamma_5}
		\right)
		{\tilde \psi} \big( {\tilde X}_i,t \big) = 0,
	\end{equation}
	
	\begin{equation}
		U^{\gamma_5} = \exp \left( i f(\mu(R)) \gamma_5 \tau_i \frac{X_i}{R} 
		\right),
	\end{equation}
	
	\begin{equation}
		x = \left( \sin \mu \sin \theta \cos \phi,
		\sin \mu \sin \theta \sin \phi,
		\sin \mu \cos \theta,
		\cos \mu
		\right).
	\end{equation}
	
	\begin{eqnarray}
		\nonumber
		\mu &=& \left\{
		\begin{array}{ll}
			\arctan \frac{2R}{R^2-1} +\pi & {\rm for}~~~ R^2<1, \\
			\frac{\pi}{2} & {\rm for}~~~ R^2 = 1,\\
			\arctan \frac{2R}{R^2-1}      & {\rm for}~~~ R^2>1,
		\end{array} 
		\right. \\
		\nonumber
		\theta &=& \arctan \frac{\sqrt{X_1^2 + X_2^2}}{X_3}, \\
		\phi &=& \arctan \frac{X_2}{X_1}.
	\end{eqnarray}
	
	\begin{eqnarray}
		\mu &=& \left\{
		\begin{array}{ll}
			\arctan \frac{2 {\tilde R}}{1-{\tilde R}^2}  & {\rm for}~~~ 
			{\tilde R}^2<1, \\ 
			\frac{\pi}{2} & {\rm for}~~~ R^2 = 1,\\
			\arctan \frac{2 {\tilde R}}{1-{\tilde R}^2} + \pi     & {\rm for}~~~
			{\tilde R}^2>1.
		\end{array} 
		\right. 
	\end{eqnarray}
	

	
	\begin{eqnarray}
		\label{unitvectors}
		e_\mu = \left(
		\begin{array}{c}
			\sin \theta \cos \phi \\
			\sin \theta \sin \phi \\
			\cos \theta
		\end{array}
		\right),\
		e_\theta = \left(
		\begin{array}{c}
			\cos \theta \cos \phi \\
			\cos \theta \sin \phi \\
			- \sin \theta
		\end{array}
		\right)\
		{\rm and}\
		e_\phi = \left(
		\begin{array}{c}
			- \sin \phi \\
			\cos \phi \\
			0
		\end{array}
		\right).
	\end{eqnarray}
	

	
	\begin{eqnarray}
		\gamma^0 =
		\left(
		\begin{array}{cc}
			I_2 & 0 \\
			0 & -I_2
		\end{array}
		\right),\
		\gamma^i =
		\left(
		\begin{array}{cc}
			0 & \sigma_i \\
			-\sigma_i & 0
		\end{array}
		\right),\
		{\rm and}\
		\gamma_5 =
		\left(
		\begin{array}{cc}
			0 & I_2 \\
			I_2 & 0
		\end{array}
		\right).
	\end{eqnarray}
	
	\begin{equation}
		\label{trafo}
		{\tilde \psi} \left({\tilde X_i}\right) =
		\left(
		\begin{array}{cc}
			-i \sigma_j {\hat X}_j & 0 \\
			0 & -i \sigma_j {\hat X}_j 
		\end{array}
		\right) 
		\psi \left(X_i \right).
	\end{equation}
	

	

	
	\begin{eqnarray}
		\nonumber
		L_x &=& i\left(\sin \phi \frac{\partial}{\partial \theta} 
		+ \cot \theta \cos \phi \frac{\partial}{\partial \phi}\right), \\
		\nonumber
		L_y &=& i\left(- \cos \phi \frac{\partial}{\partial \theta} 
		+ \cot \theta \sin \phi \frac{\partial}{\partial \phi}\right), \\
		L_z &=& -i \frac{\partial}{\partial \phi}.  
	\end{eqnarray}
	
	\begin{equation}
		{\hat P}~ \psi\big(X_i \big) = \gamma_0~ \psi \big( - X_i \big)\
		{\rm and}\ {\hat P}~ X_i~ {\hat P}^{-1} = - X_i. 
	\end{equation}
	
	\begin{eqnarray} 
		\begin{array}{ccc}
			{\hat P}~ \gamma_0~ {\hat P}^{-1} = \gamma_0 & {\rm and} & 
			{\hat P}~ \gamma_i~ {\hat P}^{-1} = - \gamma_i, \\
			{\hat P}~ \partial_0~ {\hat P}^{-1} = \partial_0 & {\rm and} &
			{\hat P}~ \partial_i~ {\hat P}^{-1} = - \partial_i. 
		\end{array}
	\end{eqnarray}
	
	\begin{equation}
		{\hat P}~ {\tilde \psi}\big({\tilde X}_i\big) = 
		- \gamma_0~ {\tilde \psi}\big({\tilde X}_i\big).
	\end{equation}
	


	
     \begin{eqnarray}
		\label{general2}
		\frac{{\rm d}}{{\rm {d}} \mu}
		\left(
		\begin{array}{c}
			{\tilde F} \\
			{\tilde G}
		\end{array}
		\right) = \left(
		\begin{array}{cc}
			- \frac{1 + 3 \cos \mu}{2 \sin \mu} + g \sin f & E + g \cos f \\
			- \left(E - g \cos f \right) & 
			\frac{3 \left(1 - \cos \mu \right)}{2 \sin \mu} - g \sin f
		\end{array}
		\right) \left(
		\begin{array}{c}
			{\tilde F} \\
			{\tilde G}
		\end{array}
		\right).
	\end{eqnarray}
	



\begin{equation}
	\label{OD1}
	\left.
	+ \frac{3 g f^{\prime} \sin f}{2 \left(E + g \cos f \right)} 
	\frac{(1+u)}{\sqrt{1-u^2}} 
	- \frac{g^2 f^\prime \sin^2 f}{E + g \cos f} - g f^\prime \cos f
	\right) G(u) = 0, 
\end{equation}



\begin{equation}
	\label{OD2}
	\left.
	+ \frac{3 g f^\prime \sin f}{2 \left(E - g \cos f \right)} 
	\frac{(1-u)}{\sqrt{1-u^2}} 
	- \frac{g^2 f^\prime \sin^2 f}{E - g \cos f} + g f^\prime \cos f
	\right) {\tilde G}(u) = 0. 
\end{equation}

\begin{equation}
	\label{FGtilde} 
	F(u) = \sqrt{\frac{1+u}{1-u}}{\tilde G}(u).
\end{equation}

\begin{equation}
	\label{F}
	F(u) = \frac{1}{E + g \cos f}\left(
	\left(
	-\frac{3 (1 + u)}{2 \sqrt{1-u^2}} + g \sin f
	\right) G(u) 
	- \sqrt{1 - u^2} \frac{{\rm d}}{{\rm d}u} G(u)
	\right),
\end{equation}

\begin{equation}
	\label{tildeF}
	{\tilde F}(u) = \frac{1}{E - g \cos f}\left(
	\left(
	\frac{3(1 - u)}{2 \sqrt{1-u^2}} - g \sin f
	\right) {\tilde G}(u) 
	- \sqrt{1 - u^2} \frac{{\rm d}}{{\rm d}u} {\tilde G}(u)
	\right).
\end{equation}

\begin{equation}
	\label{diffeq}
	y^{\prime \prime}(u) + P(u) y^\prime(u) + Q(u) y(u) = 0.
\end{equation}

\begin{equation}
	y(u) = \sum\limits_{n=0}^\infty a_n (u-u_0)^n.
\end{equation}

\begin{eqnarray}
	P(u) &=& \frac{p(u)}{u-u_0}, \\
	Q(u) &=& \frac{q(u)}{(u-u_0)^2},
\end{eqnarray}

\begin{equation}
	p(u) = p_0 + p_1 (u-u_0) + \dots\ {\rm and}~~~
	q(u) = q_0 + q_1 (u-u_0) + \dots,
\end{equation}

\begin{equation}
	\label{eqansatz}
	y(u) = (u-u_0)^\rho \sum\limits_{n=0}^\infty a_n (u-u_0)^n.
\end{equation}

\begin{equation}
	\label{indicial}
	\rho^2 + (p_0 - 1) \rho + q_0 = 0,
\end{equation}

\begin{equation}
	\label{difeqz}
	(u-u_0)^2 z^{\prime \prime}(u) + \left(
	2 (u-u_0)^2 \frac{y_1^\prime(u)}{y_1(u)} + (u-u_0) p(u) 
	\right) z^\prime(u) = 0.
\end{equation}


		

		
		\begin{equation}\label{swaB1}
			E(A,\Phi)=SW(A,\Phi)-8\pi^2 Q,
		\end{equation}
		


		

		
		\begin{eqnarray}
			F^+_A &=& \sigma^+(\Phi\otimes\Phi^\dagger)_0,\\
			\mathcal{D}_A\Phi &=&0,
		\end{eqnarray}
		


		
		\begin{eqnarray}
			F^+_A+\chi^+ &=& \sigma^+(\Phi\otimes\Phi^\dagger)_0,\label{pe2}\\
			\mathcal{D}_A\Phi &=& 0,
		\end{eqnarray}
		



		
		\begin{equation}
			SW_\chi(A,\Phi)=E_\chi(A,\Phi)+16\pi^2\,K_\chi+8\pi^2 Q.
		\end{equation}

		

		

		
		\begin{equation}\label{cr1}
			[x^{\alpha},x^{\beta}]_{\star}:=
			x^{\alpha}\star x^{\beta}-x^{\beta}\star x^{\alpha}
			=i\theta^{\alpha\beta}.
		\end{equation}
		

		

		

		
		\begin{equation}
			F_{\alpha\beta}=\partial_\alpha A_\beta -\partial_\beta A_\alpha+
			[A_\alpha,A_\beta]_\star
		\end{equation}

		

		

		

		

		

		

	


		
		\begin{equation}\label{eq:cd}
			D_{z^a}\phi := \partial_{z^a}\phi+A_{+z^a}\star\phi-\phi\star
			A_{-z^a}.
		\end{equation}
		

		

		

	
	\begin{equation}
		-p_0 -2\rho_1 = -s-1.
	\end{equation}
	
	\begin{equation}
		\label{gs}
		g(u) = 1 + \sum\limits_{n=1}^{\infty} g_n (u-u_0)^n.
	\end{equation}
	
	\begin{equation}
		y(u) = A y_1(u) + B \left(g_s y_1(u) \log(u - u_0) + {\tilde y}_2(u)
		\right),
	\end{equation}
	
	\begin{equation}
		{\tilde y}_2(u) = (u-u_0)^{\rho_2} \left( -\frac{1}{s} + 
		\sum\limits_{n=1}^\infty h_n(u-u_0)^n \right),
	\end{equation}
	
	\begin{equation}
		\label{log}
		y(u) = A y_1(u) + B \left(y_1(u) \log(u - u_0) 
		+ (u-u_0)^{\rho_2} \sum\limits_{n=1}^\infty h_n(u-u_0)^n
		\right).
	\end{equation}
	
	\begin{equation}
		\label{erelation}
		\rho_1^{(\infty)} + \rho_2^{(\infty)} + \sum\limits_{i=1}^{n} \left(
		\rho_1^{(i)} + \rho_2^{(i)} \right) = n-1.
	\end{equation}

	

	
	\begin{eqnarray}
		\nonumber
		G(u) &=& 
		\left( \frac{1+u}{1-u} \right)^{\frac{1}{2}}
		\left( -i \sigma \cdot e_\mu~ i \sigma \cdot e_\mu 
		{\tilde F}(u) \right), \\
		&=& \left( \frac{1+u}{1-u} \right)^{\frac{1}{2}}{\tilde F}(u),
	\end{eqnarray}
	

	
	\begin{equation}
		{\tilde F}(u) = \frac{1}{E-g} \left(
		- \frac{3(1-u)}{2 \sqrt{1-u^2}}~ {\tilde G}(u) 
		+ \sqrt{1-u^2} \frac{{\rm d}}{{\rm d} u} {\tilde G}(u) 
		\right).
	\end{equation}
	

	


	
	\begin{equation}
		\label{G(u)k=1}
		\left(1-u^2\right)\left(E+gu\right) \frac{{\rm d}^2 G}{{\rm d} u^2} 
		- \left(\left(4u+1\right)\left(E+gu\right) + g\left(1-u^2\right)\right) 
		\frac{{\rm d} G}{{\rm d} u} + 
	\end{equation}
	


	









\begin{equation}\label{eq:uplus-cd}
	D_{z^a}\phi = \partial_{z^a}\phi+\mathcal{A}_{z^a}\star\phi.
\end{equation}




\begin{equation}
	D_{z^a}\phi = \partial_{z^a}\phi-\phi\star\mathcal{B}_{z^a}
\end{equation}











\begin{equation}
	\label{ansatza}
	G(u) = \sum_{n=0}^{\infty} a_n \left(u+1 \right)^n.
\end{equation}





\begin{equation}
	a_n = \prod_{i=0}^{n-1} b_i.
\end{equation}



\begin{eqnarray}
	\nonumber
	R_{{\rm conv.}} &=& \lim_{n \to \infty} \left| \frac{a_n}{a_{n+1}} \right|, \\
	\nonumber
	&=& \lim_{n \to \infty} \frac{1}{ |b_n| },  \\
	&=& 2. 
\end{eqnarray}

\begin{equation}
	\label{G(1)}
	G(1) = \sum_{n=0}^\infty a_n 2^n.
\end{equation}





\begin{eqnarray}
	\nonumber
	{y_1}^\prime (u) &=& y_2(u), \\
	{y_2}^\prime (u) &=& h\left(y_1(u),y_2(u),E \right).
\end{eqnarray}

\begin{eqnarray}
	\label{system}
	\nonumber
	{y_1}^\prime (u) &=& y_2(u), \\
	\nonumber
	{y_2}^\prime (u) &=& h \left( y_1(u),y_2(u),y_3(u) \right), \\
	{y_3}^\prime (u) &=& 0.
\end{eqnarray}







\begin{equation}
	f^{(n)}(a) = \frac{n!}{2 \pi i} \oint\limits_{C}
	\frac{f(z){\rm d} z}{(z-u)^{n+1}},
\end{equation}

\begin{equation}
	G(u) = \sum\limits_{k=0}^\infty a_k u^k,
\end{equation}

\begin{equation}
	G(u) = \frac{1}{\pi} \Re \left( 
	\int\limits_{0}^{\pi} \frac{G({\rm e}^{i \phi})
		{\rm e}^{i \phi} {\rm d} \phi}
	{{\rm e}^{i \phi} - u}
	\right), 
\end{equation}



\begin{equation}
	Q = \int\limits_{S^3} {\bar \psi} \psi.
\end{equation}

\begin{eqnarray}
	\label{Qtilde}
	Q  &=& \int\limits_{0}^{\pi} {\tilde Q}(\mu)~ {\rm d} \mu,\\
	\label{Qtilde2}
	&=& 4 \pi \int\limits_{0}^{\pi} \left(
	(1-\cos\mu)~G^2 + 
	(1+\cos \mu)~{\tilde G}^2
	\right)
	\sin^2 \mu~ {\rm d} \mu.
\end{eqnarray}

\begin{equation}
	\label{bc}
	{\tilde G}(1) = -\frac{3}{2(E+g)}.
\end{equation}

\begin{equation}
	{\tilde B}(\mu) = \frac{2}{\pi} f^\prime(\mu) \sin^2 f(\mu).
\end{equation}

\begin{equation}
	{\tilde Q}(\mu) = 8 \pi a_0^2 \sin^2 \mu.
\end{equation}








\begin{equation}\label{eq:cdo}
\hat{D}_{z^a}\phi = \hat{\partial}_{z^a}\hat{\phi}+
\hat{A}_{+z^a}\hat{\phi}-\hat{\phi}\hat{A}_{-z^a}.
\end{equation}












\begin{equation}\label{eq:topo}
\mathcal{T} = \sum_{i=\pm}\left(
F_{i\mu\nu}\,{*F_{i\mu\nu}}+\nabla_{i\mu}\mathcal{J}_{i\mu}.
\right),
\end{equation}

\begin{equation}\label{nabla1}
\nabla_{\pm\mu}\,\cdot\,:=
\partial_\mu\,\cdot\,+[A_{\pm\mu},\,\cdot\,\,],
\end{equation}












\begin{eqnarray}
	&&-(\Delta \alpha)_r = \left( \partial_r^2 + \frac 1r \partial_r +
	\frac 1{r^2} \partial_\theta^2 -\frac 1{r^2} \right) \alpha_r -
	\frac 2{r^3} \partial_\theta \alpha_\theta \,,\nonumber \\
	&&-(\Delta \alpha)_\theta = \left( \partial_r^2 -\frac 1r
	\partial_r +\frac 1{r^2} \partial_\theta^2 \right) \alpha_\theta +
	\frac 2r \partial_\theta \alpha_r \,.\label{apa4}
\end{eqnarray}

\begin{equation}
	-\frac 1r (\Delta \alpha)_\theta = -\frac 1r \omega^2
	\alpha_\theta - 2e \left( \varphi^* \partial_r \phi +
	\varphi\partial_r \phi^* \right) \,. \label{apa5}
\end{equation}

\begin{equation}
	\partial_r \Re (\delta \chi_-) |_{\partial M}\sim
	\left[ -\frac 1r \omega^2 \alpha_\theta + e\Re (\phi^*
	\partial_r \varphi -\varphi \partial_r \phi^*) \right]|_{\partial M}=0
	\label{apa6}
\end{equation}\begin{eqnarray}
	\nonumber
	G_0(u) &=& a_0, \\
	\nonumber
	G_n(u) &=& \sum\limits_{k=0}^{n}  
	a_k \left(u+1\right)^k,
\end{eqnarray}



\begin{eqnarray}
	\Psi=\Psi_0+\Phi,
\end{eqnarray}



\begin{eqnarray}
	Q(f)=\oint \frac{dw}{2\pi i} f(w) J_{\rm B}(w),\ \ \ 
	C(f)=\oint \frac{dw}{2\pi i} f(w) c(w).
\end{eqnarray}

\begin{eqnarray}
	\label{Eq:ha}
	h_a(w)=\log\left(1+\frac{a}{2}\left(w+\frac{1}{w}\right)^2\right),
\end{eqnarray}


\begin{eqnarray}
	\Phi'=e^{K(h_a)}\Phi,
\end{eqnarray}

\begin{eqnarray}
	K(f)=\oint\frac{dw}{2\pi i}f(w)\,
	\left(J_{\rm gh}(w)-\frac{3}{2}\,w^{-1}\right).
\end{eqnarray}

\begin{eqnarray}
	K(h_a)=-\tilde{q}_0 \log(1-Z(a))^2
	-\sum_{n=1}^\infty \frac{(-1)^n}{n}(q_{2n}+q_{-2n}) Z(a)^n,
\end{eqnarray}





\begin{eqnarray}
	b_0\Phi=0.
\end{eqnarray}



\begin{eqnarray}
	\label{Eq:La1}
	L(a)=(1+a)L_0 +\frac{a}{2}(L_2+L_{-2})+a(q_2-q_{-2})+4aZ(a),
\end{eqnarray}

\begin{eqnarray}
	\label{Eq:La2}
	L(a)=(1+a)L_0'+\frac{a}{2}(L_2'+L_{-2}')+4aZ(a)-1-a.
\end{eqnarray}

\begin{eqnarray}
	\label{Eq:La3}
	L(a)=2(1+a)l_0+a(l_2+l_{-2})+4aZ(a)-4(1+a),
\end{eqnarray}

\begin{eqnarray}
	l_0=\frac{1}{2}(L_0'+3),\ \ \ l_{\pm 2}=\frac{1}{2}L_{\pm 2}',
\end{eqnarray}

\begin{eqnarray}
	\label{Eq:SLcomrel}
	\left[l_0,\,l_{\pm 2}\right] = \mp l_{\pm 2},\ \ \ 
	\left[l_2,\,l_{-2}\right] = 2l_0.
\end{eqnarray}

\begin{eqnarray}
	\label{Eq:SLelemnt}
	U(s,t,u)=\exp(s\,l_{-2})\,\exp(t\,l_0)\,\exp(u\,l_2),
\end{eqnarray}



\begin{eqnarray}
	\label{Eq:stu1}
	&& au^2+2(1+a)u+a=0, \\
	\label{Eq:stu2}
	&& a\,e^t-2a su -2(1+a) s =0.
\end{eqnarray}

\begin{eqnarray}
	\label{Eq:stu3}
	a\,e^t \mp 2\sqrt{1+2a}\,s=0.
\end{eqnarray}

\begin{eqnarray}
	\label{Eq:Ldiag}
	U'(a)L(a)U'(a)^{-1}=\sqrt{1+2a}\,L_0,
\end{eqnarray}



\begin{eqnarray}
	\label{Eq:fa}
	f_a(w)=\left(\frac{w^2+Z(a)}{Z(a)\,w^2+1}\right)^{\frac{1}{2}}.
\end{eqnarray}

\begin{eqnarray}
	U'(a)\phi(w)U'(a)^{-1} = \left(\frac{df_a(w)}{dw}\right)^h
	\phi\left(f_a(w)\right).
\end{eqnarray}

\begin{eqnarray}
	\label{Eq:LT}
	L(a)=\oint \frac{dw}{2\pi i}\,w\,e^{h_a(w)}\,T'(w)
	+4Z(a)-1-a,
\end{eqnarray}

\end{document}
