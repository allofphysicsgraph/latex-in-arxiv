\begin{equation}
%\begin{array}{ccl}
f(\t)&: \quad& e^{-\frac{2}{3} \t} (f_{2,2} \t^2+f_{2,1} \t  +f_{2,0})+e^{-\frac{4}{3} \t} (f_{4,5} \t^5+ \ldots+f_{4,0})\nn\\&&+e^{-2 \t} (f_{6,9} \t^9+\ldots+f_{6,1} \t  +V_f)+e^{-\frac{8}{3} \t} (f_{8,12} \t^{12}+\ldots)+\ldots\label{UVKSexp}\\[-2mm]&&\nn
%\end{array}
\end{eqnarray}
%\end{flushleft}
where we have labeled the free parameters as $M_a$, $V_a$ according whether they correspond to a mass or to a vev (actually there is a certain ambiguity in choosing which combinations of the coefficients $M_{1,2}$ and $V_{1,2}$ correspond exactly to the masses and vevs of the two gauginos)
, and the derived coefficients as $\phi_{i,j}$, with $i$ order of the negative exponential and $j$ of the power of $\t$ multiplying the coefficient itself.
%Since it is a matter of convention which of the coefficients appearing at order 1 (and 3) can be taken as $M_{N_2}$ and which as $M_1$  (and as $V_{N_2}$ and $V_1$) we choose as independent the ones that simplify the notations and label them simply as $M_{1,2}$ ($V_{1,2}$). 
%Since they appear in varie eq c'e' certa ambiguita' in identificare chi di m1 e'chi
%There is a certain ambiguity in choosing which combinations of the coefficients appearing at order 1 (and 3) represent exactly the mass (and vev) parameters for the two gauginos. thus we label the free parameters of that orders simply as $M_{1,2}$ ($V_{1,2}$). 

The derived coefficients that will be useful in future computations are, at linear order in the mass parameters: 
\[
y_{1,0}= -\frac{4}{3 P} e^{\Phi_{\infty}}M_2+2 M_1 ,\qquad
F_{1,1}= \frac{3}{2} P M_1, \qquad F_{1,0}=-3 e^{-\Phi_{\infty}} M_2+ \frac{33}{8} P M_1,\\[1mm]
\]
\[
F_{3,1}=P V_1+O(M_i^3), \quad F_{3,0}= P V_1-e^{- \Phi_{\infty}} V_2+O(M_i^3),\quad
k_{3,1}=-g_{3,1}=e^{\Phi_{\infty}} P V_1+O(M_i^3), \\[2mm]
\]
\[
A_{4,1}=-\frac{1}{2} M_1 V_1 + O(M_i^4), \qquad A_{4,0}=-\frac{1}{8}V_{\Phi}-\frac{13}{16} M_1 V_1+\frac{5}{6 P}e^{-\Phi_{\infty}} M_2 V_1 + O(M_i^4),\\[2mm]
\]
\[
q_{4,1}=-\frac{1}{20} M_1 V_1 + O(M_i^4), \qquad q_{4,0}=-\frac{3}{20} V_{\Phi}+\frac{M_1 V_1}{16}+\frac{M_2 V_1}{5 P e^{\Phi_{\infty}}} +\frac{M_1 V_2}{40 P e^{\Phi_{\infty}}} + O(M_i^4),\\[2mm]
\]
\[
f_{4,1}=-\frac{3}{10} M_1 V_1 + O(M_i^4), \qquad f_{4,0}=\frac{1}{10}V_{\Phi}-\frac{M_1 V_1}{2}+\frac{8 M_2 V_1}{15 P e^{\Phi_{\infty}}} -\frac{M_1 V_2}{10 P e^{\Phi_{\infty}}} + O(M_i^4),\\[2mm]
\]
\[
\Phi_{4,0}=-\frac{5}{4}V_{\Phi}-\frac{4 e^{\Phi_{\infty}}V_s}{3 P}+\frac{21}{2} M_1 V_1-\frac{16 M_2 V_1}{3 P e^{\Phi_{\infty}}} +\frac{8 M_2 V_2}{3 P^2 e^{2 \Phi_{\infty}}} -\frac{11 M_1 V_2}{2 P e^{\Phi_{\infty}}} + O(M_i^4),\\[2mm]
\]
\[
k_{4,2}=\frac{3}{2} P e^{\Phi_{\infty}} M_1 V_1, \qquad k_{4,1}=-\frac{3}{2}V_{\Phi} P e^{\Phi_{\infty}}+\frac{51}{8} M_1 V_1 P e^{\Phi_{\infty}}-2  M_2 V_1 -\frac{3}{2} M_1 V_2 + O(M_i^4),\\[2mm]
\]
\begin{equation}
g_{4,2}=k_{4,2}, \qquad g_{4,1}=k_{4,1},\qquad V_{\Phi}= k_{\Phi} (\frac{M_2 V_2}{P e^{\Phi_{\infty}}}+M_1 V_1).
\label{ksexpcoeff}
\end{equation}
where $k_{\Phi}$ scales in such a way that even when $M_{2,1} \to 0$, $V_{\Phi}$ remains finite.\\
All the other derived coefficients $\phi_{i,j}$ for $i \le 4 $ (and almost all for $i \ge 5$) are of order $(M_a)^i$ in the mass parameters. 
%As in the MN case it is possible to express some vev parameters in a different way; in particular we can write:
%\[ V_{\Phi}= k_{\Phi} (\frac{M_2 V_2}{P e^{\Phi_{\infty}}}+M_1 V_1)
%\]

%$\subsection{Numerical Interpolation}
Once the IR and the UV expansions are found it is possible to relate the UV parameters to the IR ones, and find numerical solutions. Anyway plots are not particularly enlightening and we do not report them here.\
\subsection{The gluino condensate}
It was already pointed out in section \ref{glui} that the supergravity field dual to the gluino bilinear tr$\lambda \lambda$ is one of the polarization of $C_2=C_2^{RR}+i B_2^{NS}$.
Since
\begin{eqnarray}
H_3 = d B_2 &=& d u \wedge (\dot{g} g_1 \wedge g_2+\dot{k} g_3 \wedge g_4)+\frac{1}{2} (k-g) g_5 \wedge (g_1 \wedge g_3+g_2 \wedge g_4) \nn \\
F_3 = d C_2^{RR} &=& \dot{F} d u \wedge (g_1 \wedge g_3+ g_2 \wedge g_4)+ F g_5 \wedge g_1 \wedge g_2+ (2 P-F) g_5 \wedge g_3 \wedge g_4\nn
\end{eqnarray}
we have that $G_3 = d C_2$ behaves at $\t\to \infty$ as (we write only the polarizations along $T^{1,1}$):
\begin{displaymath}
\begin{array}{ccl}
G_3 &\simeq& \,\left[P+e^{-\frac{\t}{3}} (F_{1,1} \t +F_{1,0})+e^{-\t} (F_{3,5} \t^5+ \ldots+(P V_1 +O(M_i^3) )\t+\ldots)\right] g_5 \wedge g_1 \wedge g_2
\nn \\
&+& \, \left[P-e^{-\frac{\t}{3}}(F_{1,1} \t +F_{1,0})-e^{-\t} (F_{3,5} \t^5+ \ldots+(P V_1 +O(M_i^3) )\t+ \ldots)\right] g_5 \wedge g_3 \wedge g_4 \nn \\
&+& \,i \left[M_2 e^{-\frac{\t}{3}}+ e^{-\t}(k_{3,5} \t^5+ \ldots+ (P e^{\Phi_{\infty}} V_1+O(M_i^3)) \t+\ldots)\right]  g_5 \wedge (g_1 \wedge g_3+g_2 \wedge g_4)
\end{array}
\end{displaymath}
If we subtract the values of $G_3$ for two solutions with the same masses $M_i$ we get, at leading order (we relabel for simplicity $\Delta V_1$ as $V_1$):
\[
\Delta G_3 = (P V_1 \t e^{-\t}+\ldots) \ \omega_3, \quad \omega_3=-\left[g_5 \wedge(g_3 \wedge g_4 - g_1 \wedge g_2)+ i g_s g_5 \wedge (g_1 \wedge g_3+g_2 \wedge g_4) \right] 
\]
where $g_s= e^{\Phi_{\infty}}$.
Thus the polarization of $C_2$ we are interested in is
\[
C_2= - P V_1 \t e^{-\t} \omega_2, \qquad \omega_2=- \left[(g_1 \wedge g_3 + g_2 \wedge g_4)+ i g_s (g_1 \wedge g_2-g_3 \wedge g_4) \right]%, \qquad %g_s = e^{\Phi_{\infty}}
\]
For large $\t$ we can perform the change of variables $\t \to \frac{1}{3}$ Log$(\epsilon^{-\frac{2}{3}} u)$.
Since the deformation parameter $\epsilon$ is related to the 4d mass scale as
$\epsilon \sim m^{-\frac{2}{3}}$, finally we get that for large $\t$
the operator we would like to associate with the gluino condensate scales as
\[
P V_1 \frac{m^3}{u^3} \texttt{Log}\frac{u^3}{m^3}
\]
This result can be expressed simply as $P V_1 m^3/u^3$, according to field theory prediction, with a redefinition of the scale as in \cite{dvlm,beme}; it would be interesting to evaluate the consequences of this issue.
\subsection{Vacuum energy}
Looking at the expression (\ref{potks}) for the effective action it is clear that, after integration by parts and the use of the equations of motion, the action becomes
\begin{equation}
I \sim \lim_{r \to \infty} \frac{3}{2} \dot{A}(r) e^{4 A(r)}
\end{equation}
In order to use the results of previous section we also have to perform the change of variables $d r \to d \t e^{4 p(\t)}$. The action is therefore given by:
\begin{equation}
I \sim \lim_{\t \to \infty} \frac{9}{2} \dot{A}(\t) e^{4 A(\t) -4 q(\t)+ 4 f(\t)}
\end{equation}
Even in the KS case, in order to cancel the divergences common to all vacuum energies, we have to subtract the actions of two solutions with different vevs. 
Again we take the supersymmetric solution, that is the KS one, as our  reference geometry.
It reads:
\begin{equation}
\begin{array}{cccccccc}
A_{KS}(\t)&=&\frac{1}{3} \t &+\frac{1}{6}\texttt{Log}(\t-1/4)&+&\frac{1}{6}\texttt{Log}(2^{-6} 3 P^2 e^{\Phi_{\infty}})
%A_{\infty}
&+& O(e^{-2 \t}) \\
q_{KS}(\t)&=&  &+\frac{1}{6}\texttt{Log}(\t-1/4)&+&\frac{1}{6}\texttt{Log}(3 \sqrt{3} P^2 e^{\Phi_{\infty}})&+& O(e^{-2 \t})\\
f_{KS}(\t)&=& & & & & &O(e^{-2 \t})
\end{array}
\end{equation}
Accounting also for the freedom to shift $\t$ and $\Phi_{\infty}$ the action for this supersymmetric background is given, at leading order, by:
\begin{equation}
I_{KS} \propto \lim_{\t \to \infty} \;e^{\frac{4}{3} (\t+\t^*)}\;   \partial_{\t} \left(\frac{\t+\t^*}{3} +
\frac{1}{6} \log\left(\t-1/4+\t^*\right)+\frac{1}{6} \log (2^{-6} 3 P^2 e^{\Phi_{\infty}+\Phi^*}) \right)
\label{evks}
\end{equation}
The general solution have of course the form given in (\ref{UVKSpol}) and (\ref{UVKSexp}).\\
In KS solution $S_{\infty}$ is chosen in such a way that polynomial coefficients $a_{0,j}$, $q_{0,j}$ in (\ref{UVKSpol}) vanish. 
To account for the freedom to shift $s=g+k$  we take the value of $S_{\infty}$ for the general solution as the one that makes the above mentioned coefficients vanish, plus a correction $S^*$.
The polynomial series in $A(\t)$, $q(\t)$ thus becomes, at leading order in $S^*$:
\[
S^* \frac{1}{ 6 P \t e^{\Phi_{\infty}}} (1+\frac{1}{4 \t}+\frac{1}{(4 \t)^2}+\ldots) = S^* \frac{1}{6 P \t e^{\Phi_{\infty}}}\, \cdot \, \frac{1}{1-\frac{1}{4 \t}} = S^* \frac{1}{6 P e^{\Phi_{\infty}}(\t-\frac{1}{4})}
\]
So the action for the general solution is:
\begin{eqnarray}
%%%%%\[
%\begin{array}{c}
I_{} \propto \lim_{\t \to \infty}  e^{4/3 \t+4 (A_{S}(\t)- q_{S}(\t)+f_{S}(\t))} \times
\partial_{\t}\left(\frac{\t}{3} +\frac{1}{6} \log (\t-\frac{1}{4})+ \frac{1}{6} \log (2^{-6} 3 P^2 e^{\Phi_{\infty}})
\right.\qquad \label{evgen}\\
%\end{equation}

