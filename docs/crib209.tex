%grep -i math crib209.tex |grep -oE '" is .*?\('|sed -r 's/ \(|" is //g'|sed -r 's/^\$|\$$//g'|sed 's/\\verb/\n/g'|sed -r 's/^"\\.*?\\/\\/g'|sed 's/$.//g'|sort|uniq|awk '{print length,$0}'|sort -nr|cut -d ' ' -f2-


% CRIB209.TEX -- Body of LaTeX Command Summary for v2.09

\thispagestyle{empty}

\noindent
{\large\bf \LaTeX{} Command Summary}
\smallskip

\begin{flushleft}
\hspace*{2em} Lloyd Botway and Chris Biemesderfer\\
\hspace*{2em} Space Telescope Science Institute\\
\hspace*{2em} Baltimore, MD\ \ 21218\\[.5ex]
\hspace*{2em} \today
\end{flushleft}
\smallskip

\ifsmallcrib \footnotesize \fi

\noindent
This listing contains short descriptions of the control sequences that are
likely to be handy for users of \LaTeX~v2.09 layered on \TeX~v2.0.  Some
of these commands are \LaTeX{} macros, while others belong to plain \TeX{};
no attempt to differentiate them is made.  We would
appreciate hearing about errors in the list.
\bigskip

\raggedright \parindent -1em

\verb*"\ " --- ordinary space after period.

\verb"\!" --- negative thin space $ = -\frac{1}{6} $ quad;
	\ifsmallcrib \relax \else \linebreak \fi
	\verb"xx\!x" yields $xx\!x$ (math mode).

\verb.\". makes an umlaut, as \"o.

\verb"\#" prints a pound sign: \#.

\verb"\$" prints a dollar sign: \$.

\verb"\%" prints a percent sign: \%.

\verb"\&" prints an ampersand: \&.

\verb"\'" in \verb"tabbing" environment moves current column to the right
	of the previous column. Elsewhere, acute accent, as \'o.

\verb"\(" --- start math mode. Same as \verb"\begin{math}" or \verb"$".

\verb"\)" --- end math mode.  Same as \verb"\end{math}" or \verb"$".

\verb"\*" is a discretionary multiplication sign, at which a line break
	is allowed.

\verb"\+" moves left margin to the right by one tab stop.
	Begin tabbed line.

\verb"\," --- thin space $ = \frac{1}{6} $ quad;
	\verb"xx\,x" yields $xx\,x$.  It is not restricted to math mode.

\verb"\-" in \verb"tabbing" environment, moves left margin to the left
	by one tab stop. Elsewhere, optional hyphenation.

\verb"\." puts a dot accent over a letter, as \.o.

\verb"\/" inserts italics adjustment space.

\verb"\:" --- medium space $ = \frac{2}{9} $ quad;
	\verb"xx\:x" yields $xx\:x$ (math mode).

\verb"\;" --- thick space $ = \frac{5}{18} $ quad;
	\verb"xx\;x" yields $xx\;x$ (math mode).

\verb"\<" in \verb"tabbing" environment, puts text to left of local
	left margin.

\verb"\=" in \verb"tabbing" environment, sets a tab stop.
	Elsewhere, makes a macron accent, as \=o.

\verb"\>" in \verb"tabbing" environment is a forward tab. Otherwise,
	medium space $ = \frac{2}{9}$ quad (math mode).

\verb"\@" declares that the period that follows to be a sentence-ending
	period.

\verb"\[" --- same as \verb"\begin{displaymath}" or \verb"$$".

\verb"\\" terminates a line.

\verb"\\*" terminates a line, but disallows a pagebreak.

\verb"\]" --- same as \verb"\end{displaymath}" or \verb"$$".

\verb"\^" makes a circumflex, as \^o.

\verb"\_" is an underscore, as in {\em hours\_worked}.

\verb"\`" in \verb"tabbing" environment moves all text which follows (up to
	\verb"\\") to the right margin. Elsewhere, grave accent
	, as \`o.

\verb"\{" prints a curly left brace: \{.

\verb"\|" is $\|$ (math mode).

\verb"\}" prints a curly right brace: \}.

\verb"\~" makes a tilde, as \~n.

%aaa
\verb"\a'" makes an acute accent in \verb"tabbing" environment, as \'o.

\verb"\a`" makes a grave accent in \verb"tabbing" environment, as \`o.

\verb"\a=" makes a macron accent in \verb"tabbing" environment, as \=o.

\verb"\aa" is \aa. \verb"\AA" is \AA.

\verb"\acute" makes an acute accent: $\acute a$ (math mode).

\verb"\addcontentsline{toc}{section}{name}" adds the command
	\verb"\contentsline{section}{name}" to the \verb".toc" file.

\verb"\address{text}" declares the return address in the \verb"letter"
	document style.

\verb"\addtocontents{toc}{text}" writes \verb"text" to the \verb".toc" file.

\verb"\addtocounter{name}{amount}" adds \verb"amount" to counter \verb"name".

\verb"\addtolength{\nl}{length}" adds \verb"length" to length command
	\verb"\nl". See also \verb"\setlength", \verb"\newlength",
	\verb"\settowidth".

\verb"\ae" is \ae.  \verb"\AE" is \AE.

\verb"\aleph" is $\aleph$ (math mode).

\verb"\alph{counter}" prints \verb"counter" as lower-case letters.
	\verb"\Alph{counter}" prints upper-case letters.

\verb"\alpha" is $\alpha$ (math mode).

\verb"\amalg" is $\amalg$ (math mode).

\verb"\and" separates multiple authors for the \verb"\maketitle" command.

\verb"\angle" is $\angle$ (math mode).

\verb"\appendix" starts appendices.

\verb"\approx" is $\approx$ (math mode).

\verb"\arabic{counter}" prints \verb"counter" as arabic numerals 1, 2, etc.

\verb"\arccos" is $\arccos$ (math mode).

\verb"\arcsin" is $\arcsin$ (math mode).

\verb"\arctan" is $\arctan$ (math mode).

\verb"\arg" is $\arg$ (math mode).

\verb"\arraycolsep" --- width of the space between columns in an
	\verb"array" environment.

\verb"\arrayrulewidth" --- width of the rule created in \verb"tabular"
	or \verb"array" environment by \verb"|", \verb"\hline", or
	\verb"\vline".

\verb"\arraystretch" --- scale factor for interrow spacing in \verb"array"
	and \verb"tabular" environments.

\verb"\ast" is $\ast$ (math mode).

\verb"\asymp" is $\asymp$ (math mode).

\verb"\author{names}" declares author(s) for the \verb"\maketitle" command.

%bbb
\verb"\b" is a ``bar-under'' accent, as \b o.

\verb"\backslash" is $\backslash$ (math mode).

\verb"\bar" puts a macron over a letter: $\bar a$ (math mode).

\verb"\baselineskip" --- distance from bottom of one line of a paragraph to
	bottom of the next line.

\verb"\baselinestretch" --- factor by which \verb"\baselineskip" is multiplied
	each time a type size changing command is executed.

\ifamsmath
\verb"\Bbb" produces blackboard-style letters, as $\Bbb B$ (math mode
	and \verb"amsmath" substyle).
\fi

\verb"\begin{"{\it environment\/}\verb"}" --- always paired with
	\ifsmallcrib \relax \else \linebreak \fi
	\verb"\end{"{\it environment\/}\verb"}".
	Following are the assorted environments.

\verb"\begin{abstract}" starts an environment for producing an abstract.

\verb"\begin{array}{lrc}" starts array environment with 3 columns,
	left-justified, right-justified, and centered. Separate columns
	with \&, and end lines with \verb"\\". \verb"@{text}" between
	\verb"l", \verb"r" or \verb"c" arguments puts \verb"text"
	between columns.

\verb"\begin{center}" starts an environment in which every line is centered.
	End lines with \verb"\\".

\verb"\begin{description}" starts a labeled list.  Items are
	indicated by \verb"\item[label]".

\verb"\begin{displaymath}" sets mathematics on lines of its own.
	Same as \verb"\[" or \verb"$$".

\verb"\begin{document}" starts the actual text of a document.  Required.

\verb"\begin{enumerate}" starts a numbered list.

\verb"\begin{eqnarray}" starts a \verb"displaymath" environment in which
	more than one equation can be accommodated.  Separate equations 
	with \verb"\\" or \verb"\\*"; use \verb"\nonumber" to suppress
	numbering a particular equation.

\verb"\begin{eqnarray*}" begins an environment like the \verb"eqnarray"
	environment except that the equations aren't numbered.

\verb"\begin{equation}" starts a \verb"displaymath" environment and adds
	an equation number.

\verb"\begin{figure}[pos]" begins a floating environment, which may be
	optionally placed at \verb"pos" (see \verb"positions" on
	page~\pageref{pos-ref}). Document styles \verb"report" and
	\verb"article" use the default \verb"tbp".

\verb"\begin{figure*}[pos]" begins a two-column-wide figure.
	See \verb"\begin{figure}".

\verb"\begin{flushleft}" starts environment with ragged right-hand margin.
	Separate lines with \verb"\\". See \verb"\raggedright".

\verb"\begin{flushright}" starts environment with ragged left-hand margin.
	Separate lines with \verb"\\". See \verb"\raggedleft".

\verb"\begin{itemize}" starts a ``bulleted'' ($\bullet$) list. 
        Start each item with \verb"\item".

\verb"\begin{list}{labeling}{spacing}" starts a general list environment.
	\verb"labeling" specifies how items are labeled when \verb"\item"
	has no argument.  \verb"spacing" is an optional list of
	spacing parameters.

\verb"\begin{math}" starts a math display like this: $x^2 + y^2$, within text.
	Same as \verb"$".

\verb"\begin{minipage}[pos]{vsize}" starts a box of height \verb"vsize".
	Text will be positioned according to \verb"pos" (see \verb"positions"
	on page~\pageref{pos-ref}).

\verb"\begin{picture}"$(x,y)(x_l,y_l)$ starts a picture environment whose
	width is $x$ units, height is $y$ units, and lower-left corner is
	the point $(x_l,y_l)$.  Set units with \verb"\unitlength".

\verb"\begin{quotation}" starts an environment with wider margins, normal
	paragraph indenting, and offset from the text at top and bottom.

\verb"\begin{quote}" starts an environment with wider margins, no paragraph
	indenting, and offset from the text at top and bottom.

\verb"\begin{tabbing}" starts a columnar environment.  Use commands
	\verb"\=" (set tab), \verb"\>" (tab), \verb"\<" (backtab),
	\verb"\+" (indent one tab stop), \verb"\-" (outdent one tab stop),
	\verb"\`" (flush right), \verb"\'" (flush left), \verb"\pushtabs",
	\verb"\poptabs", \verb"\kill", \verb"\\".

\verb"\begin{table}[pos]" begins a floating environment, which may be
	optionally placed at \verb"pos" (see \verb"positions" on
	page~\pageref{pos-ref}). Document styles \verb"report" and
	\verb"article" use the default \verb"tbp".

\verb"\begin{table*}[pos]" begins a two-column-wide table.
	See \verb"\begin{table}".

\verb"\begin{tabular}{arg}" starts an array environment which can be used
	in or out of math mode. \verb"arg" contains column text positioning
	commands \verb"r", \verb"l", \verb"c",
	\verb"@{...}", \verb"p{length}" (see \verb"positions" on
	page~\pageref{pos-ref}). \ifsmallcrib \relax \else \linebreak \fi
	\verb"|" produces vertical line
	between columns. \verb"*{7}{r|l|}" repeats that entry 7 times.

\verb"\begin{theorem}" --- see \verb"\newtheorem".

\verb"\begin{titlepage}" is an environment with no page number, and causes
	following page to be numbered ``1''.

\verb"\begin{verbatim}" starts an environment which will be typeset exactly
	as you type it, carriage returns and all, usually in \verb"typewriter"
	font.

\verb"\begin{verse}" starts an environment for poetry with wider margins,
	no paragraph indenting, and ragged right margin.

\verb"\beta" is $\beta$ (math mode).

\verb"\bf" switches to {\bf bold face} type.

\verb"\bibitem{ref} text" creates a bibliography entry \verb"text",
	numbers it, and labels it with reference label \verb"ref".

\verb"\bibliography{file}" --- insert bibliography from file \verb"name.bib"
	at this point in text.

\verb"\bibliographystyle{style}" --- a format specifier, like
	\verb"\documentstyle".

\verb"\bigcap" is $\bigcap$ (math mode).

\verb"\bigcirc" is $\bigcirc$ (math mode).

\verb"\bigcup" is $\bigcup$ (math mode).

\verb"\bigodot" is $\bigodot$ (math mode).

\verb"\bigoplus" is $\bigoplus$ (math mode).

\verb"\bigotimes" is $\bigotimes$ (math mode).

\verb"\bigtriangledown" is $\bigtriangledown$ (math mode).

\verb"\bigtriangleup" is $\bigtriangleup$ (math mode).

\verb"\bigskip" --- standard ``big'' vertical skip.

\verb"\bigskipamount" --- default length for \verb"\bigskip".

\verb"\bigsqcup" is $\bigsqcup$ (math mode).

\verb"\biguplus" is $\biguplus$ (math mode).

\verb"\bigvee" is $\bigvee$ (math mode).

\verb"\bigwedge" is $\bigwedge$ (math mode).

\verb"\bmod" is binary modulo expression $u \bmod m$ (math mode).

\verb"\boldmath" changes math italics and math symbols to boldface.  Should
	be used {\em outside} of math mode.

\verb"\bot" is $\bot$ (math mode).

\verb"\bottomfraction" --- maximum fraction of page occupied by floats at
	the bottom.

\verb"\bowtie" is $\bowtie$ (math mode).

\verb"\Box" is $\Box$ (math mode).

\verb"\breve" makes a breve accent: $\breve a$ (math mode).

\verb"\bullet" is $\bullet$ (math mode).

%ccc
\verb"\c" is a cedilla, as \c c.

\verb"\cal" produces calligraphic letters, as $\cal B$ (math mode).

\verb"\cap" is $\cap$ (math mode).

\verb"\caption[loftitle]{text}" creates a numbered caption in a \verb"figure"
	or \verb"table" environment. Optional \verb"loftitle" contains
	entry for the list of figures if different from \verb"text".

\verb"\cc{text}" declares list of copy recipients for \verb"letter"
	document style.

\verb"\cdot" is $\cdot$ (math mode).

\verb"\cdots" makes three dots centered on the line: $\cdots$
	(cf.\ \verb"\ldots") (math mode).

\verb"\centering" declares that all text following is to be centered
	(cf.\ \verb"\begin{center}").

\verb"\chapter[toctitle]{text}" begins a new section, automatically headed and
	numbered.  Optional \verb"toctitle" contains entry for the
	table of contents if different from \verb"text".

\verb"\chapter*{title}" is like \verb"\chapter{title}", but adds no
	chapter number or table of contents entry.

\verb"\check" makes a h\'a\v cek, as $\check a$ (math mode).

\verb"\chi" is $\chi$ (math mode).

\verb"\circ" is $\circ$ (math mode).

\verb"\circle{diameter}" as a valid argument for \verb"\put" in a
	\verb"picture" environment, draws a circle.

\verb"\circle*{diameter}" is like \verb"\circle", but draws a solid circle.

\verb"\cite[subcit]{ref}" produces a reference, in square brackets, to a
	bibliographic item created with \verb"\bibitem{ref}".
	optional sub-citation \verb"subcit" can be inserted in the entry.

\verb"\cleardoublepage" forces next page to be a right-hand, odd-numbered page.

\verb"\clearpage" ends a page where it is, and puts pending figures or tables
	on separate float pages with no text.

\verb"\cline{i-j}" draws a horizontal line across columns \verb"i" through
	\verb"j" inclusive in \verb"array" or \verb"tabular" environments.

\verb"\closing{text}" declares the closing in \verb"letter" document style.

\verb"\clubsuit" is $\clubsuit$ (math mode).

\verb"\columnsep" --- distance between columns in two-column text.

\verb"\columnseprule" --- width of the rule between columns on two-column
	pages.

\verb"\columnwidth" --- width of the current column.  Equals \verb"\textwidth"
	in single-column text.

\verb"\cong" is $\cong$ (math mode).

\verb"\coprod" is $\coprod$ (math mode).

\verb"\copyright" is \copyright.

\verb"\cos" is $\cos$ (math mode).

\verb"\cosh" is $\cosh$ (math mode).

\verb"\cot" is $\cot$ (math mode).

\verb"\coth" is $\coth$ (math mode).

\verb"\csc" is $\csc$ (math mode).

\verb"\cup" is $\cup$ (math mode).

%ddd
\verb"\d" is a ``dot under'' accent, as \d o.

\verb"\dag" is \dag.

\verb"\dagger" is $\dagger$ (math mode).

\verb"\dashbox{dwid}(width,height)[pos]{text}" creates a dashed rectangle
	around \verb"text" in a \verb"picture" environment. Dashes are
	\verb"dwid" units wide; dimensions of rectangle are \verb"width"
	and \verb"height";  text is positioned at optional \verb"pos"
	(see \verb"positions"\ on page~\pageref{pos-ref}).

\verb"\dashv" is $\dashv$ (math mode).

\verb"\date{adate}" declares the date for the \verb"\maketitle"
	command.  The default is \verb"\today".

\verb"\day" --- current day of the month.

\verb"\dblfloatpagefraction" --- minimum fraction of a float page that must
	be occupied by floats, for two-column float pages.

\verb"\dblfloatsep" --- distance between floats at the top or bottom of a
	two-column float page.

\verb"\dbltextfloatsep" --- distance between double-width floats at the top
	or bottom of a two-column page and the text on that page.
	
\verb"\dbltopfraction" --- maximum fraction at the top of a two-column
	page that may be occupied by floats.

\verb"\ddag" is \ddag.

\verb"\ddagger" is $\ddagger$ (math mode).

\verb"\ddot" makes a dieresis over a letter: $\ddot a$ (math mode).

\verb"\ddots" produces a diagonal ellipsis $\ddots$ (math mode).

\verb"\deg" is $\deg$ (math mode).

\verb"\delta" is $\delta$. \verb"\Delta" is $\Delta$ (math mode).

\verb"\det" is $\det$ (math mode).

\verb"\diamond" is $\diamond$. \verb"\Diamond" is $\Diamond$ (both math mode).

\verb"\diamondsuit" is $\diamondsuit$ (math mode).

\verb"\dim" is $\dim$ (math mode).

\verb"\displaystyle" switches to \verb"displaymath" or \verb"equation"
	environment typesetting (math mode).

\verb"\div" is $\div$ (math mode).

\verb"\documentstyle[substy]{sty}" determines default font, headings, etc. for
	document of style \verb"sty" (and optional substyle \verb"substy")
	Styles: \verb"article", \verb"book", \verb"letter", \verb"report",
	\verb"slides".  Substyles: \verb"11pt", \verb"12pt", \verb"acm",
	\verb"draft", \verb"fleqn", \verb"leqno",
	\verb"twocolumn", \verb"twoside".

\verb"\dot" makes a dot over a letter: $\dot a$ (math mode).

\verb"\doteq" is $\doteq$ (math mode).

\verb"\dotfill" expands to fill horizontal space with row of dots.

\verb"\doublerulesep" --- horizontal distance between vertical rules created
	by \verb"||" in \verb"tabular" or \verb"array" environment.

\verb"\downarrow" is $\downarrow$. \verb"\Downarrow" is $\Downarrow$
	(math mode).

%eee
\verb"\ell" is $\ell$ (math mode).

\verb"\em" toggles between roman and {\em italic} fonts for {\em emphasis}.

\verb"\emptyset" is $\emptyset$ (math mode).

\verb"\encl{text}" declares a list of enclosures for \verb"letter"
	document style.

\verb"\end{"{\it environment\/}\verb"}" ends an environment begun by
	\verb"\begin{"{\it environment\/}\verb"}" (q.v.).

\verb"\epsilon" is $\epsilon$ (math mode).

\verb"\equiv" is $\equiv$ (math mode).

\verb"\eta" is $\eta$ (math mode).

\verb"\evensidemargin" --- distance between left side of page and text's
	normal left margin, for even-numbered pages in two-sided printing.

\verb"\exists" is $\exists$ (math mode).

\verb"\exp" is $\exp$ (math mode).

%fff
\verb"\fbox{text}" makes a \fbox{framed box} around \verb"text".

\verb"\fboxrule" --- thickness of ruled frame for \verb"\fbox" and
	\verb"\framebox".

\verb"\fboxsep" --- space between frame and text for \verb"\fbox" and
	\verb"\framebox".

\verb"\fill" --- rubber length (glue) that can stretch to arbitrary length.
	Usually used to justify text a particular way.

\verb"\flat" is $\flat$ (math mode).

\verb"\floatpagefraction" --- minimum fraction of a float page occupied by
	floats.

\verb"\floatsep" --- distance between floats that appear at the top
	or bottom of a text page.

\verb"\flushbottom" causes pages to be stretched to \verb"\textheight".

\verb"\fnsymbol{counter}" prints \verb"counter" as one of the set of
	``footnote symbols''.  \verb"counter" must be less than 10.

\verb"\footheight" --- height of box at bottom of page that holds page number.

\verb"\footnote{text}" creates a footnote of \verb"text".

\verb"\footnotemark" puts a footnote number into the text.

\verb"\footnotesep" --- height of strut placed at beginning of footnote.

\verb"\footnotesize" switches to footnote-sized type.

\verb"\footskip" --- vertical distance between bottom of last line of text
	and bottom of page footing.

\verb"\footnotetext{text}" specifies the text for a footnote which was
	indicated by a \verb"\footnotemark".

\verb"\forall" is $\forall$ (math mode).

\verb"\frac{numerator}{denominator}" produces a fraction in \verb"math"
	environments.

\verb"\frame{text}" makes a framed (outlined) box around \verb"text", with
	no margin between the text and the frame.

\verb"\framebox[size][pos]{text}" produces a \fbox{framed box} of dimension
	\verb"size" containing \verb"text", optionally \verb"pos"itioned
	\verb"l" or \verb"r". \ifsmallcrib \relax \else \linebreak \fi
	In \verb"picture" environment,
	\verb"\framebox(width,height)[pos]{text}" creates a rectangle
	around \verb"text"; dimensions of rectangle are \verb"width"
	and \verb"height";  text is positioned at optional \verb"pos"
	(see \verb"positions" on page~\pageref{pos-ref}).

\verb"\frown" is $\frown$ (math mode).

\verb"\fussy" is the default declaration for the line-breaking algorithm
	(cf. \verb"\sloppy").

%ggg
\verb"\gamma" is $\gamma$. \verb"\Gamma" is $\Gamma$ (math mode).

\verb"\gcd" is $\gcd$ (math mode).

\verb"\ge" is $\ge$ (math mode).

\verb"\geq" is $\geq$ (math mode).

\verb"\gets" is $\gets$ (math mode).

\verb"\gg" is $\gg$ (math mode).

\verb"\glossary{text}" appends \verb"text" to the \verb".glo" file
	by writing a \verb"\glossaryentry" command.
	
\verb"\glossaryentry{text}{ref}" is written to the \verb".glo" file for
	\verb"\glossary{text}" occuring at reference \verb"ref".

\verb"\grave" makes a grave accent: $\grave a$ (math mode).

%hhh
\verb"\H" prints a long Hungarian umlaut, as \H o.

\verb"\hat" makes a circumflex: $\hat a$ (math mode).

\verb"\hbar" is $\hbar$ (math mode).

\verb"\headheight" --- height of box at top of page that holds running head.

\verb"\headsep" --- vertical distance between bottom of head and top of text.

\verb"\heartsuit" is $\heartsuit$ (math mode).

\verb"\hfill" is \verb"\hspace{\fill}" (cf. \verb"\fill").

\verb"\hline" draws a horizontal line across all columns of a \verb"tabular"
	or \verb"array" environment.

\verb"\hom" is $\hom$ (math mode).

\verb"\hookleftarrow" is $\hookleftarrow$ (math mode).

\verb"\hookrightarrow" is $\hookrightarrow$ (math mode).

\verb"\hrulefill" expands to fill horizontal space with horizontal rule.

\verb"\hspace{len}" leaves a horizontal space of dimension \verb"len".

\verb"\hspace*{len}" is like \verb"\hspace{len}" but space is not removed at the
	beginning or end of a line.

\verb"\huge" switches to a very large typeface.
	\verb"\Huge" is even bigger.

\verb"\hyphenation{wordlist}" declares hyphenation as indicated;
	\verb"wordlist" contains words separated by spaces, with hyphens
	indicated (e.g. ``\verb"aard-vark cal-i-bra-tion"'').

%iii
\verb"\i" is \i.

\verb"\iff" is $\iff$ (math mode).

\verb"\Im" is $\Im$ (math mode).

\verb"\imath" is $\imath$ (math mode).

\verb"\in" is $\in$ (math mode).

\verb"\include{filename}" brings in \verb"filename" text at that point.

\verb"\includeonly{file1,file2,...}" limits recognition of \verb"\include"
	files.

\verb"\index{text}" appends \verb"text" to the \verb".idx" file
	by writing an \verb"\indexentry" command.
	
\verb"\indexentry{text}{ref}" is written to the \verb".idx" file for
	\verb"\index{text}" occuring at reference \verb"ref".

\verb"\indexspace" puts blank space before first index entry starting with
	a new letter.

\verb"\inf" is $\inf$ (math mode).

\verb"\infty" is $\infty$ (math mode).

\verb"\input{file}" brings in text from \verb"file.tex" at that point.

\verb"\int" is $\int$ (math mode).

\verb"\intextsep" --- vertical space placed above and below float in middle
	of text.

\verb"\iota" is $\iota$ (math mode).

\verb"\it" switches to {\it Italic\/} type.

\verb"\item[text]" indicates a list entry.  \verb"text" is optional, used in
	\verb"description" environment.

\verb"\itemindent" --- extra indentation before label in list item.  Default
	is \verb"0mm".

\verb"\itemsep" --- vertical space between successive list items.

%jjj
\verb"\j" is \j.

\verb"\jmath" is $\jmath$ (math mode).

\verb"\Join" is $\Join$ (math mode).

%kkk
\verb"\kappa" is $\kappa$ (math mode).

\verb"\ker" is $\ker$ (math mode).

\verb"\kill" --- in a \verb"\tabbing" environment, deletes previous line so
	tabs can be set without outputting text.

%lll
\verb"\l" is \l.  \verb"\L" is \L.

\verb"\label{text}" provides a reference point that is accessed with
	\verb"\ref{text}" or \verb"\pageref{text}".

\verb"\labelwidth" --- width of box containing list item label.

\verb"\labelsep" --- space between box containing list item label and text
	of the item.

\verb"\lambda" is $\lambda$. \verb"\Lambda" is $\Lambda$ (math mode).

\verb"\land" is $\land$ (math mode).

\verb"\langle" is $\langle$ (math mode).

\verb"\large", \verb"\Large", and \verb"\LARGE" switch to successively
	larger than \verb"\normalsize" type sizes.

\verb"\LaTeX" produces the \LaTeX{} logo.

\verb"\lbrace" is $\lbrace$ (math mode).

\verb"\lbrack" is $\lbrack$ (math mode).

\verb"\lceil" is $\lceil$ (math mode).

\verb"\ldots" makes three dots at the base of the line: \ldots
	(cf.\ \verb"\cdots").

\verb"\le" is $\le$ (math mode).

\verb"\leadsto" is $\leadsto$ (math mode).

\verb"\left*" (where \verb"*" is a delimiter) must be paired with
	\verb"\right*" (not necessarily using the same delimiter).
	`.' acts as a null delimiter (math mode).

\verb"\leftarrow" is $\leftarrow$. \verb"\Leftarrow" is $\Leftarrow$
	(math mode).

\verb"\lefteqn{formula}" is used in the \verb"eqnarray" environment to
	break a long \verb"formula" across lines.

\verb"\leftharpoondown" is $\leftharpoondown$ (math mode).

\verb"\leftharpoonup" is $\leftharpoonup$ (math mode).

\verb"\leftmargin", in \verb"list" environment, horizontal distance between left
	margin of enclosing environment and left margin of list.
	Settable for nesting levels 1 through 6, as \verb"\leftmargini"
	through \verb"\leftmarginvi".

\verb"\leftrightarrow" is $\leftrightarrow$. \verb"\Leftrightarrow" is
	$\Leftrightarrow$ (math mode).

\verb"\leq" is $\leq$ (math mode).

\verb"\lfloor" is $\lfloor$ (math mode).

\verb"\lg" is $\lg$ (math mode).

\verb"\lhd" is $\lhd$ (math mode).

\verb"\lim" is $\lim$ (math mode).

\verb"\liminf" is $\liminf$ (math mode).

\verb"\limsup" is $\limsup$ (math mode).

\verb"\line(x,y){len}" in \verb"picture" environment, in \verb"\put" command,
	draws line from \verb"\put" argument with length \verb"len" and
	slope \verb"(x,y)".

\verb"\linebreak[n]" forces a line to break exactly at this point, and
	adjusts line just terminated (cf.\ \verb"newline"). \verb"n" is
	optional: 0 is an optional break, 4 is a mandatory
	break, 1, 2 and 3 are intermediate levels of insistency.
\label{break-ref}

\verb"\linethickness{dimen}" sets the thickness for all lines in a
	\verb"picture".

\verb"\linewidth" is the width of the current line in a paragraph.

\verb"\listoffigures" begins a list of figures with heading.

\verb"\listoftables" begins a list of tables with heading.

\verb"\listparindent" --- extra indentation added to first line of every
	paragraph of an item after the first, in \verb"list" environment.

\verb"\ll" is $\ll$ (math mode).

\verb"\ln" is $\ln$ (math mode).

\verb"\lnot" is $\lnot$ (math mode).

\verb"\log" is $\log$ (math mode).

\verb"\longleftarrow" is $\longleftarrow$.
	\verb"\Longleftarrow" is $\Longleftarrow$ (math mode).

\verb"\longleftrightarrow" is $\longleftrightarrow$.
	\verb"\Longleftrightarrow" is $\Longleftrightarrow$ (math mode).

\verb"\longmapsto" is $\longmapsto$ (math mode).

\verb"\longrightarrow" is $\longrightarrow$.
	\verb"\Longrightarrow" is $\Longrightarrow$ (math mode).

\verb"\lor" is $\lor$ (math mode).

\verb"\lq" is a left-quote: \lq.

%mmm
\verb"\makebox[size][pos]{text}" creates a box of dimension \verb"size"
	containing \verb"text" at optional \verb"pos".
	\verb"\makebox(width,height)[pos]{text}" puts \verb"text" in a box;
	dimensions of box are \verb"width" and \verb"height";  \verb"text"
	is positioned at optional \verb"pos"
	(see \verb"positions" on page~\pageref{pos-ref}).

\verb"\makeglossary" enables writing of \verb"\glossaryentry" commands to
	a \verb".glo" file.

\verb"\makeindex" enables writing of \verb"\indexentry" commands to a
	\verb".idx" file.

\verb"\maketitle" produces a title with \verb"\title", \verb"\author",
	and, optionally, \verb"\date".

\verb"\mapsto" is $\mapsto$ (math mode).

\verb"\marginpar{text}" puts \verb"text" in the margin as a note.

\verb"\marginparpush" --- minimum amount of vertical space between two
	marginal notes.

\verb"\marginparsep" --- horizontal space between margin and marginal note.

\verb"\marginparwidth" --- width of a marginal note.

\verb"\markboth{lhd}{rhd}" defines the left-hand heading \verb"lhd" and
	the right-hand heading \verb"rhd" for the \verb"headings" and
	\verb"myheadings" page styles.

\verb"\markright{rhd}" defines the right-hand heading \verb"rhd" for the
	\verb"headings" and \verb"myheadings" page styles.

\verb"\max" is $\max$ (math mode).

\verb"\mbox{text}" places \verb"text" into a horizontal box.

\verb"\medskip" --- standard ``medium'' vertical skip.

\verb"\medskipamount" --- default length for \verb"\medskip".

\verb"\mho" is $\mho$ (math mode).

\verb"\mid" is $\mid$ (math mode).

\verb"\min" is $\min$ (math mode).

\verb"\mit" is ``math italic'' as in ${\mit \Pi}$ (math mode).

\verb"\models" is $\models$ (math mode).

\verb"\month" --- current month of the year.

\verb"\mp" is $\mp$ (math mode).

\verb"\mu" is $\mu$ (math mode).

\verb"\multicolumn{noc}{fmt}{text}" in \verb"tabular" environment puts
	\verb"text" across \verb"noc" columns using positioning format
	\verb"fmt" (\verb"c", \verb"r", \verb"l", and/or \verb"|").

\verb"\multiput("$x,y$\verb")("$\Delta x,\Delta y$\verb"){n}{obj}" is
	\ifsmallcrib \relax \else \linebreak \fi
	\verb"\put("$x,y$\verb"){obj}" \newline
	\verb"\put("$x+\Delta x,y+\Delta y$\verb"){obj}" \newline
	$\cdots$\newline
	\verb"\put("$x+(n-1)\Delta x,y+(n-1)\Delta y$\verb"){obj}".

%nnn
\verb"\nabla" is $\nabla$ (math mode).

\verb"\natural" is $\natural$ (math mode).

\verb"\ne" is $\ne$ (math mode).

\verb"\nearrow" is $\nearrow$ (math mode).

\verb"\neg" is $\neg$ (math mode).

\verb"\neq" is $\neq$ (math mode).

\verb"\newcommand{\cs}[narg]{def}" defines a new control sequence \verb"\cs"
	with definition \verb"def". Optionally, \verb"narg" is the number
	of arguments, indicated in \verb"def" as \verb"#1", \verb"#2",
	etc.

\verb"\newcounter{counter}[name]" defines a \verb"counter" optionally
	to be zeroed whenever the \verb"name" counter is incremented.

\verb"\newenvironment{envname}[narg]{def1}{def2}" defines a new environment,
	optionally with some number of arguments \verb"narg".  \verb"def1"
	is executed when the environment in entered and \verb"def2" is
	executed when it is exited.

\verb"\newfont{cs}{name}" defines a control sequence \verb"\cs" that
	chooses the font \verb"name".

\verb"\newlength{\nl}" sets up \verb"\nl" as a length of \verb"0in". See
	also \verb"\setlength", \verb"\addtolength", \verb"\settowidth".

\verb"\newline" breaks a line right where it is, with no stretching of
	terminated line (cf.\ \verb"\linebreak").

\verb"\newpage" ends a page where it appears. (cf.\ \verb"\clearpage").

\verb"\newsavebox{\binname}" declares a new \verb"bin" to hold a
	\verb"\savebox".

\verb"\newtheorem{env}[env2]{label}[sectyp]" defines a new theorem environment
	\verb"env" (optionally with the same numbering scheme as environment
	\verb"env2") with labels \verb"label".  Optionally, theorem numbers
	can be related to document section \verb"sectyp".

\verb"\ni" is $\ni$ (math mode).

\verb"\nofiles" suppresses writing of auxiliary files \verb".idx",
	\verb".toc", etc.

\verb"\noindent" suppresses indentation of first line of paragraph.

\verb"\nolinebreak[n]" prevents a line break at that point
	(cf.\ \verb"\linebreak" on page~\pageref{break-ref}).

\verb"\nonumber" is used in an \verb"eqnarray" environment to suppress
	equation numbering.

\verb"\nopagebreak[n]" prevents a page break at that point
	(cf.\ \verb"\linebreak" on page~\pageref{break-ref}).

\verb"\normalmarginpar" is default declaration for placement of marginal
	notes (cf.\ \verb"\reversemarginpar").

\verb"\normalsize" is the default type size for the document.

\verb"\not" puts a slash through a relational operator:
	\verb"\not=" is $\not=$ (math mode).

\verb"\notin" is $\notin$ (math mode).

\verb"\nu" is $\nu$ (math mode).

\verb"\nwarrow" is $\nwarrow$ (math mode).

%ooo
\verb"\o" is \o.  \verb"\O" is \O.

\verb"\obeycr" makes embedded carriage returns act like line terminators.

\verb"\oddsidemargin" --- distance between left side of page and text's normal
	left margin.

\verb"\odot" is $\odot$ (math mode).

\verb"\oe" is \oe.  \verb"\OE" is \OE.

\verb"\oint" is $\oint$ (math mode).

\verb"\omega" is $\omega$.  \verb"\Omega" is $\Omega$ (math mode).

\verb"\ominus" is $\ominus$ (math mode).

\verb"\onecolumn" sets text in single column (default) (cf.\ \verb"\twocolumn".

\verb"\opening{text}" declares an opening for \verb"letter" document style.

\verb"\oplus" is $\oplus$ (math mode).

\verb"\oslash" is $\oslash$ (math mode).

\verb"\otimes" is $\otimes$ (math mode).

\verb"\oval(x,y)" as an argument to \verb"\put" draws an oval x units
	wide and y units high.

\verb"\overbrace{text}" gives $\overbrace{text}$ (math mode).

\verb"\overline{text}" gives $\overline{text}$ (math mode).

\verb"\owns" is $\owns$ (math mode).

%ppp
\verb"\P" is \P.

\verb"\pagebreak[n]" forces a page break at that point
	(cf.\ \verb"\linebreak" on page~\pageref{break-ref}).

\verb"\pagenumbering{style}" determines page number style; \verb"style" may be
	\verb"arabic" (3), \verb"roman" (iii), \verb"Roman" (III),
	\verb"alph" (c), \verb"Alph" (C).

\verb"\pageref{text}" is the page number on which \verb"\label{text}" occurs.

\verb"\pagestyle{sty}" determines characteristics of a page's
	head and foot.  \verb"sty" may be \verb"plain" (page number only),
	\verb"empty" (no page number), \verb"headings" (running headings
	on each page), \verb"myheadings" (user headings).

\verb"\paragraph[toctitle]{text}" begins a new paragraph, automatically headed and
	numbered.  Optional \verb"toctitle" contains entry for the
	table of contents if different from \verb"text".

\verb"\paragraph*{text}" begins a paragraph and prints a title, but doesn't include
	a number or make a table of contents entry.

\verb"\parallel" is $\parallel$ (math mode).

\verb"\parbox[pos]{size}{text}" is a box created in paragraph mode. Text is
	positioned optionally at \verb"pos" (see \verb"positions" on
	page~\pageref{pos-ref}). Width is \verb"size".

\verb"\parindent" --- horizontal indentation added at beginning of paragraph.

\verb"\parsep" --- extra vertical space between paragraphs within a list item.

\verb"\parskip" --- extra vertical space between paragraphs, normally.

\verb"\part[toctitle]{text}" begins a new part, automatically headed and
	numbered.  Optional \verb"toctitle" contains entry for the
	table of contents if different from \verb"text".

\verb"\part*{text}" begins a part and prints a title, but doesn't include
	a number or make a table of contents entry.

\verb"\partial" is $\partial$ (math mode).

\verb"\partopsep" --- extra vertical space added before first list item if
	environment starts a new paragraph.

\verb"\perp" is $\perp$ (math mode).

\verb"\phi" is $\phi$. \verb"\Phi" is $\Phi$ (math mode).

\verb"\pi" is $\pi$. \verb"\Pi" is $\Pi$ (math mode).

\verb"\pm" is $\pm$ (math mode).

\verb"\pmod{modulus}" is ``parenthesized'' modulo expression
	$u \pmod{2^{e_j}-1}$ (math mode).

\verb"\poptabs" undoes the previous \verb"\pushtabs" command (restore
	prior tab settings).

\verb"positions", for boxing commands: \verb"t"=top, \verb"b"=bottom,
	\verb"h"=here, \verb"l"=left, \verb"c"=center, \verb"r"=right,
	\verb"p"=new page (\verb"figure" environment),
	\verb"p"=parbox (\verb"tabular" environment), .
\label{pos-ref}

\verb"\pounds" is \pounds.

\verb"\Pr" is $\Pr$ (math mode).

\verb"\prec" is $\prec$ (math mode).

\verb"\preceq" is $\preceq$ (math mode).

\verb"\prime" is $\prime$ (math mode).

\verb"\prod" is $\prod$ (math mode).

\verb"\propto" is $\propto$ (math mode).

\verb"\protect" permits the use of ``dangerous'' commands in
	\verb"@"-expressions, or in sectioning command and \verb"\caption"
	arguments.

\verb"\ps" in \verb"letter" document style permits additional text after
	\verb"\closing".

\verb"\psi" is $\psi$. \verb"\Psi" is $\Psi$ (math mode).

\verb"\pushtabs" in \verb"tabbing" environment lets you stack tab stop
	definitions.  Undo with \verb"\poptabs".

\verb"\put(x,y){stuff}" is the basic picture-drawing command.
	\verb"(x,y)" is the {\em reference point}, whose meaning varies
	for different \verb"stuff".  \verb"stuff" may be anything
	that goes in an \verb"\mbox".

%rrr
\verb"\raggedbottom" causes pages to assume natural height.

\verb"\raggedleft" declares all text that follows is to be flush against
	the right margin (cf.\ \verb"\begin{flushright}").

\verb"\raggedright" declares all text that follows is to be flush against
	the left margin (cf.\ \verb"\begin{flushleft}").

\verb"\raisebox{dim}[d2][d3]{text}" moves \verb"text" up by \verb"dim"
	(which may be negative). Optional \verb"d2" makes system think
	that \verb"text" extends \verb"d2" above the baseline (and optionally
	\verb"d3" below it).

\verb"\rangle" is $\rangle$ (math mode).

\verb"\rbrace" is $\rbrace$ (math mode).

\verb"\rbrack" is $\rbrack$ (math mode).

\verb"\rceil" is $\rceil$ (math mode).

\verb"\Re" is $\Re$ (math mode).

\verb"\ref{text}" is the section number in which \verb"\label{text}" occurs.

\verb"\renewcommand{\cs}[narg]{def}" redefines an old control sequence
	\verb"\cs" with definition \verb"def". Optionally, \verb"narg" is
	the number of arguments, indicated in \verb"def" as \verb"#1",
	\verb"#2", etc.

\verb"\renewenvironment{envname}[narg]{def1}{def2}" redefines an old new
	environment.  See \verb"\newenvironment".

\verb"\restorecr" undoes the \verb"\obeycr" command (makes carriage return
	a space-producing character).

\verb"\reversemarginpar" causes opposite margin to be used for marginal
	notes (e.g., left margin on odd-numbered pages).

\verb"\rfloor" is $\rfloor$ (math mode).

\verb"\rhd" is $\rhd$ (math mode).

\verb"\rho" is $\rho$ (math mode).

\verb"\right*" (where \verb"*" is a delimiter) must be paired with
	\verb"\left*" (not necessarily using the same delimiter).
	`.' acts as a null delimiter (math mode).

\verb"\rightarrow" is $\rightarrow$.
	\verb"\Rightarrow" is $\Rightarrow$ (math mode).

\verb"\rightharpoondown" is $\rightharpoondown$ (math mode).

\verb"\rightharpoonup" is $\rightharpoonup$ (math mode).

\verb"\rightleftharpoons" is $\rightleftharpoons$ (math mode).

\verb"\rightmargin" --- in \verb"list" environment, horizontal distance
	between right margin of enclosing environment and right margin of
	list. Default \verb"0in".

\verb"\rm" switches to {\rm Roman} type.

\verb"\roman{counter}" prints \verb"counter" in lower-case roman numerals.
	\verb"\Roman{counter}" prints upper-case roman numerals.

\verb"\rq" is a right-quote: \rq.

\verb"\rule[height]{length}{width}" makes a rectangular blob of ink
	\verb"length" long, \verb"width" wide, with optional \verb"height"
	above baseline.

%sss
\verb"\S" is \S.

\verb"\savebox{\binname}[width][pos]{text}" is exactly like \verb"\makebox"
	(q.v.), but saves box definition in bin \verb"\binname". Access
	with \verb"\usebox{\binname}".

\verb"\sbox{\binname}{text}" saves \verb"text" in box \verb"\binname" (see
	\verb"\savebox", above.).

\verb"\sc" switches to caps and small caps font.

\verb"\scriptsize" switches subscript size type.

\verb"\scriptstyle" switches to sub- or superscript-sized typesetting.
	\verb"\scriptscriptstyle" switches to second-level (very small)
	sub- or superscript-sized typesetting (math mode).

\verb"\searrow" is $\searrow$ (math mode).

\verb"\sec" is $\sec$ (math mode).

\verb"\section[toctitle]{text}" begins a new section, automatically headed and
	numbered.  Optional \verb"toctitle" contains entry for the
	table of contents if different from \verb"text".

\verb"\section*{text}" begins a section, prints a title, but doesn't include
	a number or make a table of contents entry.

\verb"\setcounter{counter}{value}" resets the value of \verb"counter".

\verb"\setlength{\nl}{length}" sets value of length command \verb"\nl"
	to \verb"length".  See also \verb"\addtolength", \verb"\newlength",
	\verb"\settowidth".

\verb"\setminus" is $\setminus$ (math mode).

\verb"\settowidth{\nl}{text}" sets value of length command \verb"\nl" to the
	width of \verb"text". See also \verb"\setlength", \verb"\newlength",
	\verb"\addtolength".

\verb"\sf" switches to {\sf sans serif} font.

\verb"\sharp" is $\sharp$ (math mode).

\verb"\shortstack[pos]{x\\yy\\zzz}" yields \shortstack{x\\yy\\zzz},
	a one-column tabular arrangement of its arguments.  Optional
	\verb"pos" can be \verb"l" or \verb"r" for text position.

\verb"\sigma" is $\sigma$. \verb"\Sigma" is $\Sigma$ (math mode).

\verb"\signature{text}" declares a signature for \verb"letter" document style.

\verb"\sim" is $\sim$ (math mode).

\verb"\simeq" is $\simeq$ (math mode).

\verb"\sin" is $\sin$ (math mode).

\verb"\sinh" is $\sinh$ (math mode).

\verb"\sl" switches to {\sl slanted\/} typeface.

\verb"\sloppy" relaxes the line-breaking algorithm to allow more or less
	distance between words.  Default is \verb"\fussy".

\verb"\small" switches to smaller than \verb"normalsize" typeface.

\verb"\smallint" is $\smallint$ (math mode).

\verb"\smallskip" --- standard ``small'' vertical skip.

\verb"\smallskipamount" --- default length for \verb"\smallskip".

\verb"\smile" is $\smile$ (math mode).

\verb"\spadesuit" is $\spadesuit$ (math mode).

\verb"\sqcap" is $\sqcap$ (math mode).

\verb"\sqcup" is $\sqcup$ (math mode).

\verb"\sqrt[3]{arg}" is $\sqrt[3]{arg}$.  \verb"3" (root) is optional.

\verb"\sqsubset" is $\sqsubset$ (math mode).

\verb"\sqsubseteq" is $\sqsubseteq$ (math mode).

\verb"\sqsupset" is $\sqsupset$ (math mode).

\verb"\sqsupseteq" is $\sqsupseteq$ (math mode).

\verb"\ss" is \ss.

\verb"\stackrel{stuff}{delim}" puts \verb"stuff" above the \verb"delim"iter;
	\verb"\stackrel{f}{\longrightarrow}"\ yields
	$\stackrel{f}{\longrightarrow}$ (math mode).

\verb"\star" is $\star$ (math mode).

\verb"\stop" --- type this if \TeX{} stops with a \verb"*" and no error
	message.

\verb"\subparagraph[toctitle]{text}" begins a subparagraphs, automatically headed and
	numbered.  Optional \verb"toctitle" contains entry for the
	table of contents if different from \verb"text".

\verb"\subparagraph*{text}" begins a subparagraph and prints a title, but doesn't include
	a number or make a table of contents entry.

\verb"\subsection[toctitle]{text}", \verb"\subsubsection[toctitle]{text}" begin new
	subsections, automatically headed and numbered.  Optional
	\verb"toctitle" contains entry for the table of contents if
	different from \verb"text".

\verb"\subsection*{text}", \verb"\subsubsection*{text}" begin subsections,
	but suppress section number and table of contents entry.

\verb"\subset" is $\subset$ (math mode).

\verb"\subseteq" is $\subseteq$ (math mode).

\verb"\succ" is $\succ$ (math mode).

\verb"\succeq" is $\succeq$ (math mode).

\verb"\sum" is $\sum$ (math mode).

\verb"\sup" is $\sup$ (math mode).

\verb"\supset" is $\supset$ (math mode).

\verb"\supseteq" is $\supseteq$ (math mode).

\verb"\surd" is $\surd$ (math mode).

\verb"\swarrow" is $\swarrow$ (math mode).

\verb"\symbol{cc}" produces the symbol (glyph) character code \verb"cc"
	in the current font.

%ttt
\verb"\t" prints a ``tie-after'' accent, as \t oo.

\verb"\tabbingsep" --- distance to left of a tab stop moved by \verb"\'".

\verb"\tabcolsep" --- half the width of the space between columns in
	\verb"tabular" environment.

\verb"\tableofcontents" produces a table of contents. A \verb".toc" file
	must have been generated during a previous \LaTeX{} run.

\verb"\tan" is $\tan$ (math mode).

\verb"\tanh" is $\tanh$ (math mode).

\verb"\tau" is $\tau$ (math mode).

\verb"\TeX" produces the \TeX{} logo.

\verb"\textfloatsep" --- distance between floats at the top
	or bottom of a single-column page and the text on that page.

\verb"\textfraction" --- minimum fraction of a text page that must contain text.

\verb"\textheight" is the normal vertical dimension of the body of the page.

\verb"\textstyle" switches to \verb"math" environment typesetting (math mode).

\verb"\textwidth" is the normal horizontal dimension of the body of the page.

\verb"\thanks{footnote}" adds an acknowledgement footnote to an author's
	name used in a \verb"\maketitle" command.

\verb"\theta" is $\theta$. \verb"\Theta" is $\Theta$ (math mode).

\verb"\thicklines" is an alternate line thickness for lines in a \verb"picture"
	environment.  See also \verb"linethickness".

\verb"\thinlines" is the default declaration for line thicknesses in a
	\verb"picture" environment.  See \verb"\thicklines".

\verb"\thinspace" is the proper space between single and double quotes, as
	in '\thinspace''.

\verb"\thispagestyle{sty}" determines characteristics of
	head and foot for the current page only.  Used to
	override \verb"\pagestyle" (q.v.) temporarily.

\verb"\tilde" makes a tilde, as: $\tilde a$ (math mode).

\verb"\times" is $\times$ (math mode).

\verb"\tiny" switches to a very small typeface.

\verb"\title{text}" declares a document title for the \verb"\maketitle" command.

\verb"\to" is $\to$ (math mode).

\verb"\today" generates today's date.

\verb"\top" is $\top$ (math mode).

\verb"\topfraction" --- maximum fraction at the top of a single-column
	page that may be occupied by floats.

\verb"\topmargin" --- space between top of \TeX{} page (1 inch from top of
	paper) and top of header.

\verb"\topsep" --- extra vertical space added before first list item and
	after last list item.

\verb"\topskip" --- minimum distance between top of page body to bottom
	of first line of text.

\verb"\triangle" is $\triangle$ (math mode).

\verb"\triangleleft" is $\triangleleft$ (math mode).

\verb"\triangleright" is $\triangleright$ (math mode).

\verb"\tt" switches to {\tt typewriter} type.

\verb"\twocolumn[text]" declares a two-column page, with optional full-page
	width heading \verb"text".

\verb"\typein[\cs]{text}" displays \verb"text" on the screen and waits for
	you to enter stuff which will be put in the document at that point.
	Optional control sequence \verb"\cs" can be assigned the value
	of your input, to be used later.

\verb"\typeout{text}" displays \verb"text" on the screen and writes it to
	the \verb".lis" file.

%uuu
\verb"\u" prints a breve accent, as \u o.

\verb"\unboldmath" unemboldens math italics and math symbols.  Should
	be used {\em outside} of math mode.

\verb"\underbrace{text}" gives $\underbrace{text}$ (math mode).

\verb"\underline{text}" gives \underline{text} (math mode or not).

\verb"\unitlength" --- length of coordinate units for \verb"picture"
	environment.

\verb"\unlhd" is $\unlhd$ (math mode).

\verb"\unrhd" is $\unrhd$ (math mode).

\verb"\uparrow" is $\uparrow$. \verb"\Uparrow" is $\Uparrow$ (math mode).

\verb"\updownarrow" is $\updownarrow$. \verb"\Updownarrow" is $\Updownarrow$
	(math mode).

\verb"\uplus" is $\uplus$ (math mode).

\verb"\upsilon" is $\upsilon$. \verb"\Upsilon" is $\Upsilon$ (math mode).

\verb"\usebox{\binname}" recalls box definition saved in box
	\verb"\binname".

\verb"\usecounter{counter}" is used in a \verb"list" environment to cause
	\verb"counter" to be used to number the items.

%vvv
\verb"\v" prints a h\'a\v cek, as \v o.

\verb"\value{counter}" produces the numeric value of \verb"counter".

\verb"\varepsilon" is $\varepsilon$ (math mode).

\verb"\varphi" is $\varphi$ (math mode).

\verb"\varpi" is $\varpi$ (math mode).

\verb"\varrho" is $\varrho$ (math mode).

\verb"\varsigma" is $\varsigma$ (math mode).

\verb"\vartheta" is $\vartheta$ (math mode).

\verb"\vdash" is $\vdash$ (math mode).

\verb"\vdots" is $\vdots$ (math mode).

\verb"\vec" puts a vector over a letter: $\vec a$ (math mode).

\verb"\vector(x,y){len}" in \verb"picture" environment, in \verb"\put"
	command, draws vector from \verb"\put" argument with length \verb"len"
	and slope \verb"(x,y)", with arrowhead.

\verb"\vee" is $\vee$ (math mode).

\verb"\verb/text/" creates a local \verb"verbatim" environment for \verb"text",
	printed in \verb"typewriter" font. Note that \verb"text"
	is {\em not} in curly braces; it is between two identical delimiters,
	neither of which appears in \verb"text".

\verb"\verb*/text/" is like \verb"\verb/text/", but spaces print out
	as \verb*" ".

\verb"\vert" is $\vert$. \verb"\Vert" is $\Vert$ (math mode).

\verb"\vfill" is \verb"\vspace{\fill}" (cf. \verb"\fill").

\verb"\vspace{len}" leaves a vertical space of dimension \verb"len".

\verb"\vspace*{len}" is like \verb"\vspace{len}" but space is not removed at the
	beginning or end of a page.

%www
\verb"\wedge" is $\wedge$ (math mode).

\verb"\widehat{arg}" is $\widehat{arg}$ (math mode).

\verb"\widetilde{arg}" is $\widetilde{arg}$ (math mode).

\verb"\wp" is $\wp$ (math mode).

\verb"\wr" is $\wr$ (math mode).

%xxx
\verb"\xi" is $\xi$. \verb"\Xi" is $\Xi$ (math mode).

%yyy
\verb"\year" --- current year (A.D.).

%zzz
\verb"\zeta" is $\zeta$ (math mode).

\endinput
